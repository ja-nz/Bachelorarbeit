\section{Ausgangssituation}\label{ausgangssituation}

Softwareentwicklung für den Browser ist ohne Zweifel eine komplexe
Angelegenheit. Die Gründe für diesen Umstand sind sehr vielfältig.
Einige Probleme sind ``hausgemacht'' wie beispielsweise Designschwächen
der de facto einzigen Zielsprache aller Browser JavaScript. Andere
Probleme sind einfach dem Umstand der rasanten Evolution im Internet von
statischen Seiten hin zu dynamischen Web Apps geschuldet.

Das Resultat dieser Entwicklungen sind gravierend. Die Anzahl der
Frameworks, Werkzeuge und Bibliotheken ist schier unüberblickbar
geworden und wandelt sich in einer Geschwindigkeit, die Entwicklern in
der JavaScript Welt Schwierigkeiten bereitet.\footnote{https://hackernoon.com/how-it-feels-to-learn-javascript-in-2016-d3a717dd577f}
Aber auch die andere Seite, die Konsumenten der Webseiten, bekommen
diese Probleme spüren. Frederic Filloux zeigte in einem Blogpost, dass
sich in einem 4667 Buchstaben langen Zeitungsartikel der britischen
Zeitung ``The Guardian'' 485527 Buchstaben an Quelltext
verstecken.{[}1{]}

\section{Zielsetzung}\label{zielsetzung}

Ziel dieser Arbeit ist die systematische Erfassung der neuen
Technologien sowie eine praktische Erläuterung möglicher
Architekturmodelle.

\section*{Vorgehensweise}\label{vorgehensweise}
\addcontentsline{toc}{section}{Vorgehensweise}

\hypertarget{refs}{}
\hypertarget{ref-Filloux2016}{}
{[}1{]} F. Filloux, ``Bloated html, the best and the worse,'' 2016.
