\documentclass[]{article}
\usepackage{lmodern}
\usepackage{amssymb,amsmath}
\usepackage{ifxetex,ifluatex}
\usepackage{fixltx2e} % provides \textsubscript
\ifnum 0\ifxetex 1\fi\ifluatex 1\fi=0 % if pdftex
  \usepackage[T1]{fontenc}
  \usepackage[utf8]{inputenc}
\else % if luatex or xelatex
  \ifxetex
    \usepackage{mathspec}
  \else
    \usepackage{fontspec}
  \fi
  \defaultfontfeatures{Ligatures=TeX,Scale=MatchLowercase}
\fi
% use upquote if available, for straight quotes in verbatim environments
\IfFileExists{upquote.sty}{\usepackage{upquote}}{}
% use microtype if available
\IfFileExists{microtype.sty}{%
\usepackage{microtype}
\UseMicrotypeSet[protrusion]{basicmath} % disable protrusion for tt fonts
}{}
\usepackage[unicode=true]{hyperref}
\hypersetup{
            pdftitle={Browsernative Microservices},
            pdfauthor={Jan Peteler, FH Würzburg-Schweinfurt, jan.peteler@student.fhws.de},
            pdfborder={0 0 0},
            breaklinks=true}
\urlstyle{same}  % don't use monospace font for urls
\usepackage{color}
\usepackage{fancyvrb}
\newcommand{\VerbBar}{|}
\newcommand{\VERB}{\Verb[commandchars=\\\{\}]}
\DefineVerbatimEnvironment{Highlighting}{Verbatim}{commandchars=\\\{\}}
% Add ',fontsize=\small' for more characters per line
\newenvironment{Shaded}{}{}
\newcommand{\KeywordTok}[1]{\textcolor[rgb]{0.00,0.44,0.13}{\textbf{{#1}}}}
\newcommand{\DataTypeTok}[1]{\textcolor[rgb]{0.56,0.13,0.00}{{#1}}}
\newcommand{\DecValTok}[1]{\textcolor[rgb]{0.25,0.63,0.44}{{#1}}}
\newcommand{\BaseNTok}[1]{\textcolor[rgb]{0.25,0.63,0.44}{{#1}}}
\newcommand{\FloatTok}[1]{\textcolor[rgb]{0.25,0.63,0.44}{{#1}}}
\newcommand{\ConstantTok}[1]{\textcolor[rgb]{0.53,0.00,0.00}{{#1}}}
\newcommand{\CharTok}[1]{\textcolor[rgb]{0.25,0.44,0.63}{{#1}}}
\newcommand{\SpecialCharTok}[1]{\textcolor[rgb]{0.25,0.44,0.63}{{#1}}}
\newcommand{\StringTok}[1]{\textcolor[rgb]{0.25,0.44,0.63}{{#1}}}
\newcommand{\VerbatimStringTok}[1]{\textcolor[rgb]{0.25,0.44,0.63}{{#1}}}
\newcommand{\SpecialStringTok}[1]{\textcolor[rgb]{0.73,0.40,0.53}{{#1}}}
\newcommand{\ImportTok}[1]{{#1}}
\newcommand{\CommentTok}[1]{\textcolor[rgb]{0.38,0.63,0.69}{\textit{{#1}}}}
\newcommand{\DocumentationTok}[1]{\textcolor[rgb]{0.73,0.13,0.13}{\textit{{#1}}}}
\newcommand{\AnnotationTok}[1]{\textcolor[rgb]{0.38,0.63,0.69}{\textbf{\textit{{#1}}}}}
\newcommand{\CommentVarTok}[1]{\textcolor[rgb]{0.38,0.63,0.69}{\textbf{\textit{{#1}}}}}
\newcommand{\OtherTok}[1]{\textcolor[rgb]{0.00,0.44,0.13}{{#1}}}
\newcommand{\FunctionTok}[1]{\textcolor[rgb]{0.02,0.16,0.49}{{#1}}}
\newcommand{\VariableTok}[1]{\textcolor[rgb]{0.10,0.09,0.49}{{#1}}}
\newcommand{\ControlFlowTok}[1]{\textcolor[rgb]{0.00,0.44,0.13}{\textbf{{#1}}}}
\newcommand{\OperatorTok}[1]{\textcolor[rgb]{0.40,0.40,0.40}{{#1}}}
\newcommand{\BuiltInTok}[1]{{#1}}
\newcommand{\ExtensionTok}[1]{{#1}}
\newcommand{\PreprocessorTok}[1]{\textcolor[rgb]{0.74,0.48,0.00}{{#1}}}
\newcommand{\AttributeTok}[1]{\textcolor[rgb]{0.49,0.56,0.16}{{#1}}}
\newcommand{\RegionMarkerTok}[1]{{#1}}
\newcommand{\InformationTok}[1]{\textcolor[rgb]{0.38,0.63,0.69}{\textbf{\textit{{#1}}}}}
\newcommand{\WarningTok}[1]{\textcolor[rgb]{0.38,0.63,0.69}{\textbf{\textit{{#1}}}}}
\newcommand{\AlertTok}[1]{\textcolor[rgb]{1.00,0.00,0.00}{\textbf{{#1}}}}
\newcommand{\ErrorTok}[1]{\textcolor[rgb]{1.00,0.00,0.00}{\textbf{{#1}}}}
\newcommand{\NormalTok}[1]{{#1}}
\usepackage{longtable,booktabs}
% Fix footnotes in tables (requires footnote package)
\IfFileExists{footnote.sty}{\usepackage{footnote}\makesavenoteenv{long table}}{}
\IfFileExists{parskip.sty}{%
\usepackage{parskip}
}{% else
\setlength{\parindent}{0pt}
\setlength{\parskip}{6pt plus 2pt minus 1pt}
}
\setlength{\emergencystretch}{3em}  % prevent overfull lines
\providecommand{\tightlist}{%
  \setlength{\itemsep}{0pt}\setlength{\parskip}{0pt}}
\setcounter{secnumdepth}{0}
% Redefines (sub)paragraphs to behave more like sections
\ifx\paragraph\undefined\else
\let\oldparagraph\paragraph
\renewcommand{\paragraph}[1]{\oldparagraph{#1}\mbox{}}
\fi
\ifx\subparagraph\undefined\else
\let\oldsubparagraph\subparagraph
\renewcommand{\subparagraph}[1]{\oldsubparagraph{#1}\mbox{}}
\fi

% set default figure placement to htbp
\makeatletter
\def\fps@figure{htbp}
\makeatother


\title{Browsernative Microservices}
\providecommand{\subtitle}[1]{}
\subtitle{Modular web architecture through new W3C specifications}
\author{Jan Peteler, FH Würzburg-Schweinfurt, jan.peteler@student.fhws.de}
\date{Januar 2017}

\begin{document}
\maketitle
\begin{abstract}
Building complex web applications nowadays require additional layers of
abstraction and often heavily depend on indispensable frameworks. While
they perfectly circumstance the global paradigm of the DOM they remain
highly proprietary attached to the framework. New w3c specifications
build right into the browser provide a native service API in the need to
create standardized web components for the browser. As of 2017 all major
browsers will ship those technologies in their browsers. The paper not
only provides an introduction to the specification. As an assessment
criteria of it will match those specifications against the concept of
microservices which is an modular system architecture to build scalable
applications.
\end{abstract}

{
\setcounter{tocdepth}{3}
\tableofcontents
}
\section{1. Simplicity and the web}\label{simplicity-and-the-web}

\begin{quote}
Simplicity is prerequisite for reliability. - Edsger W. Dijkstra
\end{quote}

Computers can scale, humans can't. Ever since complex systems made by
humans have been constrained by humans mental capabilities. Like in the
analogy of juggling balls our brain can just ``juggle'' a few items at a
time. Rich Hickey, the inventor of the programming language Clojure gave
an inspirational keynote on the topic of \textbf{simplicity}.{[}1{]} In
every sphere of a human's life simplicity aligns perception with mental
capacities.

Derived from the Latin word \textbf{simplex}, \emph{simple} can be
understood as ``literally, uncompounded or onefold''\footnote{\href{http://www.etymonline.com/index.php?term=simple}{Etymology
  Dictionary}} which points towards an unidimensional state. While
complexity describes the multilayered und entangled nature of conditions
simplicity empowers the human brain to reason about issues
straightforward. It certainly has some overlappings with \emph{easy},
but while \emph{easy} is more of a relative spirit, \emph{simple} can be
laid out as an objective manner and therefore universally applicable.

Software development is undoubtedly rich in complexity and full of
subtle pitfalls. In a typical scenario a growing software project
evolves in one or another opinionated direction over time. Layers of
abstractions wrestle against aged legacy code requiring additional
middleware. Mutating assets create subtle bugs and so forth. Eventually
the small piece of software may end up in a highly complected monolith
which will determine future design decisions to a painful degree.
Opaqueness of the system will slow down innovation to a minimum in the
need of `keeping the lights on'.

On the other side of this dystopian scenario a truly modular system
architecture abandons many of those potential inconsistencies. The whole
system is divided in pluggable parts, object mutation is either
traceable or avoided altogether in favor of immutable data structures.
As Rich Hickey argues, design decisions should be made under the
\textbf{impression of extending, substitution, moving, combining and
repurposing}.{[}1{]} The ability to reason about the program at any
given time is crucial for future decisions and implementations,
recalling the unidimensional nature of simplicity.

Simplicity in `the web', read as a loose generalization of `everything
that runs in the browser' is certainly a story full of misconceptions.
While simplicity in the backend is mostly a matter of principles and
patterns, any browser-based frontend is restricted on the highly
deterministic nature of the browser platform.

In the last four years the average transfer size of a webpage doubled to
currently around 2.5 MB.\footnote{\href{http://httparchive.org/trends.php}{HTTPArchive
  Trends}} Leaving images, fonts or other content aside, the size of
HTML, CSS and JS sums up to a total average of 550 KB. One character
weights around 1 byte which means an average webpage is delivering
550.000 characters or around 125 pages of single-spaced text. Frederic
Filloux analyzed the payload on different newspaper websites and came to
the conclusion, that only roundabout 5-6 \% of the transferred
characters are made for human consumption.{[}2{]}

Having a 95 \% overhead is rather undesirable for both the consumers and
the creators of the website. Since it is a widespread problem without a
definable point of failure one can argue `the platform itself' is the
failure. By design, every pageload results in a monolithic DOM tree
managed by the browser engine. Whether rendering just a bunch of static
text nodes or an ever changing webapp, the underlying global nature of
the DOM tree remains the same. Every additional piece of code added to
the webpage will invisibly add another fold of complexity to this global
object.

In a non-deterministic runtime environment, encapsulation and
modularization is a typical pattern to make complexity manageable and to
accommodate future uncertainty.{[}3, p. 1{]} Since years the average JS
payload is steadily rising which can be interpreted as a trend towards
more dynamic websites. The demands to the browser platform have changed
from a static page renderer to a \textbf{dynamic UI machine} without
significantly changing the underlying architecture. Under the current
situation only additional layers of abstraction can handle complexity.

In the recent years many \textbf{frameworks}, libraries and
methodologies approached the global nature of the DOM by scoping assets
and design rules into maintainable components. While the DOM cannot be
scoped, JS can. Many frameworks like React, Angular or Vue, just to name
a few, ditched the old rule of separated HTML, CSS and JS in favor of an
additional layer of abstracted JS components. Quite often those
frameworks mimic a MVC pattern on top of the browser engine which is a
reasonably simple pattern to build UIs. While frameworks are a valid
approach for building scalable web applications they remain highly
opinionated, embody inherent complexity themselves and can change and
break over time. Another drawback is code inflation which is a crucial
point for performance. All of those bottlenecks in the web demand for a
new standardized ways for creating and evolving complex web services.

In the year 2013 thinkers, creators and browser vendors joined together
to propose \emph{The Extensible Web Manifesto}.\footnote{\href{https://extensiblewebmanifesto.org/}{The
  Extensible Web Manifesto}} The claim of the manifesto was to enhance
the current web platforms with new low-level capabilities. Those
capabilities aimed to empower creators of the web to write more
declarative code and to abandon artificial abstractions. Four years
later, the enhancement of JavaScript leapfrogged and many new low-level
APIs were brought to life. With this new APIs at hand, a progressive web
developer can create robust websites with less code and less additional
libraries. This paper is an approach to unfold these
\textbf{browsernative} technologies to create overall simple and
resilient \textbf{microservices} for the browser

\section{2. Microservices}\label{microservices}

In search of a better, simpler web architecture one might look on
already established patters proofed to fulfill enterprise needs.
Microservices are a good approach for tearing big monolithic systems
into fine-grained simple services with explicit defined boundaries. In a
nutshell a microservice is a small, autonomous service that works
together with other services seamlessly.{[}4, p. 2{]} Or with the words
of Fowler and Lewis: ``\ldots{} the microservice architectural style is
an approach to developing a single application as a suite of small
services, each running in its own process and communicating with
lightweight mechanisms, often an HTTP resource API.''{[}5{]} Yet at this
point a reader might spot some similarities with microservices and the
browser-based development: Both wrestle with the problem of monolithic
architecture and both use lightweight communication mechanisms. In fact,
many big companies of `the web' like Amazon or Netflix successfully
transformed their monolithic system into a service based system which
gives a taste of the power behind microservices.{[}5{]}

Microservices incorporate a wide array of ideas from developing scalable
software like \textbf{domain-driven design} to pursuing the real world
structure in the code.{[}4, p. 2{]} Another concept focuses on
\textbf{continuous delivery} for pushing software rapidly through
\textbf{automated deployment} mechanisms into production.{[}5{]}
Furthermore, microservices transcend the technical perspective and reach
into team organization.

As a primary source of comprehensive information this paper relies on
the work of Sam Newman{[}4{]} and the work of Fowler and Lewis{[}5{]}.
The purpose of this section is to gain confidence about the architecture
of microservices in the context of the browser platform.

\subsection{2.1. Componentization via
Services}\label{componentization-via-services}

``A~\textbf{component}~is a unit of software that is independently
replaceable and upgradeable.'' {[}5{]} Components are the building
blocks of microservices. And microservices are the building blocks of
applications. Essentially the difference between microservices and
components is just the level of abstraction. Whether a concrete
microservice or a much more generic component, both share a similar set
of principles. Therefore this paper refers to both parts when talking
about \textbf{services}.

The first principle of services is the \textbf{loose coupling
principle}: changing and deploying one service should not result in
changing other parts of the system.{[}4, p. 30{]} Picking CSS for
example, where variables tend to be shadowed by higher specific values
making it increasingly difficult to keep changes, ought to merely affect
one place in the application. Therefore a \emph{browsernative
microservice} is expected to push encapsulation and to avoid variable
mutations outside its scope as much as possible. One solution would be
scoped CSS / JS to avoid variables leaking into the global namespace, as
well as to hide implementation details in order to avoid mutating
values.

The second principle of services is the \textbf{high cohesion
principle}: Whether designing a microservice or its components, we want
related behavior sit together and unrelated behavior to sit
elsewhere.{[}4, p. 30{]} High cohesion can be enhanced towards the more
dynamic \textbf{Single Responsibility Principle}: ``Gather together
those things that change for the same reason and separate those things
that change for different reasons.''{[}6{]} In a very quick and dirty
code quality analysis, the quality can be measured just by counting the
code changes which occur in order to implement a new functionality. An
arbitrary threefold MVC system should require a maximum of three changes
to implement or change functionalities. The problem in browser based
development is not only the global paradigm which makes changes
deliberately unpredictable. The high cohesion principle is violated by
the traditional separation along the siloed entities HTML, JS and CSS.
Understanding the relation of HTML markup towards another CSS file
generates incidental complexity. Different approaches emerged over the
recent years to join these entities. A \emph{browsernative service}
sought for a combination of the web native trinity HTML, JS and CSS.

Traditional web development relies on libraries to enhance the service
capabilities of the web platform. Compared to libraries a component
service offers multiple advantages for building, deployment and
shipping. As expressed earlier \emph{components for the web}, or web
components, are self-contained which means they embody all needed
functionality to get their job done. Therefore they have a much better
evolution mechanism in the service contracts. Changing functionality
will not break other services. A component can be progressively enhanced
which guarantees functionality throughout different versions whereas a
library is only loosely coupled to the implementation and therefore hard
to track in functionality. Changing a library may result in an
unforeseen amount of time fixing implementations. It is not unusual to
see websites embodying different versions of the same library to
guarantee functionality which lowers page performance significantly.

Another issue where web components stand out is related to performance
and especially the critical first page rendering. Libraries for the
browsers are traditionally `shipped' as non static immediately-invoked
functions. Following the Google RAIL model, a user-centric performance
measurement, a page load ought to take less than 1 second to catch up
users' attention.{[}7{]} There are many ways to optimize the critical
first render but as a rule of thumb a build-in web component might be
always superior to libraries in terms of first rendering. The
fine-grained lifecycle methods which will be described in the technical
section of this paper give the developer far reaching optimization
opportunities.

The last argument in favor of components over libraries is the more
explicit interface.{[}5{]} While the functionality of a library needs
documentation to be accessible, a component functionality is exposed via
the components' signature. HTML markup is an expressive syntax and
therefore convenient for steering a web component using merely
attributes and values.

\subsection{2.2. Organized around Business
Capabilities}\label{organized-around-business-capabilities}

\begin{quote}
``organizations which design systems \ldots{} are constrained to produce
designs which are copies of the communication~structures~of these
organizations''. {[}8{]}
\end{quote}

Emphasizing the human factor in microservices is a key feature.
Microservices are a product of real-world usage.{[}4, p. 1{]} Instead of
splitting the team along the technology stack (UI Experts
-\textgreater{} Middleware -\textgreater{} Database) a microservice
approach models teams around \textbf{business capabilities}.{[}5{]}
Consequently every team is capable of planning, designing, implementing,
testing and maintaining their very own microservice. Along the
technology stack every member gains high competence in the service
architecture which can be positive for evolving the service over its
whole lifecycle.

Real-world domains tend to be complex and multifaceted. To unfold their
complexity, domains can be subdivided into \textbf{bounded
context}.{[}4, p. 31{]} For example, customer service is a business
domain but with varying bounded contexts. One context can be sales,
another context could be support etc. Every context makes different
assumptions about the underlying model and draws an explicit interface
where it decides what to share with other contexts.{[}4, p. 30{]}
Evaluating each bounded context within each business domain will
eventually shape the data persistency model likewise the interface to
the service. This methodology can be iterated over and over again. A
sales service for example might be evaluated to different sales contexts
resulting in differing interfaces on differing devices shaped by
browsernative microservices. By dividing the service into clear defined
business boundaries it becomes easier to define a smart API of the
service.

Assigning service responsibility to a team, the so called
\textbf{Definition of Done} shifts from `accomplishing projects' to
`accomplishing products'. This new paradigm not only changes the
administrative overhead like budgeting or resource allocation.
Furthermore, it creates a kind of responsibility connection from the
team to the service which can be best described as \textbf{shared
governance} model. ``Each team collectively share responsibility for
evolving the technical vision of the system.''{[}4, p. 247{]}
Expectedly, those teams are more motivated within their very own service
and exhibit a more sophisticated iteration time.{[}5{]}

For many companies working in the spheres of the internet, the client
side is highly important for their business. In fact, business goals and
capabilities can be derived from frontend needs. The state of the web is
not only a story of numerous artifacts, it is also a story of a highly
fragmented market along devices, operating systems, differing sizes and
functionalities. Different devices again have different assumptions
about the technology stack. Splitting teams along the stack results in
an slow paced back and forth negotiation about every change to be made
can result in prolonged and therefor expensive iteration. As browser
technologies, design guidelines and devices change rapidly it makes
sense to shift responsibility towards the team altogether.

\subsection{2.3. Smart endpoints and dumb
pipes}\label{smart-endpoints-and-dumb-pipes}

To ensure the microservice functionality among teams and different
services requires thoughtful decentralization. Emphasizing once more the
real-world capabilities of microservices a message channel architecture
can be derived from patterns known from traditional postal services. A
physical letter has only two smart endpoints, entitled to read and
process the message while packaging as well as the connection itself is
mostly standardized. \textbf{Smart endpoints and dumb pipes} is coined
to the approach of designing communication mostly decoupled and as
cohesive as possible.{[}5{]} Applying these rules to the web platform
can result into building unified JSON message objects passed along
`dumb' middleware components. A typical browser event comes close to
this definition and suits arguably well for in-memory communication
between web components and microservices.

Microservices heavily rely on simple HTTP request-response with resource
APIs and lightweight messaging.{[}5{]} Newman recommends
technologic-agnostic REST APIs to free data persistence from
implementation constraints.{[}4, p. 247{]} The advantage of this overall
simple communication model is the suitability for both frontend-backend
likewise backend-backend communication. A service therefore can evolve
from a heavy backend with a lot of network roundtrips to a leaner
backend seamlessly. The browser build-in \textbf{fetch API} which is
essentially a HTTP request can be heavily incorporated into a
browsernative microservice to ensure communication to services in the
backend.

\subsection{2.4. Decentralized
Governance}\label{decentralized-governance}

Microservices are separate entities and decentralization is important to
ensure autonomy. This paper already described fragmented services
bounded to singular business context, choreographed by simple
communication protocols developed and evolved by autonomous teams.

This distributed nature empowers teams to create their own technology
stack, tools and services designed in the spirit of language- and
platform independence and to share their knowledge across other
parties.{[}5{]} In the recent years many big companies like Facebook,
Google, Netflix and others followed that spirit and published their
ideas and implementations open source. The previously mentioned React
for example is a product of Facebook's need to ensure a consistent
frontend experience. In fact, many tools and techniques are byproduct of
vital interaction of specific domain problems and their solutions.

The spirit of freedom cannot be applied universally to
\emph{browsernative microservices} as the browser and its underlying DOM
will, to a certain degree, be the limiting factor. Talking about the
browser, a reader might be falsely tempted to narrow his perception
towards the obvious VIEW layer only. In recent years the major browser
engines have grown to fully-fledged app deployment platforms offering
connectors to build-in databases, multithreading support and
ever-growing JS build-ins like speech synthesis or push notifications.
So-called \textbf{Progressive Web Apps}\footnote{\href{https://developers.google.com/web/progressive-web-apps}{Progressive
  Web Apps}}, a bunch of criteria for building good browser apps, can
achieve a similar look and feel like native apps. And last but not least
services like NativeScript\footnote{\href{https://www.nativescript.org}{NativeScript}
  \#\# 2.5. Decentralized Data Management} effectively compiling `the
web' to native machine code lower the boundary between native and
browser code even further.

JS is the widely accepted language of the web. Nevertheless, a
microservice engineering team might choose another language for various
reasons. Transpiling languages to JS as target language is a stable
solution nowadays. Languages like TypeScript, ClojureScript or
PureScript compile to JS even exclusively. Once web components hit a
critical mass there will be most likely some library support or foreign
function interface towards ES6 modules, which are mandatory in the new
specifications.

Another more real life decentralization aspect derives from the easiness
of deployment in a safe, sandboxed environment. Web components virtually
ship no overhead or require dedicated build tools. This makes them ideal
candidates for sharing and open source publishing similar to the largest
JS package registry NPM. In the spirit of NPM web components can be
perceived as frontend packages with an HTML interface instead of a JS
signature. Webcomponents.org is a registry for ready-to-use components
of every scale and purpose where even Google shares a lot of their
material design elements.

Data Management in a microservice follows the same modular philosophy as
the service implementation. As mentioned before, different bounded
contexts make different assumptions of the underlying models.
Decentralized decisions about conceptual models demand decentralized
data storage decisions.{[}5{]} Todays web architectures aim to leverage
an increasing amout of \textbf{processing to the client} to avoid
time-consuming roundtrips especially in mobile networks.{[}9{]} Since
network roundtrips are costly it is a good advice to only query as much
data as needed and cache as much as possible. The build-in LocalStorage
or its successor IndexedDB are mature persistence technologies and
libraries like PouchDB\footnote{\href{https://pouchdb.com/}{PouchDB}}
even offer adapters for syncing to the server out of the box.

``Microservices prefer letting each service manage its own
database.''{[}Fowler2014{]} Ben Issa, chief architect of ING Australia
emphasizes this pragmatism on APIs in a conference talk. At ING the
frontend demands tailor the backend APIs. APIs may be produced
automatically and not even Issa knows how many APIs exists.{[}10{]} At
ING, they are using a pattern called \textbf{backends for frontends}
empowering the team to craft their UI and backend in a one-to-one
relationship.{[}4, p. 72{]}

To see this pattern in the field a reader might have a look at
Facebook's GraphQL\footnote{\href{http://graphql.org/}{GraphQL}}.
GraphQL is a query language for the frontend. The backend solely replies
to the frontend needs. Another well documented example in the field is
Cognitects Datomic\footnote{\href{http://www.datomic.com/}{Datomic}},
where parts of the database will be reflected to the client. A so-called
Transactor ensures ACID compliance.

The simplified microservice example later in this paper assumes a
generic build-in API accompanied by build-in frontend components.
Instead of gluing frontend and backend together on runtime the
microservice is designed holistically containing both front- and
backends. For the sake of simplicity data management will not be
explored into depth throughout this paper.

\subsection{2.6. Infrastructure
Automation}\label{infrastructure-automation}

Microservices tend to increase complexity as this model adds a sheer
number of moving parts to the system which requires proper
orchestration.{[}4, p. 246{]} Arguably every sophisticated web developer
already came across build tools like Webpack or infrastructure
automation tools like Gulp. Therefore, testing and deploying web
components should not be an obstacle in development.

In the global nature of web development the development could not be
completely decoupled from the production environment. This circumstance
left developers switching back and forth between files developing tricky
opinionated (and more often biased) ways to glue related parts together.
Bret Victor, UI designer at Apple defined the importance of an
\textbf{immediate feedback principle} for developing user
interfaces.{[}11{]} In his talk he emphasizes the importance of an
immediate connection between the creator of a product and the product
itself. Any change must result in an immediate visible feedback. Web
components catch up with this principle as they allow isolated
development within a single file containing all bits and pieces of a
single service. Every major browser devtool offers a direct file
manipulation functionality, so development can be even in place.

When it comes to standardized development previously mentioned, Ben Issa
described the ING standard workflow as follows. Every component is
packed in its own \textbf{git repo} containing:

\begin{itemize}
\tightlist
\item
  Internationalization conformity (i18n)
\item
  Accessibility conformity (a11y)
\item
  Tests for the component
\item
  Demos of the component
\item
  Blueprints to mock the one to one APIs
\item
  Docs
\end{itemize}

Even though this example is an opinionated perception it gives a sense
of a mature component built for the web. This example should illustrate
that all parts of the component put together in one place as well as
tests, demos and blueprints are part of the component from day one.
Every check-in is handled as a release candidate and can be
independently tested and deployed by a fully automated
machinery.{[}10{]} Due to an exhaustive amount of testing and deployment
tools for JS an automated infrastructure should not be an obstacle.

\subsection{2.7. Design for failure}\label{design-for-failure}

In theory a microservice is designed with focus on monitoring of both
the architectural elements and business relevant metrics.{[}5{]} Due to
the modular structure, weak points can occur in the orchestration of the
services. A microservice should track down every communication flow and
provide defaults and meaningful error messages where communication might
stuck. Testing every single component with predefined synthetic events
ensures functionality. Nevertheless, browser support may vary and legacy
browsers remain a general problem for enhancing websites with new
technologies and therefore demand further configuration.

Combination of different resources in the browser always demand
optimization to avoid unexpected side-effects like \emph{flash of
unstyled content}. Googles Polymer propagates a general-purpose pattern
called \textbf{PRLP}\footnote{\href{https://www.polymer-project.org/1.0/toolbox/server}{PRLP
  pattern}}:

\begin{itemize}
\tightlist
\item
  Push critical resources for the initial route
\item
  Render initial route
\item
  Pre-cache remaining routes
\item
  Lazy-load and create remaining routes on demand
\end{itemize}

Following this pattern a critical resource can evaluate browser maturity
beforehand and switch to a \textbf{polyfill} or another fallback
solution instead of the latest browser optimized version. After the
initial paint, critical resources like top-level microservices or
related parts can be loaded and registered.

Regarding the evolution of the web, the `next billion' internet users
will most likely use Android, have decent specs mobile phones, use an
evergreen browser but won't have a reliable internet connection.{[}12{]}
While \textbf{Progressive Enhancement} was once related to build
websites both for browsers with and without JS support, the demands have
changed towards an \textbf{offline first} principle avoiding network
connectivity failures.{[}12{]} A \emph{browsernative microservice}
therefore not only tries to cache data as much as possible, it should
also bring in a lot of program logic as described in the previous
chapters.

\subsection{2.8. Evolutionary design}\label{evolutionary-design}

Microservices tend to become smaller over time. An evolutionary design
approach emphasizes decomposition and scrapping the service. ``The key
property of a component is the notion of independent replacement and
upgradeability.''{[}5{]} Therefore we can safely change and chop
services. Lazy components of the system which will not change often
should be separated from parts undergoing a lot of churn.{[}5{]}
Services which change for the same reason might be moved together or
even could be merged.

Pursuing flexibility in web development is a selling point as innovation
cycles in browser development are fast and technologies can change
quickly. Frontend related hardware, software and methodologies innovate
rapidly over time.

\emph{Browsernative microservices} should be perceived as complementary
technology in contrast to full-service frameworks like Angular. Being a
native technology, they pursue a strong interoperability approach to
existing systems. Andrew Rota, for example, came up with a pattern using
small, encapsulated and stateless web components as leaves in the tree
of React components instead of native HTML elements.{[}13{]} Even React
can eventually profit from the expressiveness of custom components using
a custom \texttt{\textless{}meaningful-button\textgreater{}} over a
native \texttt{\textless{}button\textgreater{}}. Most likely there will
always be some cutting edge framework promising advantages over native
code. Whatever new framework will be on the rise within the next years,
native components can eliminate future uncertainty, allowing rapid
reassembling towards new architectures.

\section{3. W3C specifications}\label{w3c-specifications}

Building a native microservice running on the `bare-metal' browser
engine requires a bunch of new specifications and assumptions. Most
importantly multiple \textbf{Web Components} specifications are needed.
Web Components is not a single standard. They are a kind of amalgam of
combined APIs from multiple w3c specs which can be used independently. A
web developer may choose one spec which can be combined with other
frameworks.

Depending on the context, some people argue for only two specs which
essentially make it possible to create a scoped component but do not
care about its distribution{[}14{]}. Some people prefer three specs
{[}15{]}, but the majority advocate the four specs variant, which is
listed on the quasi-official
\href{http://webcomponents.org}{webcomponents.org} website. For the
purpose of this article, the four specs variant will be discussed
briefly to provide a rough understanding.

\emph{Disclaimer:} This paper introduces many new browser build-ins with
the focus on accessibility. At the time of writing, all examples can be
tested in the console of the latest versions of \textbf{Google Chrome,
Opera and Apple Safari}.\footnote{\href{http://jonrimmer.github.io/are-we-componentized-yet/}{Are
  we componentized yet?}} On Mozilla Firefox they could be manually
enabled. However, browser implementation changes quickly and soon
technology adoption will not be an issue in all major browsers.
Meanwhile all new standards can be used through \textbf{polyfills} even
on legacy browsers.

\subsection{\texorpdfstring{3.1. Custom elements
\href{https://html.spec.whatwg.org/multipage/scripting.html\#custom-elements}{(whatwg)}}{3.1. Custom elements (whatwg)}}\label{custom-elements-whatwg}

\emph{Custom elements} are the fundamental building blocks for web
components introducing the \textbf{Single Responsibility Principle} to
the browser. In short, they provide a way to create custom HTML tags
subsuming behavior, design and functionality. An obligatory
\textbf{HelloWorld} will give a flavor about the spec:

\begin{Shaded}
\begin{Highlighting}[]
\OperatorTok{>} \VariableTok{HelloWorld}\NormalTok{.}\AttributeTok{js}
\KeywordTok{class} \NormalTok{HelloWorld }\KeywordTok{extends} \NormalTok{HTMLElement }\OperatorTok{\{}
 \AttributeTok{constructor}\NormalTok{() }\OperatorTok{\{}
  \KeywordTok{super}\NormalTok{()}\OperatorTok{;} \CommentTok{// mandatory in constructor}
  \KeywordTok{this}\NormalTok{.}\AttributeTok{onclick} \OperatorTok{=} \NormalTok{e }\OperatorTok{=>} \AttributeTok{alert}\NormalTok{(}\StringTok{"hello"}\NormalTok{)}\OperatorTok{;}
 \OperatorTok{\}}
\OperatorTok{\}}
\VariableTok{customElements}\NormalTok{.}\AttributeTok{define}\NormalTok{(}\StringTok{'hello-world'}\OperatorTok{,} \NormalTok{HelloWorld)}
\end{Highlighting}
\end{Shaded}

\begin{Shaded}
\begin{Highlighting}[]
\NormalTok{> index.html}
\KeywordTok{<hello-world>}\NormalTok{say hello}\KeywordTok{</hello-world>}
\end{Highlighting}
\end{Shaded}

This example should be almost self-explanatory in functionality.
\emph{Custom elements} come in the fashion of ES6 Classes in favor of
the JS prototype-based inheritance model which was part of an older
specification. Every valid element must \textbf{extend the base
\texttt{HTMLElement} interface} which~``ensures the newly created
element inherits the entire DOM API and any properties/methods that you
add to the class become part of the element's DOM interface.''{[}16{]}
Like any other ES6 class any \emph{Custom element} can be specialized
further using the typical inheritance model allowing higher levels of
abstraction.

The beauty of \emph{custom elements} comes with the \textbf{bounded
\texttt{this} keyword} which points to the element itself. Instead of
querying and assigning behavior after creation of the node, custom
elements ship their functionality on initialization of the element. The
so called \emph{fat-arrow} (\texttt{=\textgreater{}}) is just a new ES6
syntax feature for an anonymous function declaration.

After declaration the new HTML element needs to be registered in the
global build-in \texttt{customElements} object with a dedicated tag name
acting as key to the element. Mind the dash inside the tag name to
conform the spec. Finally, the new element can be mounted inside the
HTML document.

\subsubsection{3.1.1. Lifecycle methods}\label{lifecycle-methods}

In addition to the constructor which runs procedures on initialization,
the spec defines \textbf{lifecycle callbacks} for controlling elements'
behavior towards DOM interaction. Many popular frameworks like React or
Angular rely on similar approaches:

\begin{itemize}
\tightlist
\item
  connectedCallback()\\
  called upon the time of \textbf{connecting or upgrading the node}
  which means the moment the node is rendered inside the DOM. Typically
  this method is called straight after the constructor, if the node is
  inserted directly. For a faster initial render of the page it is
  highly preferable to put many proceedings in this method. Usually this
  method contains setup code such as fetching resources or rendering
  elements according to attributes.{[}16{]}
\item
  disconnectedCallback()\\
  called upon the time of \textbf{node removal}. Cleanup code like
  removing event listeners or disconnecting web sockets can be put here.
\item
  attributeChangedCallback(attrName, oldVal, newVal)\\
  This method provides an \textbf{onchange handler} for certain
  elements' attributes. This method is used to guide elements'
  transition from an old value to a new state. Due to performance issues
  this callback is only triggered for attributes registered in a
  dedicated array shipped with the element.
\item
  adoptedCallback()\\
  called when moving the node \textbf{between documents}. This method
  comes handy when using HTML Imports described later.
\end{itemize}

\subsubsection{3.1.2. Custom attributes}\label{custom-attributes}

As mentioned before any custom element must extend the
\texttt{HTMLElement} interface ensuring base properties and methods used
throughout all HTML elements like id, class, addEventListner etc.
Additionally, it is possible to define custom attributes using the
\emph{custom elements'} \textbf{getter / setter interface} to steer the
behavior of the element. Note that the get/set keywords as well as the
previously used constructor are optional!

\begin{Shaded}
\begin{Highlighting}[]
\OperatorTok{>} \VariableTok{HelloWorld}\NormalTok{.}\AttributeTok{js}
\KeywordTok{class} \NormalTok{HelloWorld }\KeywordTok{extends} \NormalTok{HTMLElement }\OperatorTok{\{}
  \NormalTok{set }\AttributeTok{sayhello}\NormalTok{(val) }\OperatorTok{\{}
    \KeywordTok{this}\NormalTok{.}\AttributeTok{_hello} \OperatorTok{=} \NormalTok{val}\OperatorTok{;}
  \OperatorTok{\}}
  \NormalTok{get }\AttributeTok{sayhello}\NormalTok{() }\OperatorTok{\{}
    \ControlFlowTok{return} \KeywordTok{this}\NormalTok{.}\AttributeTok{_hello}\OperatorTok{;}
  \OperatorTok{\}}
\OperatorTok{\}}
\VariableTok{customElements}\NormalTok{.}\AttributeTok{define}\NormalTok{(}\StringTok{'hello-world'}\OperatorTok{,} \NormalTok{HelloWorld)}\OperatorTok{;}
\CommentTok{// Instantiation via JS instead of HTML}
\KeywordTok{var} \NormalTok{el }\OperatorTok{=} \KeywordTok{new} \AttributeTok{HelloWorld}\NormalTok{()}\OperatorTok{;}
\VariableTok{el}\NormalTok{.}\AttributeTok{sayhello} \OperatorTok{=} \StringTok{"earth"}\OperatorTok{;} \CommentTok{// "Call" the setter}
\VariableTok{el}\NormalTok{.}\AttributeTok{sayhello}\OperatorTok{;}\CommentTok{// Yields "earth"}
\end{Highlighting}
\end{Shaded}

Native DOM properties always reflect their values between HTML and
JS.{[}17, Para. 2.6.1{]} Declaring
\texttt{\textless{}hello-world\ id="hello"\textgreater{}} equals to the
JS declaration \texttt{new\ HelloWorld().id\ =\ "hello"}.

This behavior will not work out-of-the-box with methods defined by
setters as they are strictly JS. Mounting
\texttt{\textless{}hello-world\ sayhello="mars"\textgreater{}} would not
result in calling the \texttt{sayhello} method in the previous setup.
Value reflection can be implemented inside \emph{custom elements} using
the native methods \texttt{getAttributes} and \texttt{setAttributes}.
Using them exhaustively throughout lifecycle methods the new components
can be configured to read and listen to HTML attributes accordingly.

Designing a \emph{custom element} this way creates HTML elements with
named attribute interfaces reaching deep into JS functionality. With
this mental model in mind a web developer can create highly dynamic web
components.

\subsubsection{3.1.3. Customized build-in
elements}\label{customized-build-in-elements}

One aspect not mentioned yet is the possibility of extending other
build-in elements by extending other interfaces instead of the
\texttt{HTMLElement} interface. While this functionality is perfectly
spec'd it is strongly rejected by some browser vendors.\footnote{https://github.com/w3c/webcomponents/issues/509}
Most likely the spec will change in future in one or other direction
concerning this issue and therefore customized build-in elements are
left out in this paper intentionally.

\subsection{\texorpdfstring{3.2. Shadow DOM
\href{http://w3c.github.io/webcomponents/spec/shadow/}{(w3c)}}{3.2. Shadow DOM (w3c)}}\label{shadow-dom-w3c}

A \emph{shadow DOM} is just an isolated DOM tree living inside another
DOM tree. The spec refers the hosting tree as \textbf{light DOM tree}
and the attached DOM as \textbf{shadow DOM tree}. Conceptually
\emph{shadow DOM} issues a single important function for building
scalable software which is namely \textbf{encapsulation}. While custom
elements provide a good way to encapsulate JS behavior \emph{shadow DOM}
tends strongly to the direction of style and event encapsulation.

With an ever increasing complexity of single-page applications the
global nature of the DOM creates a daunting situation for code
organization and leads over times to highly fragmented bits of CSS and
obscured CSS selectors. Of course this situation dramatically lowers
code clarity and reusability. The only solution which will not break
with the existing global paradigm of the DOM is to allow separate pieces
of encapsulated code sit on top of the global DOM - introducing the
shadowed DOM approach.

Enhancing the previous example the new encapsulated \texttt{HelloWorld}
would look like the following code snippet:

\begin{Shaded}
\begin{Highlighting}[]
\OperatorTok{>} \VariableTok{HelloWorld}\NormalTok{.}\AttributeTok{js}
\KeywordTok{class} \NormalTok{HelloWorld }\KeywordTok{extends} \NormalTok{HTMLElement }\OperatorTok{\{}
 \AttributeTok{constructor}\NormalTok{() }\OperatorTok{\{}
   \KeywordTok{this}\NormalTok{.}\AttributeTok{attachShadow}\NormalTok{(}\OperatorTok{\{}\DataTypeTok{mode}\OperatorTok{:} \StringTok{'open'}\OperatorTok{\}}\NormalTok{)}\OperatorTok{;}
   \VariableTok{shadowRoot}\NormalTok{.}\AttributeTok{innerHTML} \OperatorTok{=} \StringTok{'<p>hello</p>'}\OperatorTok{;}
 \OperatorTok{\}}
\OperatorTok{\}}
\end{Highlighting}
\end{Shaded}

The new global method \texttt{attachShadow} adds a new document root to
the \texttt{HelloWorld} which has the same properties as a normal, light
document object. Note that the \texttt{shadowRoot} object is
\textbf{marked as open} which ensures that some events can bubble out
and outer JS can capture the new root.

Filling the \emph{shadow DOM} with an \texttt{innerHTML} string is
rather impractical. To fill a \emph{shadow DOM} with life, it usually
pulls light DOM child nodes nested under the hosting node using a
technique called \texttt{slots}.

\subsubsection{3.2.1. Slots}\label{slots}

Contradicting the simplified \texttt{HelloWorld} example, a \emph{shadow
DOM} should not contain dynamic content. Changing or interacting with
the paragraph node from the previous example would require nested JS
calls querying the hosting node, entering the \emph{shadow DOM} and
applying a function. Imported JS behavior from third-party libraries in
the \emph{light DOM} cannot easily reach inside the \emph{shadow DOM},
too.

This is the reason why the shadowed document root should rather be
perceived as \textbf{static document} filled and managed solely by the
render engine. \texttt{Slots} are target areas for \emph{light DOM}
nodes used to mark the endpoints in question.

\begin{Shaded}
\begin{Highlighting}[]
\NormalTok{> index.html}
\KeywordTok{<HelloWorld-with-ShadowDOM>}
  \CommentTok{<!--}
\CommentTok{    All child nodes will be moved inside the}
\CommentTok{    shadowRoot if shadowRoot.innerHTML = '<slot></slot>'}
\CommentTok{    -->}
  \KeywordTok{<p>}\NormalTok{hello I will be scoped}\KeywordTok{</p>}
\KeywordTok{</HelloWorld-with-ShadowDOM>}
\end{Highlighting}
\end{Shaded}

From a technical perspective, the \emph{light DOM} nodes are not moved
inside the \emph{shadow DOM}. They are just rendered in place. This is a
subtle but important difference towards handling a node. All JS behavior
and CSS styles applied in the \emph{light DOM} will be valid in the
\emph{shadow DOM}. The render engine literally takes the nodes and drops
them inside the \texttt{slot} tag. This procedure is commonly referred
as \textbf{flattening} of the DOM trees.

It is possible to add semantics to the \emph{shadow DOM} in naming the
slots which free the \emph{light DOM} from the responsiblity to deliver
nodes in a correct top-to-bottom order. Combining \emph{shadow DOM} with
HTML templates equips the web developer with a flexible \textbf{HTML
template engine}.

\begin{longtable}[]{@{}ll@{}}
\toprule
FROM: light DOM & TO: shadow DOM\tabularnewline
\midrule
\endhead
\texttt{\textless{}p\ slot="hello"\textgreater{}Named\textless{}/p\textgreater{}}
&
\texttt{\textless{}slot\ name="hello"\textgreater{}hello\ only\textless{}/slot\textgreater{}}\tabularnewline
\texttt{\textless{}p\textgreater{}Unnamed\textless{}/p\textgreater{}} &
\texttt{\textless{}slot\textgreater{}Unnamed\ nodes\textless{}/slot\textgreater{}}\tabularnewline
\bottomrule
\end{longtable}

Writing a little documentation inside the
\texttt{\textless{}slot\textgreater{}} tag is considered as a good
practice as it provides the developer with visual clues about what nodes
must be delivered. This functionality makes a \emph{shadow DOM} pretty
much self-explanatory. Inside a default slot tag the render engine
expands all \emph{light DOM} children without a named \texttt{slot}
attribution.

\subsubsection{3.2.2. Styling}\label{styling}

As mentioned in the last section, there is a distinct difference between
nodes. Nodes declared and rendered exclusively in the \emph{shadow DOM}
are not affected by any styling from outside. Nodes which are
distributed will be styled in the \emph{light DOM} and can be
additionally painted in the \emph{shadow DOM}.

Note that styles from the outside have an higher specificity than styles
assigned after distribution. Therefore it is generally a good advice to
minimize the global stylings to some base stylings for uniformity of the
web site while leaving the specific stylings to the component. It is
possible to \textbf{reset all styles} inside the \emph{light DOM} before
distributing nodes using the \textbf{\texttt{all:\ initial}} reset. To
ensure a consistent look between different shadow roots this technique
should be used carefully.

Regarding the importance of style encapsulation, a couple of \textbf{new
CSS rules} emerged that are exclusively targeting the \emph{shadow DOM}.
The points below outline styling possibilities for the use INSIDE the
\emph{shadow DOM}:

\begin{itemize}
\tightlist
\item
  ::slotted(selector)\\
  applies to distributed nodes and repaints them after distribution.
  Slotted will not override outsides styles but can complement
  previously unset style rules.
\item
  :host\\
  The host property will add styles or change inherited ones inside
  shadow DOM. Aforementioned style resets can be placed here.
\item
  :host(condition)\\
  Like the previous rule this selector will style the shadow DOM but
  this time based on attributes/conditions assigned to the hosting node.
\item
  :host-context(condition)\\
  Like the previous rule this selector will style the shadow DOM but
  will look after context set at the host node or even at the host
  ancestor.
\end{itemize}

Using the \textbf{functional selectors} \texttt{host()} or
\texttt{host-context()} allows the creation of context-aware custom
elements. A possible usecase would be `theming' a component.

\subsubsection{3.2.3. Behavior}\label{behavior}

As mentioned before, any logic applied to \emph{light DOM} nodes stays
within the node even after redistribution. For the sake of separation of
concerns the business logic should be part of the custom element (the
\emph{light DOM}) and not the part of the \emph{shadow DOM}. On the
other hand there are numerous scenarios where JS is used for styling or
animation of an element. In this case it might be more straightforward
to \textbf{apply JS inside the \emph{shadow DOM}} to avoid mixing logic
handlers with styling.

It seems that a \emph{light DOM} node is in the \emph{shadow DOM}
context after distribution. Visually this may be true but logically the
node stays in the normal document. To reach a distributed node from the
\emph{shadow DOM} to apply some JS behavior for styling, it needs the
extra way over the slot node. Calling \texttt{assinedNodes()} on the
hosting slot element returns a linkage to the distributed node which can
be accessed and manipulated like in the \emph{light DOM} context.

Wrapping up this section, \emph{shadow DOM} provides a non-hacky way to
create uniform looking custom elements and even enhance styling
possibilities without adding overhead. For small components with just a
little styling, \emph{shadow DOM} might be over engineered. Eventually
it all depends on the question of `how hard is it to implement it
without shadow DOM' - which cannot be answered in general. For a more
in-depth guide, Google Engineer Eric Bidelman wrote a primer on
\emph{shadow DOM}{[}18{]}.

There is still a missing link between \emph{light DOM} and \emph{shadow
DOM}. The attentive reader may have already noticed the weak point in
the HelloWorld example: how to `vitalize' the \emph{shadow DOM} with
slot tags or other element structuring instead of markup strings. While
strings work perfectly fine in this simple case a string of markup is
rather cumbersome and error-prone and does not scale well. When putting
quotes inside other quotes things break quickly. Strings make
development harder because code editor features like indentation or
syntax highlighting will not be supported. The HTML templates are set to
fill this gap.

\subsection{\texorpdfstring{3.3. HTML templates
\href{https://html.spec.whatwg.org/multipage/scripting.html\#the-template-element}{(whatwg)}}{3.3. HTML templates (whatwg)}}\label{html-templates-whatwg}

Among all other new specifications \emph{HTML templates} are the most
mature and adopted standard in the browser environment. All major
browsers support it since years.

A core concept in templates is browser performance. Elements inside a
\texttt{template} tag will be parsed on runtime - but not constructed
and rendered into the content tree. They are remaining plain HTML
markups sitting in the document until the time of activation.

Activation usually takes four steps:

\begin{enumerate}
\def\labelenumi{\arabic{enumi}.}
\tightlist
\item
  \textbf{Querying the template node in question}\\
  const node = document.querySelector(`template');
\item
  \textbf{Parsing the content and preparing the templates' content}\\
  const content = node.content;\\
  Returns a DocumentFragment object.
\item
  \textbf{Optional: Cloning the fragment for multiple use}\\
  const clone = content.cloneNode(``deep'');
\item
  \textbf{Appending the clone/original to destination}\\
  document.body.appendChild(clone);
\end{enumerate}

As easy and minimal as \emph{HTML templates} are, they skip a crucial
feature other template implementations used to have. As template tags
are basically just containers for HTML markup there is no idiomatic way
to define \textbf{placeholders} for dynamic content. Templates could be
mocked up this way using JS for altering the content but a much cleaner
way leverages the previously described slot technique from shadow DOM.

\begin{Shaded}
\begin{Highlighting}[]
\NormalTok{> hello-world-component.html}
\KeywordTok{<hello-world>}
  \KeywordTok{<template}\OtherTok{ id=}\StringTok{"hello"}\KeywordTok{>}
  \CommentTok{<!-- styling -->}
    \KeywordTok{<style>}
      \FloatTok{#hellowrap} \KeywordTok{\{}
        \KeywordTok{font-weight:} \DataTypeTok{bold}\KeywordTok{;}
        \KeywordTok{color:} \NormalTok{orange}\KeywordTok{;}
      \KeywordTok{\}}
    \KeywordTok{</style>}
    
    \CommentTok{<!-- structure -->}
    \KeywordTok{<div}\OtherTok{ id=}\StringTok{"hellowrap"}\KeywordTok{>}
      \KeywordTok{<slot}\OtherTok{ name=}\StringTok{"placeholder"}\KeywordTok{>}
        \NormalTok{Named placeholder}
      \KeywordTok{</slot>}
    \KeywordTok{</div>}
  \KeywordTok{</template>}
  
  \CommentTok{<!-- light DOM / typical inserted at index.html -->}
  \KeywordTok{<p}\OtherTok{ slot=}\StringTok{"placeholder"}\KeywordTok{>}
    \NormalTok{Hello World Web Component}
  \KeywordTok{</p>}
\KeywordTok{</hello-world>}

\KeywordTok{<script>}
  \KeywordTok{class} \NormalTok{HelloWorld }\KeywordTok{extends} \NormalTok{HTMLElement }\OperatorTok{\{}
    \AttributeTok{constructor}\NormalTok{() }\OperatorTok{\{}
      \KeywordTok{super}\NormalTok{()}\OperatorTok{;}
      \KeywordTok{this}\NormalTok{.}\AttributeTok{attachShadow}\NormalTok{(}\OperatorTok{\{}\DataTypeTok{mode}\OperatorTok{:} \StringTok{'open'}\OperatorTok{\}}\NormalTok{)}\OperatorTok{;}
      \KeywordTok{const} \NormalTok{temp }\OperatorTok{=} \KeywordTok{this}\NormalTok{.}\AttributeTok{querySelector}\NormalTok{(}\StringTok{'template#hello'}\NormalTok{)}\OperatorTok{;}
      \KeywordTok{this}\NormalTok{.}\VariableTok{shadowRoot}\NormalTok{.}\AttributeTok{appendChild}\NormalTok{(}\VariableTok{temp}\NormalTok{.}\AttributeTok{content}\NormalTok{)}\OperatorTok{;}
    \OperatorTok{\}}
  \OperatorTok{\}}
  \VariableTok{customElements}\NormalTok{.}\AttributeTok{define}\NormalTok{(}\StringTok{'hello-world'}\OperatorTok{,} \NormalTok{HelloWorld)}\OperatorTok{;}
\OperatorTok{<}\SpecialStringTok{/script>}
\end{Highlighting}
\end{Shaded}

The updated \texttt{HelloWorld} component looks already pretty mature.
It combines all the previously mentioned standards into one HTML file.
Custom elements serve the logic, shadow DOM scopes the styles and
\emph{HTML templates} efficiently glue light DOM and shadow DOM
together. This separation of concerns comes with a surplus in
flexibility. In a real world scenario \texttt{HelloWorld} could
reference multiple \emph{HTML templates} and switch them around without
any fuss. Even further a developer might split up templates into named
\textbf{STYLE templates and CONTENT templates} to increase reusability
even further.

The last standard in the row of four is not concerned with the
implementation of a web component. HTML imports serves the need for an
efficient distribution mechanism of components and other HTML resources.

\subsection{\texorpdfstring{3.4. HTML imports
\href{https://www.w3.org/TR/html-imports/}{(w3c)}}{3.4. HTML imports (w3c)}}\label{html-imports-w3c}

Importing the \texttt{HelloWorld} component is a one-liner:

\begin{Shaded}
\begin{Highlighting}[]
\NormalTok{> index.html}
\KeywordTok{<link}\OtherTok{ rel=}\StringTok{"import"}\OtherTok{ href=}\StringTok{"Hello.html"}\OtherTok{ async}\KeywordTok{>}

\KeywordTok{<hello-world>}
  \KeywordTok{<p}\OtherTok{ slot=}\StringTok{"placeholder"}\KeywordTok{>}
    \NormalTok{Hello World Web Component}
  \KeywordTok{</p>}
\KeywordTok{</hello-world>}
\end{Highlighting}
\end{Shaded}

The \texttt{async} flag is optional but recommended like in any other
fetching event. Once the imported HTML document comes into scope,
activation follows a very similar process compared to the aforementioned
HTML templates:

\begin{enumerate}
\def\labelenumi{\arabic{enumi}.}
\item
  \textbf{Querying the link node}
\item
  \textbf{Parsing the content and preparing the render}\\
  const content = linknode.import; -\textgreater{} Contrary to HTML
  templates a complete document object is constructed.
\item
  \textbf{Optional: Cloning some nodes for multiple use}
\item
  \textbf{Appending the clone/original to destination}
\end{enumerate}

Again this is the imperative way to handle a generic \emph{HTML import}.
In the declarative world of web components a component is
\textbf{parsed, auto-activated and anchored} solely by its tag name
\texttt{\textless{}hello-world\textgreater{}} on purpose. The very own
lifecycle method \textbf{adoptedCallback()} in custom elements shows the
strong interconnection between those standards.

\emph{HTML imports} can import everything wrappable in HTML markup.
Stylesheets, scripts, documents, media files and even further import
statements can form a semantic \emph{HTML import} statement.

The far reaching possibilities of a single standard has its drawbacks.
According to Mozilla \emph{HTML imports} are not fully compatible with
the dependency model of the new ES6 modules.\footnote{https://hacks.mozilla.org/2014/12/mozilla-and-web-components/}
In the current situation Mozilla, Apple and Microsoft will not implement
\emph{HTML imports} soon if any. Only Google's blink web engine supports
\emph{HTML imports} as they are the driving force behind the web
components specs in general. Despite the discrepancies among browser
vendors, \emph{HTML imports} are part of this paper as no other native
standard can ship bundled HTML, CSS and JS which is a core concept
behind web components.

\subsection{\texorpdfstring{3.5. CustomEvent
\href{https://dom.spec.whatwg.org/\#interface-customevent}{(whatwg)}}{3.5. CustomEvent (whatwg)}}\label{customevent-whatwg}

Events are first-class citizens in the browser providing a neat
communication channel for dynamic interactions. \emph{CustomEvent} is
part of the DOM since years but with the rise of web components it will
most likely become an indispensable building block of web components.

\begin{Shaded}
\begin{Highlighting}[]
\NormalTok{> hello-world-component.html}

\KeywordTok{<hello-world>}
 \KeywordTok{<button>}\NormalTok{Launch CustomEvent}\KeywordTok{</button>}
\KeywordTok{</hello-world>}

\KeywordTok{<script>}
 \VariableTok{customElements}\NormalTok{.}\AttributeTok{define}\NormalTok{(}\StringTok{'hello-world'}\OperatorTok{,} 
 \KeywordTok{class} \KeywordTok{extends} \NormalTok{HTMLElement }\OperatorTok{\{}
  \AttributeTok{constructor}\NormalTok{() }\OperatorTok{\{}
    \KeywordTok{super}\NormalTok{()}\OperatorTok{;}
   
   \CommentTok{// Craft a helloevent}
    \KeywordTok{const} \NormalTok{helloevent }\OperatorTok{=} \KeywordTok{new} \AttributeTok{CustomEvent}\NormalTok{(}\StringTok{'helloevent'}\OperatorTok{,} \OperatorTok{\{}
    \DataTypeTok{bubbles}\OperatorTok{:} \KeywordTok{true}\OperatorTok{,}
    \DataTypeTok{detail}\OperatorTok{:} \StringTok{'Contains scalar or object'}
   \OperatorTok{\}}\NormalTok{)}\OperatorTok{;}
   
   \CommentTok{// Launch helloevent if child button clicks}
   \KeywordTok{this}\NormalTok{.}\AttributeTok{addEventListener}\NormalTok{(}\StringTok{'click'}\OperatorTok{,} \NormalTok{click }\OperatorTok{=>} \OperatorTok{\{}
    \KeywordTok{this}\NormalTok{.}\AttributeTok{dispatchEvent}\NormalTok{(helloevent)}
    \VariableTok{click}\NormalTok{.}\AttributeTok{stopPropagation}\NormalTok{()}\OperatorTok{;}
   \OperatorTok{\}}\NormalTok{)}
  \OperatorTok{\}}
 \OperatorTok{\}}\NormalTok{)}\OperatorTok{;}
\OperatorTok{<}\SpecialStringTok{/script>}
\end{Highlighting}
\end{Shaded}

Naming events after the emitting tag makes the \textbf{API almost
self-explanatory}. The \texttt{detail} property can be loaded with
scalars as well as objects. Subsequent parent nodes may catch the custom
event with a clear understanding about the source node.

Chaining and aggregating events from child nodes can be frequently used
within web components. One use case could be creating a custom button
like in the previous example. As mentioned earlier in the Custom
Elements section at 3.1.3. , the pattern of \textbf{extending native
elements} should be dismissed as certain browser vendors imposed
distaste towards this functionality. A common workaround to an eventual
creation of an extended native element could be made with a thin wrapper
around the native element and chaining a \emph{CustomEvent} after the
click event.

Another common web components' use case can be a \textbf{middleware}
subscribing to certain child events and acting upon them. For example
embedding another third-party web component or widget which can be
wrapped in a middleware component to align it to system conventions.
This involves catching events, buffering, destructuring and creation of
own events. By design, events only bubble upstream towards parent nodes.
For handing down information towards the child nodes we need to query
the child node directly.

\subsection{\texorpdfstring{3.6. Web worker
\href{https://html.spec.whatwg.org/multipage/workers.html}{(whatwg)}}{3.6. Web worker (whatwg)}}\label{web-worker-whatwg}

Like CustomEvent, \emph{web workers} have been around for a long time
and therefore enjoy full browser support. They emerged at around 2009
when discussions about browser performance were still in the early days.
\emph{Web workers} however addressed a fundamental performance
bottleneck of the JS language.

JS runs in a single-threaded language environment. Every script in the
browser environment, from handling UI events to query and process large
amounts of data or manipulating the DOM runs on a single thread{[}19{]}.
Putting a lot of work into the single main thread can slow down the web
service significantly. From time to time scripts can block or fail for
whatever reason which leads to a frozen or crashed UI. A worker can
overcome the bottleneck of the single-threaded nature with spawning new
\textbf{background threads} which allows the UI to stay responsive even
when computation-heavy tasks need to perform. Furthermore, a worker adds
a performance advantage embracing the multi core CPU architecture most
devices are running on today. To grasp the full potential of workers, a
reader might dive deeper into the Angular 2 architecture, where most of
the application layer is abstracted from the main rendering thread into
worker threads.\footnote{\href{https://docs.google.com/document/d/1M9FmT05Q6qpsjgvH1XvCm840yn2eWEg0PMskSQz7k4E}{Angular
  2 Rendering Architecture}}

\section{4. Building a browsernative
microservice}\label{building-a-browsernative-microservice}

After getting confidence in microservice principles and technical
background the paper should briefly join them to form a
\emph{browsernative microservice}. Needless to say that the following
example is overall simplified towards illustrating the connection
between browsernative technologies and microservice patterns.
Furthermore it is an opinionated approach and should not be perceived as
a `single source of truth'.

Google's Polymer project is a good place for learning about web
components in depth and make use of their toolbox. One of their proof of
concept is the so-called \textbf{Polymer Shop}\footnote{\href{https://shop.polymer-project.org/}{Polymer
  Shop}} which is a fully-fledged online shop nested within a single
root element \texttt{\textless{}shop-app\textgreater{}}. This app is
made of several main views and many more invisible wrapper elements for
routing, service worker caching, theming, etc. The whole shop runs as a
single application fetching and updating remote resources and switching
views. Let us assume we work in a sales engineering team of the Polymer
Shop and we need to rebuild the checkout microservice.

The current checkout can be found at
https://shop.polymer-project.org/checkout. At the time of writing, the
checkout is a single, 671 lines of code long Polymer component including
all required fields for signing in, shipping, billing and summarizing
the order. In the spirit of microservices we will split up the
microservice into independent components. The shopping cart data is
pulled out of a local storage JSON entity, set up previously by another
custom element.

By breaking down the service the \textbf{team defined the business
boundaries} within the checkout process and came up with following
granular service blocks:

\begin{enumerate}
\def\labelenumi{\arabic{enumi}.}
\tightlist
\item
  Sign in
\item
  Shipping details
\item
  Payment details
\item
  Review and place order
\end{enumerate}

Translated into a raw \textbf{Custom Element} HTML structure, the
top-level microservice might look like the following snippet:

\begin{Shaded}
\begin{Highlighting}[]
\NormalTok{>shop-checkout.html}
\KeywordTok{<shop-checkout>}
  \KeywordTok{<sign-in></sign-in>}
  \KeywordTok{<shipping-details></shipping-details>}
  \KeywordTok{<payment-details></payment-details>}
  \KeywordTok{<place-order></place-order>}
\KeywordTok{</shop-checkout>}
\end{Highlighting}
\end{Shaded}

Yet already we see the simplicity arouse from web components as they
persue a clean markup. Each of the child components may act
independently over other child nodes utilizing the \textbf{loose
coupling principle}. Each child ships all the HTML, CSS and JS code
needed to fulfil its work following the \textbf{high cohesion
principle}. Each component may contain different views to accommodate
different \textbf{bounded contexts resulting from different devices}.
And last but not least, all of them communicate over an
\textbf{unobtrusive message bus} via the service root component
\texttt{\textless{}shop-checkout\textgreater{}}.

Before diving deeper into implementation, it is worth to clarify an
\textbf{architectural pattern} behind components. To any reader of the
paper who already came across React, the concept of components may look
familiar. Dan Abramov, the creator of Redux, once defined a simple
dichotomous pattern for creating UI components.

Firstly, Abramov defined a pattern around \textbf{presentational
components} only related with the concern about \emph{how things look}.
This component literally does not know anything about the service in
question which makes the component highly flexible and reusable. It is
controlled solely from the outside, receiving data and dispatching
unbiased events on user interaction.{[}20{]} Most probably every
presentational component embodies more HTML/CSS markup and less JS code.
It should encapsulate its styles from bleeding out and protect its
styles from being overwritten. Furthermore, it may contain several
templates to change its look on different demands.

Secondly, Abramov described components he refers to as
\textbf{containers}. A container component is concerned with \emph{how
things work}.{[}20{]} Containers act as invisible wrappers around
presentational components acting in the sense of UNIX filters. Their job
is to fetch data from child nodes, aggregating events, interacting with
the model and push state back to the presentational components.
Consequently they might contain more JS and less if any HTML markup. We
probably do not need to utilize shadow DOM as no styles are involved.

Last but not least, the pattern can be expanded for illustrational
purposes to \textbf{native components} such as every build-in
HTMLElement like an ordinary HTMLButtonElement. Native components are
mostly deeply nested elements providing the actual functionality in the
browser UI. They are solely controllable and styleable from the outside
and are therefore wrapped in presentational components and/or
containers.

Lets start the service description top-down beginning with the
\textbf{service root container} managing the overall service.

\subsection{4.1. Service root}\label{service-root}

The root container \texttt{\textless{}shop-checkout\textgreater{}} is
basically just an encapsulation layer in terms of service
functionalities. Encapsulation of CSS will not be an issue at this point
as no styling is involved. A simplified \emph{root container} for the
checkout might look like the following code snippet:

\begin{Shaded}
\begin{Highlighting}[]
\NormalTok{> shop-checkout.html}
\CommentTok{<!-- IMPORTS -->}
\KeywordTok{<link}\OtherTok{ rel=}\StringTok{"import"}\OtherTok{ href=}\StringTok{"sign-in.html"}\OtherTok{ async}\KeywordTok{>}
\NormalTok{...}
\CommentTok{<!-- VIEW -->}
\KeywordTok{<shop-checkout>}
  \KeywordTok{<sign-in></sign-in>}
  \NormalTok{...}
\KeywordTok{</shop-checkout>}

\CommentTok{<!-- CONTROLLER -->}
\KeywordTok{<script>}
\KeywordTok{class} \NormalTok{ShopCheckout }\KeywordTok{extends} \NormalTok{HTMLElement }\OperatorTok{\{}
  \AttributeTok{constructor}\NormalTok{() }\OperatorTok{\{}
    \KeywordTok{super}\NormalTok{()}\OperatorTok{;}
    \KeywordTok{this}\NormalTok{.}\AttributeTok{MODEL} \OperatorTok{=} \KeywordTok{new} \AttributeTok{Worker}\NormalTok{(}\StringTok{'checkout-model.js'}\NormalTok{)}\OperatorTok{;}
  \OperatorTok{\}}
  
  \AttributeTok{connectedCallback}\NormalTok{() }\OperatorTok{\{}
    \CommentTok{// listen to msg from MODEL}
    \KeywordTok{this}\NormalTok{.}\VariableTok{MODEL}\NormalTok{.}\AttributeTok{addEventListener}\NormalTok{(}\StringTok{'message'}\OperatorTok{,} \NormalTok{[Msg Handler])}
    \CommentTok{// listen to child nodes}
    \KeywordTok{this}\NormalTok{.}\AttributeTok{addEventListener}\NormalTok{(}\StringTok{'checkout'}\OperatorTok{,} \NormalTok{e }\OperatorTok{=>} \OperatorTok{\{}
      \CommentTok{// delegate event to handler}
      \KeywordTok{this}\NormalTok{.}\AttributeTok{_sendToMODEL} \OperatorTok{=} \NormalTok{e}\OperatorTok{;} 
    \OperatorTok{\}}\NormalTok{)}\OperatorTok{;}
  \OperatorTok{\}}
  
  \NormalTok{set }\AttributeTok{_sendToMODEL}\NormalTok{(e) }\OperatorTok{\{}
    \KeywordTok{const} \NormalTok{id }\OperatorTok{=} \VariableTok{e}\NormalTok{.}\VariableTok{target}\NormalTok{.}\AttributeTok{id}\OperatorTok{,} 
          \NormalTok{name }\OperatorTok{=} \VariableTok{e}\NormalTok{.}\VariableTok{target}\NormalTok{.}\AttributeTok{localName}\OperatorTok{,}
          \NormalTok{load }\OperatorTok{=} \VariableTok{e}\NormalTok{.}\AttributeTok{detail}\OperatorTok{,}
          \NormalTok{letter }\OperatorTok{=} \VariableTok{Object}\NormalTok{.}\AttributeTok{assign}\NormalTok{(}\OperatorTok{\{\},} \OperatorTok{\{} \NormalTok{id}\OperatorTok{,} \NormalTok{name}\OperatorTok{,} \NormalTok{load }\OperatorTok{\}}\NormalTok{)}\OperatorTok{;}
    \CommentTok{// send to MODEL}
    \KeywordTok{this}\NormalTok{.}\VariableTok{MODEL}\NormalTok{.}\AttributeTok{postMessage}\NormalTok{(letter)}\OperatorTok{;} 
  \OperatorTok{\}}
  \NormalTok{...}
\OperatorTok{\}}    
\VariableTok{customElements}\NormalTok{.}\AttributeTok{define}\NormalTok{(}\StringTok{'shop-checkout'}\OperatorTok{,} \NormalTok{ShopCheckout)}\OperatorTok{;}
\OperatorTok{<}\SpecialStringTok{/script>}
\end{Highlighting}
\end{Shaded}

The purpose of this simplified code snippet is to outline the
\textbf{MVC threefold} in the \emph{service root}. The MODEL is pushed
into the worker thread, the CONTROLLER in the custom element and the
VIEW in the HTML structuring. On runtime the \emph{root container} acts
like a \textbf{message dispatcher} implementing the microservice
principle of \textbf{smart endpoints and dumb pipes}.

In order to interact with the MODEL every child node must implement the
dedicated \texttt{checkout} CustomEvent (which will be explained further
on in the next section). Illustrating the \textbf{unidirectional
communication} from the \emph{service root} perspective may look like
the following plot:

\begin{verbatim}
VIEW thread                 |               MODEL thread
                            |
+----------+                |               +----------+
| Checkout | ----------Event Msg--------->  |  Msg     |
|  Event   |                |               | Handler  |
+----------+                |               +----------+
                            |                   |
                            |                   |
                            |               +----------+
                            |               | Generate |
                            |               |  Effect  |
                            |               +----------+
                            |                   |
                            |                   |
+----------+                |               +----------+
|  Msg     |                |               | Postmsg  |
| Handler  | <--------Action Msg----------  |  Action  |
+----------+                |               +----------+    
                            |
\end{verbatim}

Effects are yielded by the asynchronous operation of messages and create
actions returned to sender. Effects can be created by external
operations, like subscription to an WebSocket, too. Effects may be
created with additional resources from the server or syncing with local
storage like those in the polymer-shop.

While the msg handlers are basic `dumb' switch statements the
\textbf{smartness} solely arises from intelligent controllers processing
the message. This communication model offers lots of possibilities for
\textbf{evolutionary design} as child nodes can be loosely dropped.
Every child node can be expanded into another full microservice or split
up into separate nodes without notice of the \emph{service root}.
Communication may fail without harming effects, providing a default
console log.

\subsection{4.2. Container components}\label{container-components}

The second layer of the checkout microservice like
\texttt{\textless{}sign-in\textgreater{}} or
\texttt{\textless{}shipping-details\textgreater{}} still contains mostly
logic and little or less styling. The job of a \emph{container
component} is to control its underlying presentational components and to
define a \textbf{set of ACTIONS towards the model}. Every
\emph{container} mounted directly under the service root may have a
dedicated area inside the worker thread reserved for its duties. For
example, a typical \texttt{\textless{}sign-in\textgreater{}} contains
merely two fields, username and password and a submit button. The
\emph{container component} waits for a submit action, aggregating the
credentials, and might add some semantics to them. Field values and the
action message will be dispatched towards the model for further
processing like initiating an authorization process.

Every \emph{container component} is eligible to aggregate subordinate
events from its children, buffer them and interact with the model via a
\textbf{unified message system}. A base class, from which
\texttt{\textless{}sign-in\textgreater{}} or
\texttt{\textless{}shipping-details\textgreater{}} can be extended,
might look like this:

\begin{Shaded}
\begin{Highlighting}[]
\KeywordTok{class} \NormalTok{SimpleContainer }\KeywordTok{extends} \NormalTok{HTMLElement }\OperatorTok{\{}
  \CommentTok{// constructor, static methods ...}
  \NormalTok{set }\AttributeTok{_dispatch}\NormalTok{(msg) }\OperatorTok{\{}
    \KeywordTok{this}\NormalTok{.}\AttributeTok{dispatchEvent}\NormalTok{(}
      \KeywordTok{new} \AttributeTok{CustomEvent}\NormalTok{(}\StringTok{'checkout'}\OperatorTok{,} \OperatorTok{\{}
      \DataTypeTok{bubbles}\OperatorTok{:} \KeywordTok{true}\OperatorTok{,}
      \DataTypeTok{detail}\OperatorTok{:} \NormalTok{msg}
      \OperatorTok{\}}\NormalTok{)}
    \NormalTok{)}\OperatorTok{;}
  \OperatorTok{\}}
  \NormalTok{set }\AttributeTok{_receive}\NormalTok{(msg) }\OperatorTok{\{}
    \CommentTok{// switch to methods}
    \VariableTok{console}\NormalTok{.}\AttributeTok{log}\NormalTok{(}\VerbatimStringTok{`}\SpecialCharTok{$\{}\KeywordTok{this}\NormalTok{.}\AttributeTok{localName}\SpecialCharTok{\}}\VerbatimStringTok{ received }\SpecialCharTok{$\{}\NormalTok{msg}\SpecialCharTok{\}}\VerbatimStringTok{`}\NormalTok{)}\OperatorTok{;}
  \OperatorTok{\}}
\OperatorTok{\}}
\end{Highlighting}
\end{Shaded}

Extending the \texttt{SimpleContainer} will equip every \emph{container
node} with the unified message interface. Calling
\texttt{this.\_dispatch(msg)} within the \emph{container} will trigger
an event. The service root will implement a simple event listener for
\texttt{checkout} events. After receiving an answer from the MODEL a
service root can query the \emph{container child node} in question and
push forward the answer over the \texttt{\_receive} property.

There might be even \textbf{more middleware containers} pulled in
between presentational leafs and the service root to fulfil some extra
work like filters on the event or without caring about checkout events
altogether. Changing or enhancing any functionality requires only the
controller in the \emph{container} in question and the endpoint section
at the model. \textbf{Infrastructure automation} might be achieved by
dynamically evaluating the mounted \emph{containers} beneath the service
root and modeling the `backend' model accordingly. Due to its
standardized message system the \emph{containers} are testable within
standardized tests exposing them to different synthetic events.

\subsection{4.3. Presentational
components}\label{presentational-components}

In comparison to the former container the \emph{presentational
component} is build mostly around views and styling with just a little
logic to switch templates around or to alter styling accordingly. As
much as the container can be perceived as a wrapper for logic, the
\emph{presentational component} can be perceived as \textbf{wrapper for
styling} needs. To extend the last
\texttt{\textless{}sign-in\textgreater{}} example a \emph{presentational
component} structure could look like the following top-level structure:

\begin{Shaded}
\begin{Highlighting}[]
\KeywordTok{<sign-in>}
  \KeywordTok{<sign-in-styling}\OtherTok{ theme=}\StringTok{"dark"} \ErrorTok{...}\KeywordTok{>}
    \CommentTok{<!-- Native components // form button ... -->}
  \KeywordTok{</sign-in-styling>}
\KeywordTok{</sign-in>}
\end{Highlighting}
\end{Shaded}

The \texttt{\textless{}sign-in-styling\textgreater{}} contains all
required input fields and Oauth connectors to Google or Facebook but
\textbf{will not care about events they create}. A \emph{presentation
component} is usually steered via HTML attributes. Contrary to the
former containers, the \emph{presentational component} highly utilizes
\textbf{shadow DOM and HTML templates}.

\begin{Shaded}
\begin{Highlighting}[]
\NormalTok{> sign-in-styling.html}
\CommentTok{<!-- IMPORTS -->}
\KeywordTok{<link}\OtherTok{ rel=}\StringTok{"import"}\OtherTok{ href=}\StringTok{"facebook-oauth.html"}\OtherTok{ async}\KeywordTok{>}
\KeywordTok{<link}\OtherTok{ rel=}\StringTok{"import"}\OtherTok{ href=}\StringTok{"mobile-template.html"}\OtherTok{ async}\KeywordTok{>}
\NormalTok{...}

\CommentTok{<!-- VIEW -->}
\KeywordTok{<sign-in-styling>}
  \KeywordTok{<template}\OtherTok{ id=}\StringTok{"highres"}\KeywordTok{>}
    \KeywordTok{<style>}\FloatTok{...}\KeywordTok{</style>}
    \KeywordTok{<div}\OtherTok{ id=}\StringTok{"content"}\KeywordTok{>}
      \KeywordTok{<facebook-oauth>}\NormalTok{Yet another element}\KeywordTok{</facebook-oauth>}
      \KeywordTok{<slot}\OtherTok{ name=}\StringTok{"form"}\KeywordTok{></slot>}
      \KeywordTok{<slot}\OtherTok{ name=}\StringTok{"button"}\KeywordTok{></slot>}
    \KeywordTok{</div>}
  \KeywordTok{</template>}
  
  \KeywordTok{<template}\OtherTok{ id=}\StringTok{"lowres"}\KeywordTok{>}\NormalTok{...}\KeywordTok{</template>}
  \KeywordTok{<mobile-template>}\NormalTok{Template custom element}\KeywordTok{</template>}
\KeywordTok{</sign-in-styling>}

\KeywordTok{<script>}
  \KeywordTok{class} \NormalTok{SignInStyling }\KeywordTok{extends} \NormalTok{HTMLElement }\OperatorTok{\{}
    \AttributeTok{constructor}\NormalTok{() }\OperatorTok{\{}
      \KeywordTok{super}\NormalTok{()}\OperatorTok{;}
      \KeywordTok{this}\NormalTok{.}\AttributeTok{attachShadow}\NormalTok{(}\OperatorTok{\{}\DataTypeTok{mode}\OperatorTok{:} \StringTok{'open'}\OperatorTok{\}}\NormalTok{)}\OperatorTok{;}
      \KeywordTok{this}\NormalTok{.}\VariableTok{shadowRoot}\NormalTok{.}\AttributeTok{appendChild}\NormalTok{(}\VariableTok{temp}\NormalTok{.}\AttributeTok{content}\NormalTok{)}\OperatorTok{;}
      \CommentTok{// f.e. append templates according to window.innerWidth}
      \CommentTok{// style related JS altering}
      \CommentTok{// this.getAttribute('theme') for configuration}
    \OperatorTok{\}}
    \AttributeTok{attributeChangedCallback}\NormalTok{(attrName}\OperatorTok{,} \NormalTok{oldVal}\OperatorTok{,} \NormalTok{newVal) }\OperatorTok{\{}
      \CommentTok{// react on changing attributes}
    \OperatorTok{\}}
  \OperatorTok{\}}
  \VariableTok{customElements}\NormalTok{.}\AttributeTok{define}\NormalTok{(}\StringTok{'sign-in-styling'}\OperatorTok{,} \NormalTok{SignInStyling)}\OperatorTok{;}
\OperatorTok{<}\SpecialStringTok{/script>}
\end{Highlighting}
\end{Shaded}

As mentioned before, HTML templates provide an opportunity to define
\textbf{CSS modules}. Tags like \texttt{\textless{}style\textgreater{}}
and \texttt{\textless{}slot\textgreater{}} create a kind of stencil
filled by content from the \texttt{index.html}. As it is possible to
expand templates in place it could also be possible to define
\textbf{template components} like the \texttt{mobile-template} to
atomize the component further and to increase code brevity.

Overall, the \texttt{\textless{}shop-checkout\textgreater{}} should
provide a taste of web components and their ingredients in practical
usage. On a bigger scale things would certainly look different and less
static than in the given example. Still, the microservice approach is
intended to be somewhat visible in this example.

\section{5. Thinking further}\label{thinking-further}

In the search for an independent \textbf{browsernative} approach this
paper left out third-party libraries so far. Nevertheless, it should be
clear to this point that the example in the last section is far away
from being usable in production. At first, from a technical perspective
web components are relatively young and at the time of writing only
usable in \textbf{newer Chrome and Opera browsers}. Most probably it
will take a long time until native web components can be safely used
without additional legacy support. The troublesome \emph{HTML imports}
spec will certainly need even longer which inevitably opens up the
question if web components are useful from today onwards. And second, do
web components align with the notion of simplicity if too many downsides
and possible pitfalls require attention?!

Any profound discussion about web components would be incomplete without
touching component alternatives like \textbf{React}. In the last couple
years React introduced new concepts to web development like the idea of
state, passing data as properties and declarative event management just
to name a few. Those design decisions were made to simplify browser
development with full browser support right from the start. Overall,
React's perception about UI development created much of a hype around
the library and technologies like the virtual DOM while web components
never caught up in momentum at a comparable rate. According to Github,
Google's Polymer project is even six months older than React but has
less than one third of its stars. Other notable web component projects
like Bosonic\footnote{https://github.com/bosonic/bosonic/} or Mozilla's
X-tag\footnote{https://github.com/x-tag/core} are more dead than alive,
too.

It seems like the web developing community in a broader sense values
\textbf{standardized procedures, tooling and browser support} in
frameworks like React (and many others to be fair). Previously mentioned
web component libraries clearly missed a real selling point compared to
React or Angular. Providing guidelines, simpler syntax and features are
missed out or are too cumbersome to use at all. For the future it
remains unclear if web components eventually gain wider adoption within
the UI developing community or will be adopted as low-level technology
for framework development emphasized by Sebastian Markbage, one of the
React creators.{[}21{]}

Even though the drawbacks seems heavy weight, there are notable attempts
to bring web components into production. A very good example is the
``reactish'' library Skate\footnote{https://github.com/skatejs/skatejs/}.
Following the functional rendering model from React, it combines native
methods with additional functionalities known from React like
declarative event management. Given the focus on native technologies it
only weights around 4kb (minified and gzipped) which is ten times less
than React. \textbf{Micro UI frameworks} like Skate could certainly be
the near future of web development, offering a real selling point.
Previously mentioned specs could lead to a modular, pluggable building
pipelines offering the same features and tooling as React already does.
After all, web components fit better with the microservice
\textbf{decentralization aspect} than `walled garden' frameworks.

The role of web components may not result in an isolated implementation
but move towards a common denominator for future Reacts, Angulars, Vues
etc. This makes \textbf{web components worth considering them as
recyclable technology}. The more mature UI framework Riot already
expressed its intention to gradually develop towards web
components.\footnote{http://riotjs.com/compare/\#web-components}

\hypertarget{refs}{}
\hypertarget{ref-Hickey2012}{}
{[}1{]} R. Hickey, ``Rails conf 2012 keynote: Simplicity matters by rich
hickey.'' 2012 {[}Online{]}. Available:
\url{https://www.youtube.com/watch?v=rI8tNMsozo0\&t=46s}

\hypertarget{ref-Filloux2016}{}
{[}2{]} F. Filloux, ``Bloated html, the best and the worse,'' 2016
{[}Online{]}. Available:
\url{https://mondaynote.com/bloated-html-the-best-and-the-worse-cac6eb06496d}

\hypertarget{ref-Baldwin2006}{}
{[}3{]} C. Y. Baldwin and K. B. Clark, ``Modularity in the Design of
Complex Engineering Systems,'' in \emph{Complex engineered systems:
Science meets technology}, D. Braha, A. A. Minai, and Y. Bar-Yam, Eds.
Berlin, Heidelberg: Springer Berlin Heidelberg, 2006, pp. 175--205
{[}Online{]}. Available:
\href{http://dx.doi.org/10.1007/3-540-32834-3\%7B/_\%7D9}{http://dx.doi.org/10.1007/3-540-32834-3\{\textbackslash{}\_\}9}

\hypertarget{ref-Newman2015}{}
{[}4{]} S. Newman, \emph{Building microservices}. O'Reilly Media, Inc.,
2016.

\hypertarget{ref-Fowler2014}{}
{[}5{]} M. Fowler and J. Lewis, ``Microservices: A definition of this
new architectural term,'' Jan. 2014 {[}Online{]}. Available:
\url{http://www.martinfowler.com/articles/microservices.html}

\hypertarget{ref-Martin}{}
{[}6{]} R. C. Martin, ``The single responsibility principle.''
{[}Online{]}. Available:
\url{http://programmer.97things.oreilly.com/wiki/index.php/The_Single_Responsibility_Principle}

\hypertarget{ref-Kearney2016}{}
{[}7{]} M. Kearney, ``Measure performance with the rail model.'' 2016
{[}Online{]}. Available:
\url{https://developers.google.com/web/fundamentals/performance/rail}

\hypertarget{ref-Conway1968}{}
{[}8{]} M. E. Conway, ``How do committees invent?'' 1968 {[}Online{]}.
Available: \url{http://www.melconway.com/Home/Committees_Paper.html}

\hypertarget{ref-Boner2012}{}
{[}9{]} J. Bonér, ``Latency numbers every programmer should know.'' 2012
{[}Online{]}. Available: \url{https://gist.github.com/jboner/2841832}

\hypertarget{ref-Issa2016}{}
{[}10{]} B. Issa, ``The way of the web.'' Polymer Summit 2016, Oct-2016
{[}Online{]}. Available:
\url{https://www.youtube.com/watch?v=8ZTFEhPBJEE}

\hypertarget{ref-Victor2012}{}
{[}11{]} B. Victor, ``Inventing on principle.'' 2012 {[}Online{]}.
Available: \url{https://vimeo.com/36579366}

\hypertarget{ref-Lawson2016}{}
{[}12{]} N. Lawson, ``Progressive enhancement isn't dead, but it smells
funny.'' 2016 {[}Online{]}. Available:
\url{https://nolanlawson.com/2016/10/13/progressive-enhancement-isnt-dead-but-it-smells-funny/}

\hypertarget{ref-Rota2015}{}
{[}13{]} A. Rota, ``React.js conf 2015 - the complementarity of react
and web components.'' 2015 {[}Online{]}. Available:
\url{https://www.youtube.com/watch?t=124\&v=g0TD0efcwVg}

\hypertarget{ref-Buchner2016}{}
{[}14{]} D. Buchner, ``Demythstifying web components,'' 2016
{[}Online{]}. Available:
\url{http://www.backalleycoder.com/2016/08/26/demythstifying-web-components/}

\hypertarget{ref-vanKesteren2014}{}
{[}15{]} A. van Kesteren, ``Mozilla and web components: Update,'' 2014
{[}Online{]}. Available:
\url{https://hacks.mozilla.org/2014/12/mozilla-and-web-components/}

\hypertarget{ref-Bidelman2016}{}
{[}16{]} E. Bidelman, ``Custom elements v1: reusable web components.''
2016 {[}Online{]}. Available:
\url{https://developers.google.com/web/fundamentals/primers/customelements/}.
{[}Accessed: 01-Dec-2016{]}

\hypertarget{ref-HTML}{}
{[}17{]} \emph{HTML living standard --- last updated 11 january 2017}.
{[}Online{]}. Available: \url{https://html.spec.whatwg.org/multipage/}

\hypertarget{ref-Bidelman2016shadow}{}
{[}18{]} E. Bidelman, ``Shadow dom v1: Self-contained web components.''
2016 {[}Online{]}. Available:
\url{https://developers.google.com/web/fundamentals/getting-started/primers/shadowdom}

\hypertarget{ref-Bidelman2010}{}
{[}19{]} E. Bidelman, ``The basics of web workers.'' 2010 {[}Online{]}.
Available: \url{https://www.html5rocks.com/en/tutorials/workers/basics/}

\hypertarget{ref-Abramov2015}{}
{[}20{]} D. Abramov, ``Presentational and Container Components --
Medium.'' 2015 {[}Online{]}. Available:
\url{https://medium.com/@dan_abramov/smart-and-dumb-components-7ca2f9a7c7d0}.
{[}Accessed: 01-Dec-2016{]}

\hypertarget{ref-Rauschmayer2015}{}
{[}21{]} D. A. Rauschmayer, ``Tree-shaking with webpack 2 and babel 6.''
2015 {[}Online{]}. Available:
\url{http://www.2ality.com/2015/12/webpack-tree-shaking.html}

\end{document}
