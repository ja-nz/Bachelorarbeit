\documentclass[]{assets/latex/ieee}
\usepackage{lmodern}
\usepackage{amssymb,amsmath}
\usepackage{ifxetex,ifluatex}
\usepackage{fixltx2e} % provides \textsubscript
\ifnum 0\ifxetex 1\fi\ifluatex 1\fi=0 % if pdftex
  \usepackage[T1]{fontenc}
  \usepackage[utf8]{inputenc}
\else % if luatex or xelatex
  \ifxetex
    \usepackage{mathspec}
  \else
    \usepackage{fontspec}
  \fi
  \defaultfontfeatures{Ligatures=TeX,Scale=MatchLowercase}
\fi
% use upquote if available, for straight quotes in verbatim environments
\IfFileExists{upquote.sty}{\usepackage{upquote}}{}
% use microtype if available
\IfFileExists{microtype.sty}{%
\usepackage{microtype}
\UseMicrotypeSet[protrusion]{basicmath} % disable protrusion for tt fonts
}{}
\usepackage[unicode=true]{hyperref}
\hypersetup{
            pdfborder={0 0 0},
            breaklinks=true}
\urlstyle{same}  % don't use monospace font for urls
\IfFileExists{parskip.sty}{%
\usepackage{parskip}
}{% else
\setlength{\parindent}{0pt}
\setlength{\parskip}{6pt plus 2pt minus 1pt}
}
\setlength{\emergencystretch}{3em}  % prevent overfull lines
\providecommand{\tightlist}{%
  \setlength{\itemsep}{0pt}\setlength{\parskip}{0pt}}
\setcounter{secnumdepth}{0}
% Redefines (sub)paragraphs to behave more like sections
\ifx\paragraph\undefined\else
\let\oldparagraph\paragraph
\renewcommand{\paragraph}[1]{\oldparagraph{#1}\mbox{}}
\fi
\ifx\subparagraph\undefined\else
\let\oldsubparagraph\subparagraph
\renewcommand{\subparagraph}[1]{\oldsubparagraph{#1}\mbox{}}
\fi

% set default figure placement to htbp
\makeatletter
\def\fps@figure{htbp}
\makeatother


\date{}

\begin{document}

\section{Introduction}\label{introduction}

\begin{quote}
Simplicity is prerequisite for reliability. - Edsger W. Dijkstra
\end{quote}

Der Erfinder der Programmiersprache Clojure, Rich Hickey, ist ohne
Zweifel eine Koryphäe auf seinem Gebiet, der Strukturierung von
komplexen Systemen. In einer vielbeachteten Keynote aus dem Jahr 2012
geht er auf etymologisch-philosophische Spurensuche nach dem Wort
\textbf{simplicity } aus Sicht eines Softwareentwicklers.\footnote{\href{https://www.youtube.com/watch?v=rI8tNMsozo0\&t=48s}{Rails
  Conf 2012 Keynote: Simplicity Matters by Rich Hickey}} Das Adjektiv
\emph{simple} hat demnach seinen Ursprung im lateinischen Wort
\emph{simplex}, was soviel wie \emph{einfach} oder \emph{einzeln}
bedeutet. In Gegensatz dazu stehen Eigenschaften wie \emph{complex} oder
\emph{multiplex}. Qualitative Software ist, so Hickey, vor allem simpel
- im Prozess, im Design, in der Struktur und in der Entwicklung.

Bezeichnend für diese Keynote ist, dass sie von Rich Hickey im Rahmen
einer Webentwicklerkonferenz gehalten wurde. Software für den Browser
war (und ist) seit langer Zeit maßgeblich geprägt von
\textbf{Komplexität} auf mehreren Ebenen. Im Laufe dieser Bachelorarbeit
werden diese Stru

\begin{itemize}
\tightlist
\item
  Das \emph{Document Object Model} als Rückgrad jeder Webapplikation ist
  hierarchisch strukturiert und deren Elemente damit keinesfalls
  unabhängig in ihrer Darstellung und Reihenfolge.
\item
  JavaScript als single-threaded Skriptsprache lässt sich schlecht ihn
  ihrem Verhalten isolieren, ist Fehleranfällig und hatte lange Zeit nur
  sehr wenig idiomatische Lösungsansätze für komplexe Probleme, wie
  beispielsweise Asynchronität
\item
  CSS ist geprägt von stetigem Überschreiben vorher definierter Regeln
  und verletzt damit Simplizität in ihrer Struktur, Design und dem
  Entwicklungsprozesses auf bester Art und Weise.
\end{itemize}

\begin{center}\rule{0.5\linewidth}{\linethickness}\end{center}

Betrachtet man andere populäre Webframeworks dieser Zeit, wie
beispielsweise React oder Angular, lässt sich ziemlich schnell ein
gemeinsames Designpattern ausmachen, wie Komponenten intern ihre
Zuständen verwalten. Der Facebook Entwickler Dan Abramov hat dieses
dichotome Pattern systematisch erfasst und in zwei Kategorien
eingeteilt.

Nach Abramov existieren zum einen Komponenten, die alleine für das
\textbf{Darstellen von Information} zuständig sind. Diese Komponenten
nehmen keinerlei Einfluss auf Informationen oder anderen Komponenten um
sich herum. Sie sind im wahrsten Sinne passiv und fremdgesteuert über
klar definierte Schnittstellen. Diese Komponenten sind häufig flexibel
einsetzbar und hochgradig wiederwerwendbar. Abramov nennt diese
Komponenten ``Presentational Components''.{[}1{]} Verortet man diese Art
von Komponenten innerhalb des \emph{MVC Pattern}, sind die Komponenten
reine \emph{View-Elemente}.

Im Gegensatz dazu stehen Komponenten, die für die \textbf{Verarbeitung
von Informationen} zuständig sind. Diese so genannten ``Container
Components'' haben oft einen internen Zustand, den sie verändern können
und an ihre Kindkomponenten weiterreichen können.{[}1{]} Im \emph{MVC
Pattern} handelt es sich um die \emph{Controller}. In der funktionalen
Programmiersprache Elm werden diese Elemente \emph{Updater} genannt, was
ihre Funktion noch besser umschreibt.

Ein üblicher eventgesteuerter Webservice setzt sich aus
unterschiedlichsten Komponenten zusammen, die wiederum
unterschiedlichste Eventlistener \& -emitter in sich subsummieren. Diese
inhärente Komplexität verlangt geradezu nach einer klaren,
deterministischen Struktur des Webservices, die das Zusammenspiel
orchestriert. In der Analogie des Orchesters gesprochen, benötigt der
Webservice (oder sogar die gesamte Webapplikation) einen Dirigenten, der
für die Steuerung verantwortlich ist.

\section{Store}\label{store}

\begin{verbatim}
   +-----------------------+
   |         Store         |
   +-----------------------+
       /\|  |
       
\end{verbatim}

die Änderungen des Zustands vorhersehbar aber auch zurückverfolgbar
macht.

\section{Microservices}\label{microservices}

In a nutshell a microservice is a small, autonomous service that works
together with other services seamlessly.{[}2, p. 2{]} While the concept
of microservices lacks a formal standardization there are certain
defacto standardized ideas emergine from this pattern. The purpose of
this section is to match those ideas against the manifestations of web
components.

First of all, from a technical perspective, a microservice reinforces
the \emph{Single Responsibility Principle} defined by Robert C. Martin:
``Gather together those things that change for the same reason and
separate those things that change for different reasons.''{[}3{]} An a
way this principle tackles another often cited design principle of the
\emph{seperation of concerns}. Web Components incorporate this principle
in multiple ways while still remaining flexible.

Most obvious ist the gathering of all related code under the umbrellar
of a single HTML tag. Grouping together HTML, JS and CSS Code in a safe,
sandboxed environment exposes the possibility to build more cohesive and
understandable services. In the typical global nature of web development
those three pillars are separated. This circumstance left the developer
switching back and forth between code bases trying to find a tricky way
to glue related parts together.

Secondly, the sub-standard \emph{custom elements} introduces so called
lifecycle methods and a getter/setter interface exposing the
functionality to the developer. Event handling, for example, can be
registered in place which is much more declarative than assigning event
listeners from the outside. Of course, this events can be pushed down to
nested tags, allowing an increasingly granular system design. This
approach will be explained further in the upcoming sections.

The concept of microservices incooperates not only a technical
perspective. Microservice patterns are a product of real-world
usage.{[}2, p. 1{]} In a real world we typical have to deal with the so
called \emph{Conway's Law}:

\begin{quote}
``organizations which design systems \ldots{} are constrained to produce
designs which are copies of the communication~structures~of these
organizations''. {[}4{]}
\end{quote}

Following this logic any company, whether its web-related or not, should
be devided in units grouped around a destinct business service to
optimise the workflow. Fowler and Lewis outlines this approach as an
``alignment of business capabilities''{[}5, p. 2{]} While this kind of
structure may be true for companies like Google or Amazon, there is a
vast majority of companies developing for the web which are grouped
around tasks.{[}6{]} A very common structure is formed by the technology
stack: UX Designers, Frontend- \& Backend Developers.

The microservice approach models all those functions into
business-related units. Web components are one (but important) way to
tie up those diciplines as one component can host a single independent
business service. Combined with a flexible backend service these
components can be huge gain over the cumbersome functional
organizational approach.

Macroperspektive / Composition

http://alistair.cockburn.us/Hexagonal+architecture

MVC Pattern

\section{Progressive Enhancement}\label{progressive-enhancement}

\subsection{Chapter about progressive
enhancement}\label{chapter-about-progressive-enhancement}

title: Browsernative Microservices subtitle: Modulare Webarchitekturen
durch neue W3C-Spezifikationen author: Jan Peteler, FH
Würzburg-Schweinfurt, jan.peteler@student.fhws.de date: Januar 2017
lang: de-DE abstract: Webapplikationen lassen sich oft nur mit
zusätzlichen Abstraktionsebenen wie beispielsweise ``Frameworks''
verwirklichen. Neue Browserstandards ermöglichen eine native
Servicearchitektur, die direkt auf der Browser-Engine aufsetzt. Diese
Charakteristiken sollen systematisch erfasst werden. \ldots{}

\hypertarget{refs}{}
\hypertarget{ref-Abramov2015}{}
{[}1{]} D. Abramov, ``Presentational and Container Components --
Medium.'' 2015 {[}Online{]}. Available:
\url{https://medium.com/@dan_abramov/smart-and-dumb-components-7ca2f9a7c7d0}.
{[}Accessed: 01-Dec-2016{]}

\hypertarget{ref-Newman2015}{}
{[}2{]} S. Newman, \emph{Building microservices}. O'Reilly Media, Inc.,
2016 {[}Online{]}. Available:
\url{http://www.ebook.de/de/product/22539693/sam_newmann_building_microservices.html}

\hypertarget{ref-Martin}{}
{[}3{]} R. C. Martin, ``The single responsibility principle.''
{[}Online{]}. Available:
\url{http://programmer.97things.oreilly.com/wiki/index.php/The_Single_Responsibility_Principle}

\hypertarget{ref-Conway1968}{}
{[}4{]} M. E. Conway, ``How do committees invent?'' 1968 {[}Online{]}.
Available: \url{http://www.melconway.com/Home/Committees_Paper.html}

\hypertarget{ref-Fowler2015}{}
{[}5{]} M. Fowler and J. Lewis, ``Microservices: Nur ein weiteres
konzept in der softwarearchitektur oder mehr?'' \emph{OBJEKTspektrum},
Jan. 2015.

\hypertarget{ref-Issa2016}{}
{[}6{]} B. Issa, ``The way of the web.'' Polymer Summit 2016, Oct-2016
{[}Online{]}. Available:
\url{https://www.youtube.com/watch?v=8ZTFEhPBJEE}

\end{document}
