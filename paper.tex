\documentclass[]{assets/latex/ieee}
\usepackage{lmodern}
\usepackage{amssymb,amsmath}
\usepackage{ifxetex,ifluatex}
\usepackage{fixltx2e} % provides \textsubscript
\ifnum 0\ifxetex 1\fi\ifluatex 1\fi=0 % if pdftex
  \usepackage[T1]{fontenc}
  \usepackage[utf8]{inputenc}
\else % if luatex or xelatex
  \ifxetex
    \usepackage{mathspec}
  \else
    \usepackage{fontspec}
  \fi
  \defaultfontfeatures{Ligatures=TeX,Scale=MatchLowercase}
\fi
% use upquote if available, for straight quotes in verbatim environments
\IfFileExists{upquote.sty}{\usepackage{upquote}}{}
% use microtype if available
\IfFileExists{microtype.sty}{%
\usepackage{microtype}
\UseMicrotypeSet[protrusion]{basicmath} % disable protrusion for tt fonts
}{}
\usepackage[unicode=true]{hyperref}
\hypersetup{
            pdftitle={Browsernative Microservices},
            pdfauthor={Jan Peteler, FH Würzburg-Schweinfurt, jan.peteler@student.fhws.de},
            pdfborder={0 0 0},
            breaklinks=true}
\urlstyle{same}  % don't use monospace font for urls
\usepackage{color}
\usepackage{fancyvrb}
\newcommand{\VerbBar}{|}
\newcommand{\VERB}{\Verb[commandchars=\\\{\}]}
\DefineVerbatimEnvironment{Highlighting}{Verbatim}{commandchars=\\\{\}}
% Add ',fontsize=\small' for more characters per line
\newenvironment{Shaded}{}{}
\newcommand{\KeywordTok}[1]{\textcolor[rgb]{0.00,0.44,0.13}{\textbf{{#1}}}}
\newcommand{\DataTypeTok}[1]{\textcolor[rgb]{0.56,0.13,0.00}{{#1}}}
\newcommand{\DecValTok}[1]{\textcolor[rgb]{0.25,0.63,0.44}{{#1}}}
\newcommand{\BaseNTok}[1]{\textcolor[rgb]{0.25,0.63,0.44}{{#1}}}
\newcommand{\FloatTok}[1]{\textcolor[rgb]{0.25,0.63,0.44}{{#1}}}
\newcommand{\ConstantTok}[1]{\textcolor[rgb]{0.53,0.00,0.00}{{#1}}}
\newcommand{\CharTok}[1]{\textcolor[rgb]{0.25,0.44,0.63}{{#1}}}
\newcommand{\SpecialCharTok}[1]{\textcolor[rgb]{0.25,0.44,0.63}{{#1}}}
\newcommand{\StringTok}[1]{\textcolor[rgb]{0.25,0.44,0.63}{{#1}}}
\newcommand{\VerbatimStringTok}[1]{\textcolor[rgb]{0.25,0.44,0.63}{{#1}}}
\newcommand{\SpecialStringTok}[1]{\textcolor[rgb]{0.73,0.40,0.53}{{#1}}}
\newcommand{\ImportTok}[1]{{#1}}
\newcommand{\CommentTok}[1]{\textcolor[rgb]{0.38,0.63,0.69}{\textit{{#1}}}}
\newcommand{\DocumentationTok}[1]{\textcolor[rgb]{0.73,0.13,0.13}{\textit{{#1}}}}
\newcommand{\AnnotationTok}[1]{\textcolor[rgb]{0.38,0.63,0.69}{\textbf{\textit{{#1}}}}}
\newcommand{\CommentVarTok}[1]{\textcolor[rgb]{0.38,0.63,0.69}{\textbf{\textit{{#1}}}}}
\newcommand{\OtherTok}[1]{\textcolor[rgb]{0.00,0.44,0.13}{{#1}}}
\newcommand{\FunctionTok}[1]{\textcolor[rgb]{0.02,0.16,0.49}{{#1}}}
\newcommand{\VariableTok}[1]{\textcolor[rgb]{0.10,0.09,0.49}{{#1}}}
\newcommand{\ControlFlowTok}[1]{\textcolor[rgb]{0.00,0.44,0.13}{\textbf{{#1}}}}
\newcommand{\OperatorTok}[1]{\textcolor[rgb]{0.40,0.40,0.40}{{#1}}}
\newcommand{\BuiltInTok}[1]{{#1}}
\newcommand{\ExtensionTok}[1]{{#1}}
\newcommand{\PreprocessorTok}[1]{\textcolor[rgb]{0.74,0.48,0.00}{{#1}}}
\newcommand{\AttributeTok}[1]{\textcolor[rgb]{0.49,0.56,0.16}{{#1}}}
\newcommand{\RegionMarkerTok}[1]{{#1}}
\newcommand{\InformationTok}[1]{\textcolor[rgb]{0.38,0.63,0.69}{\textbf{\textit{{#1}}}}}
\newcommand{\WarningTok}[1]{\textcolor[rgb]{0.38,0.63,0.69}{\textbf{\textit{{#1}}}}}
\newcommand{\AlertTok}[1]{\textcolor[rgb]{1.00,0.00,0.00}{\textbf{{#1}}}}
\newcommand{\ErrorTok}[1]{\textcolor[rgb]{1.00,0.00,0.00}{\textbf{{#1}}}}
\newcommand{\NormalTok}[1]{{#1}}
\usepackage[normalem]{ulem}
% avoid problems with \sout in headers with hyperref:
\pdfstringdefDisableCommands{\renewcommand{\sout}{}}
\IfFileExists{parskip.sty}{%
\usepackage{parskip}
}{% else
\setlength{\parindent}{0pt}
\setlength{\parskip}{6pt plus 2pt minus 1pt}
}
\setlength{\emergencystretch}{3em}  % prevent overfull lines
\providecommand{\tightlist}{%
  \setlength{\itemsep}{0pt}\setlength{\parskip}{0pt}}
\setcounter{secnumdepth}{0}
% Redefines (sub)paragraphs to behave more like sections
\ifx\paragraph\undefined\else
\let\oldparagraph\paragraph
\renewcommand{\paragraph}[1]{\oldparagraph{#1}\mbox{}}
\fi
\ifx\subparagraph\undefined\else
\let\oldsubparagraph\subparagraph
\renewcommand{\subparagraph}[1]{\oldsubparagraph{#1}\mbox{}}
\fi

% set default figure placement to htbp
\makeatletter
\def\fps@figure{htbp}
\makeatother


\title{Browsernative Microservices}
\providecommand{\subtitle}[1]{}
\subtitle{Modular web architecture through new W3C specifications}
\author{Jan Peteler, FH Würzburg-Schweinfurt, jan.peteler@student.fhws.de}
\date{Januar 2017}

\begin{document}
\maketitle
\begin{abstract}
Building complex web applications nowadays require additional layers of
abstraction and often heavily depend on proprietary frameworks. New
specifications build right into the browserengine provide a native
service API to overcome tricky abstraction constraints.
\end{abstract}

\section{Simplicity and the state of the
web}\label{simplicity-and-the-state-of-the-web}

\begin{quote}
Simplicity is prerequisite for reliability. - Edsger W. Dijkstra
\end{quote}

Computers can scale, humans can't. Ever since a program or complex
system made by humans has been constrained by humans mental
capabilities. Like in the analogy of juggling balls, our brain can just
``juggle'' a few things at a time. Rich Hickey, the inventor of the
programming language \emph{Clojure} gave an inspirational keynote on the
topic of \textbf{simplicity}.\footnote{\href{https://www.youtube.com/watch?v=rI8tNMsozo0\&t=46s}{Rails
  Conf 2012 Keynote: Simplicity Matters by Rich Hickey}} In every sphere
of a humans life, simplicity aligns perception with our mental
capacities.

Derived from the ancient Latin word \textbf{simplex}, simple can be
understood as ``literally, uncompounded or onefold''\footnote{\href{http://www.etymonline.com/index.php?term=simple}{Etymology
  Dictionary}} which points directly to the unidimensional aspect. While
complexity describes the multilayered und entangled nature of
conditions, simplicity empowers the human brain to reason about issues
in a straightforward manner. It certainly has some overlapping's with
easy, but while easy is more of a relative nature, simple can be laid
out as a objective manner and therefore universally applicable.

Software development is undoubtedly rich in complexity and full of
subtle pitfalls for the human brain. In a typical scenario, a piece of
software evolves over time in one or another \sout{opinionated}
direction. Layers of new abstractions wrestling with old legacy
abstractions and mutation becomes untraceable. Subtle bugs start to
creep in. Eventually the small piece of software may end up in a highly
complected monolith which will determine future design decisions to a
painful degree. Future strategies of the company/organization will be
highly determined by the current state in the need of ``keeping the
lights on''.

On the other side of this dystopian example, the truly agile system
architecture is laid out in a fine-grained manner. Software becomes
pluggable, mutations traceable and modules interchangeable. As Rich
Hickey argues, design decisions should be made under the
\textbf{impression of extending, substitution, moving, combining and
repurposing}. The ability to reason about the program at any given time
is crucial for future decisions and implementations. Recalling again the
unidimensional nature of simplicity.

The \textbf{state of the web} is certainly a different kind of
complexity beast. ``The web'' is coined to ``everything that runs in the
browser''. While simplicity in the backend is mostly a matter of
principles, any frontend developer is highly restricted on the highly
deterministic nature of the browser platform.

In the last four years the average transfer size of a webpage doubled to
currently around 2.5 MB.\footnote{\href{http://httparchive.org/trends.php}{HTTPArchive
  Trends}} Subtracting images, fonts or other content the size of HTML,
CSS and JS sums up to a total average of 550 kb. One character weights
around 1 byte which means an average webpage is delivering 550.000
character or around 125 pages of single-spaced text. Frederic Filloux
calculated the ratio of real content on different newspaper websites and
came to the conclusion, that only round about 5-6 \% of the transferred
characters made for human consumption.{[}1{]}

Having an 95 \% overhead is rather undesirable for both the consumers
and creators of the website. Since it's a widespread problem without a
single point of failure one can argue the platform itself is the
failure. By design, every pageload results in a monolithic DOM tree
managed by the browser engine. Whether rendering just a bunch of static
text nodes or an ever changing webapp the underlying global nature of
the DOM tree remains the same. Every additional service added to the
webapp embodies another fold of complexity.

In an non-deterministic runtime environment, encapsulation and
modularization is a typical pattern to make complexity manageable and
accommodate future uncertainty.{[}2, p. 1{]} Today many websites tend
towards becoming complex single-page applications (and therefore the
average JS payload is soaring too). Consequently, the demands to the
browser platform changed from a static page renderer to a
\textbf{complexe UI machine}. Under the current situation only
additional layers of abstraction can handle complexity.

In the recent years many \textbf{frameworks}, libraries and
methodologies approached the global nature of the DOM by scoping assets
and design rules into maintainable components. While the DOM can't be
scoped, JS can. Therefore many frameworks, like ReactJS, AngularJS or
VueJS just to name a few, ditched the old rule of separated HTML, CSS
and JS in favor of an additional layer of abstracted JS components
(containing content, markup and styling). Quiet often those frameworks
mimic a MVC pattern on top of the browser engine which is a reasonable
simple design pattern to build graphical user interfaces. While
frameworks are a valid approach to build scalable web applications they
remain highly opinionated, embody inherent complexity themselves and can
change and break over time. Furthermore, code inflation in front-end is
a crucial point for performance and third-party libraries are no
exception on that. A standardized way for creating complex UI interfaces
painlessly requires new build-in browser capabilities.

In 2013 thinkers, creators and browser vendors joined together to
propose \emph{The Extensible Web Manifesto}.\footnote{\href{https://extensiblewebmanifesto.org/}{The
  Extensible Web Manifesto}} The claim of the manifesto is to enhance
the current web platforms with new low-level capabilities. Those
features should empower creators of the web to write more declarative
code and therefore overcome known bottlenecks and artificial
abstractions. Four years later, the enhancement of JavaScript
leapfrogged and many new low-level APIs brought to life. With this new
APIs at hand a vivid web developer can create scoped and highly reusable
microservices without additional libraries even directly in the browser
console.

\emph{Disclaimer:} This paper introduces many new browser build-ins with
the focus on try and test. As the time of writing, examples can be tried
frictionless in the consoles of the latest versions of \textbf{Google
Chrome, Opera and Apple Safari}.\footnote{\href{http://jonrimmer.github.io/are-we-componentized-yet/}{Are
  we componentized yet?}} On Mozilla Firefox technologies work behind a
flag and Microsoft Edge implementation is unfortunately far behind. But
Browser implementation changes quickly and soon technology adoption
won't be an issue. Meanwhile new standards can be used through
\textbf{polyfills} even on legacy browsers.

\section{Microservices}\label{microservices}

Calling the case for \emph{Browsernative Microservices} brings up the
question about the concept of microservices in general. In fact the
concept of microservices has many facets, stretching beyond disciplines
and technical boundaries. It lacks a formal standardization but subsumes
certain ideas emergine from this pattern. As a primary source of truth
this paper relies on the work of Sam Newman, who has written a
comprehensive guide called \emph{Building Microservices} and the work of
Fowler and Lewis. The purpose of this section is to match their ideas
against the browser platform as targeted runtime.

In a nutshell a microservice is a small, autonomous service that works
together with other services seamlessly.{[}3, p. 2{]} or with the words
of Fowler and Lewis: ``\ldots{} the microservice architectural style is
an approach to developing a single application as a suite of small
services, each running in its own process and communicating with
lightweight mechanisms, often an HTTP resource API.''{[}4{]}
Microservices incorporate many other ideas, like \emph{domain-driven
design} where it pursues the incorporation of the real world in our
code.{[}3, p. 2{]} Or making use of \emph{continuous delivery} for
pushing software rapidly through \emph{automated deployment} mechanisms
into production.{[}4{]} And, last but not least, microservices utilizes
the idea of small teams with a lot of product knowledge working mostly
autonomous on their very own service with their very own set of tools
and techniques.

\subsection{Componentization via
Services}\label{componentization-via-services}

``A~\textbf{component}~is a unit of software that is independently
replaceable and upgradeable.'' {[}4{]} Components are the building
blocks of the microservice but in fact they even share a lot of
similarities just on a smaller scale. One of the similarity is the
\emph{loose coupling principle}: changing and deploying one service
shouldn't result in changing other parts of the system.{[}3, p. 30{]}.
Same goes for the component which should know as little as it needs
about the service to fulfil it's job.

Any reader of the paper coming from the ReactJS / Redux world this
concept of components might look familiar. Dan Abramov, the creator of
Redux, once defined a simple dichotomous pattern for creating UI
components. In his believe, there are \textbf{presentational components}
only related with the concern about \emph{how things look}. This
component literally doesn't know anything about the service in question
which makes the component highly flexible and reusable. They are
controlled solely from the outside, receiving data and dispatching
unbiased events on user interaction.{[}5{]}

On the other hand Dan Abramov outlines components which he calls
\textbf{containers}. A container component is concerned with \emph{how
things work}.{[}5{]} Containers acts as invisible wrappers around
presentational components. Their job is to fetch data from child nodes,
aggregating events, interacting with the model and push state back to
the presentational components. In contrast to their presentational
counterparts, containers can't life on its own without children.
Conceptually, they show another important principle of components and
microservices in general: \emph{high cohesion principle}.

Whether designing a microservice or it's components we want related
behavior sit together, and unrelated behavior to sit elsewhere.{[}3, p.
30{]} Code quality can be measured just by counting the places changes
in the code need to be made on changing behavior. In an arbitrary MVC
system a \texttt{button} might be inserted in the VIEW, the CONTROLLER
needs some adjustment and maybe the the MODEL, too. Three places for
adjustment is a reasonable easy task for the brain, everything more
violates the concept of simplicity.

Using libraries in web development is common sense. But compared to
libraries, a component service offers multiple advantages. A library is
only loosely coupled to the implementation and therefore hard to track
in functionality. Changing a library may result in an unforeseen amount
of time fixing implementations. It is not unusual to see websites
embodying different versions of the same library (like with JQuery).
Another issue with libraries is dead code elimination which means the
process of removing code that is never going to be executed. Newer build
tools for the web, like Webpack 2 or Rollup offer this feature which
relies heavily on the static structure on ES6 modules.{[}6{]} Contrary
libraries for the browsers are traditionally ``shipped'' as
\textbf{immediately-invoked functions}. Bootstrapping those closed
functions is much harder to archive as even dead code is actually
executed on load.

Web components are necessary static ES6 modules and any capable bundler
can choose which component is needed at runtime. Even more granular, a
good web component is composed of static JS blocks which can be strapped
away too.

A web component is self-contained which means it embodys all needed
functionality to get it's job done. Therefore it has a much better
evolution mechanism in the service contracts. Changing functionality
won't break other components. A component can progressively enhanced
which guarantees functionality throughout different versions.

Compared to libraries componentized services offers a more explicit
interface.{[}4{]} While the functionality of a library needs
documentation, a component functionality is exposed via the components'
signature which comes in the fashion of an HTML element. For example, a
button can consume attributes and becomes a primary buttons just by
assignment.

\begin{Shaded}
\begin{Highlighting}[]
\KeywordTok{<my-botton}\OtherTok{ isPrimary}\KeywordTok{>}\NormalTok{Primary}\KeywordTok{</my-botton>}
\end{Highlighting}
\end{Shaded}

A microservice reinforces the \emph{Single Responsibility Principle}
defined by Robert C. Martin: ``Gather together those things that change
for the same reason and separate those things that change for different
reasons.''{[}7{]}

\subsection{Organized around Business
Capabilities}\label{organized-around-business-capabilities}

\begin{quote}
``organizations which design systems \ldots{} are constrained to produce
designs which are copies of the communication~structures~of these
organizations''. {[}8{]}
\end{quote}

Emphasizing the human factor in microservices is a key feature.
Microservices are a product of real-world usage.{[}3, p. 1{]} Instead of
splitting team structures along the technology stack (UI Experts
-\textgreater{} Middleware -\textgreater{} Database) a microservice
approach model teams around \textbf{business capabilities}.{[}4{]}
Consequently every team is capable of planning, designing, implementing,
testing and maintaining their very own microservice. Along the
technology stack every member gains high competence about the
functionality of the service.

Real-world domains tend to be complex and multifaceted. To unfold their
complexity, domains can subdivided into bounded contexts.{[}3, p. 31{]}
For example, dealing with customers is a business domain. Customers
again have different contexts depending on their demands. One context
can be sales, another context could be support. Every context makes
different assumptions about the underlying model. Each bounded context
draws an explicit interface where it decides what models to share with
other contexts.{[}3, p. 30{]} Each context can derived into multiple
microservices or, talking about \emph{browsernative microservices},
interfaces to the customers.

By assigning service responsibility to a team, the so called
\textbf{Definition of Done} (DoD) shifts from ``accomplishing projects''
to ``accomplishing products''. This new paradigm not only changes the
administrative overhead like budgeting or resource allocation. It
creates a kind of responsibility connection from the team to the
service. Expectedly, those teams are more motivated within their very
own service and exhibit a more sophisticated iteration time.{[}4{]}

For many companies working in the spheres of the internet the
client-side is highly important for their business. In fact, business
goals and capabilities can be derived from the front-end needs. The
state of the web is not only a story of numerous artifacts, it is also a
story of an highly fragmented market along devices, operating systems,
differing sizes and functionalities. Different devices again have
different assumptions about the technology stack. Splitting teams along
the stack results in an slow paced back and forth negotiation for every
change to be made. As browser technologies, design guidelines and
devices change in a fast paced manner it makes absolutely sense to shift
responsibility towards the teams altogether. A \emph{browsernative
microservice} embraces commitment over its whole lifecycle.

\subsection{Smart endpoints and dumb
pipes}\label{smart-endpoints-and-dumb-pipes}

Microservices for the browsers aim for side-effects like changing the UI
given to input parameters or emitting browser events. An isolated
microservice won't make too much sense after all. To ensure
communication between services, flexible yet powerful communication
channels must be established.

\emph{Smart endpoints and dumb pipes} is coined to the approach of
designing communication mostly decoupled and as cohesive as
possible.{[}4{]} In the analogy to the real world a message channel
should look like sending a letter: Two smart endpoints (sender and
receiver) and a mostly unified ``dumb'' letter format and channel. Each
endpoint owns their specific domain logic. In the browsernative context
we already emphasized the role of presentational components and
containers. The ``smartness'' of presentational components is altering
the UI and reacting on users input. The logic of a container follows the
idea of an filter in a Unix sense - receiving a request, applying logic
as appropriate and producing a response.{[}4{]}

Cited earlier in this paper microservices often rely on simple HTTP
request-response with resource API's and lightweight messaging as
communication protocols.{[}4{]} Those technologies highly accessible and
widely adopted and most probably usable for a browsernative service. In
the implementation part of this paper a reader will explore the
combination of REST protocols and a lightweight browser message bus in
action.

\subsection{Decentralization}\label{decentralization}

Decentralization is a meta concept of microservices. As written earlier
decentralization is part of the organizational structure by assigning
service responsibility solely to the team. Or by gluing independent
service bricks to a whole distributed system talking over the network
protocols.

This distributed nature allows teams to create their own technology
stack, tools and services designed in the spirit of language- and
platform independence - and share their knowledge with other
parties.{[}4{]} In the recent years many big companies like Facebook,
Google, Netflix and others followed that spirit and published their
ideas and implementations open source. The previously mentioned ReactJS
for example is a brainchild of Facebook. In fact, many tools and
techniques are byproduct of vital interaction of concrete domain
problems and their implementations.

The spirit of freedom can't be applied to \emph{browsernative
microservices} as the browser and its underlying DOM will be the
limitation factor. Still there are some decentralization aspects around
creating microservices for the browser worth to mention.

JS is the widely accepted language of the web. Nevertheless, a
microservice engineering team might choose another language for
different reasons. Transpiling languages to JS as target language isn't
exotic anymore. Languages like TypeScript, ClojureScript or PureScript
compile to JS even exclusively. Once web components hit a critical mass
there will be most likely some library support or foreign function
interfaces towards ES6 modules (which are mandatory for the new
specifications). With the rise of WebAssembly, a new low-level
programming language for the browser, the determination on JS will
hypothetical deteriorate and new quasi native languages for the web
might gain traction.

Another more real life decentralization aspect derives from the easiness
of deployment browsernative microservices which makes them perfectly for
open source. They are small, leightweight and can be tested live in the
browser without any additional tooling.
\href{https://webcomponents.org}{Webcomponents.org} for example is a
registry for many ready-to-use components where even Google shares a lot
of their material design elements.

Last bot not least, decentralized governance in a web native context
will be achieved just because of the encapsulation of the service.
Contrary to the HTML, CSS and JS split model, the new autonomous service
exhibits some sort of decentralization in creating the service.

\subsection{Infrastructure Automation}\label{infrastructure-automation}

\subsection{Design for failure}\label{design-for-failure}

\subsection{Evolutionary Design}\label{evolutionary-design}

Shift of paradigms / BFF / Platform agnostic / changes in infrastructure
like APIs Databanks / Deployment

Most obvious ist the gathering of all related code under the umbrellar
of a single HTML tag. Grouping together HTML, JS and CSS Code in a safe,
sandboxed environment exposes the possibility to build more cohesive and
understandable services. In the typical global nature of web development
those three pillars are separated. This circumstance left the developer
switching back and forth between code bases developing a tricky (and
sometimes biased) way to glue related parts together.

Secondly, the sub-standard \emph{custom elements} introduces so called
lifecycle methods and a getter/setter interface exposing the
functionality to the developer. Event handling, for example, can be
registered in place which is much more declarative than assigning event
listeners from the outside. Of course, this events can be pushed down to
nested tags, allowing an increasingly granular system design. This
approach will be explained further in the upcoming sections.

Following this logic any company, whether it is web-related or not,
should be devided in units grouped around a destinct business service to
optimise the workflow. Fowler and Lewis outlines this approach as an
``alignment of business capabilities''{[}4{]} While this kind of
structure may be true for companies like Google or Amazon, there is a
vast majority of companies developing for the web which are grouped
around tasks.{[}9{]} A very common structure is formed by the technology
stack (UX Designers, Frontend- \& Backend Developers) or by separating
teams along the product lifecycle (development, testing, deployment).

Advocators from the microservice approach propose a different model.
best described by . Web components are one (but important) way to tie up
those diciplines as one component can host a single independent business
service. Combined with a flexible backend service these components can
be huge gain over the cumbersome functional organizational approach.

The ideas transcending from the microservice approach offers plenty of
choices and decisions how to proceed with designing a program or to
structure a process.

\section{W3C specifications}\label{w3c-specifications}

For building a native microservice running on the ``bare-metal'' browser
engine requires a bunch of new specifications and assumptions. Most
importantly the quasi specification \textbf{Web Components} is needed.
\emph{Web Components} is not a real standard. It's an amalgam of APIs
from multiple w3c specs which can be used independently, too. A
webdeveloper may choose one spec and embrace the freedom in architecture
which can be combined with other frameworks/libraries.

Depending on the context, some people argue for only two specs which
essentially make it possible to create a scoped component but not caring
too much on it's distribution{[}10{]}. Some people prefer the three
specs {[}11{]}, but the majority advocating the four specs variant,
which is listed on \sout{the official}
\href{http://webcomponents.org}{webcomponents.org} website. For the
purpose of this article, all four specs will be discussed briefly to
provide a rough understanding. It is not meant to cover all bits and
pieces.

\subsection{\texorpdfstring{Custom Elements
\href{http://w3c.github.io/webcomponents/spec/custom/}{(w3c)}}{Custom Elements (w3c)}}\label{custom-elements-w3c}

\emph{Custom elements} are the fundamental building blocks for web
components. Essentially, they provide a way to create custom HTML tags
enriched with behavior, design and functionality. An obligatory
\textbf{HelloWorld} will help to grasp the spec:

\begin{Shaded}
\begin{Highlighting}[]
\OperatorTok{>} \VariableTok{main}\NormalTok{.}\AttributeTok{js}
\KeywordTok{class} \NormalTok{HelloWorld }\KeywordTok{extends} \NormalTok{HTMLElement }\OperatorTok{\{}
 \AttributeTok{constructor}\NormalTok{() }\OperatorTok{\{}
  \KeywordTok{super}\NormalTok{()}\OperatorTok{;} \CommentTok{// mandatory!}
  \KeywordTok{this}\NormalTok{.}\AttributeTok{onclick} \OperatorTok{=} \NormalTok{e }\OperatorTok{=>} \AttributeTok{alert}\NormalTok{(}\StringTok{"hello"}\NormalTok{)}\OperatorTok{;}
  \KeywordTok{this}\NormalTok{.}\AttributeTok{addEventListener}\NormalTok{(...)}\OperatorTok{;}
 \OperatorTok{\}}
\OperatorTok{\}}
\VariableTok{customElements}\NormalTok{.}\AttributeTok{define}\NormalTok{(}\StringTok{'hello-world'}\OperatorTok{,} \NormalTok{HelloWorld)}
\end{Highlighting}
\end{Shaded}

\begin{Shaded}
\begin{Highlighting}[]
\NormalTok{> index.html}
\KeywordTok{<hello-world>}\NormalTok{say hello}\KeywordTok{</hello-world>}
\end{Highlighting}
\end{Shaded}

Most obvious, this spec relies on the new \emph{ES6 Class Syntax} in
favor of the original prototype-based inheritance model.
``Extending~\texttt{HTMLElement}~ensures the custom element inherits the
entire DOM API and means any properties/methods that you add to the
class become part of the element's DOM interface.''{[}12{]} Like any
other \emph{ES6 class}, \emph{custom element}s can be sub-classed
further on with the typical \texttt{extends} inheritance.

The beauty of \emph{custom elements} comes with the keyword
\texttt{this} which points to the element itself. Instead of querying
and assigning behavior AFTER creation of the node in question
\texttt{this} functionality allows a \textbf{declarative programming
style}. Assigning functionality happens right in place BEFORE creation
or insertion of the DOM. The so called \emph{fat-arrow}
(\texttt{=\textgreater{}}) is just a new feature of ES6 and nothing more
than an anonymous \texttt{function()}.

After definition, the element needs to be registered in the new global
build-in \texttt{customElements} with an tag name like
\texttt{\textless{}hello-world\textgreater{}}. Mind the dash inside the
tag name to conform the spec. Finally, the new element can go live
inside the HTML Document \texttt{index.html}.

\subsubsection{Lifecycle methods}\label{lifecycle-methods}

In addition to the \texttt{constructor()}, the spec defines so called
\emph{lifecycle callbacks} for controlling the \textbf{behaviour in the
DOM}. Many popular frameworks like ReactJS or AngularJS rely on similar
approaches:

\begin{itemize}
\tightlist
\item
  \texttt{connectedCallback()}\\
  Called upon the time of \emph{connecting or upgrading the node} which
  means the moment the node is rendered inside the DOM. Typically this
  method is called straight after the \texttt{constructor()} on insert.
  This block of code contains setup code, such as fetching resources or
  rendering elements according to attributes.{[}12{]} For performance
  issues it's highly preferable to put much code in here.
\item
  \texttt{disconnectedCallback()}\\
  Called upon the time of \emph{node removal}. Cleanup code like
  removing eventListeners or disconnecting websockets can be put here.
\item
  \texttt{attributeChangedCallback(attrName,\ oldVal,\ newVal)}\\
  This method provides an \emph{Onchange handler} that runs for certain
  attributes called with three values as defined in the signature. It is
  meant to control an elements' transition from on \texttt{oldVal} to a
  \texttt{newVal}. Due to performance issues, this callback is only
  triggered for attributes registered in an \emph{observedAttributes}
  array.
\item
  \texttt{adoptedCallback()}\\
  Called when moving the node \emph{between documents}.
\end{itemize}

\subsubsection{Custom attributes}\label{custom-attributes}

As previously mentioned, the custom elements must \texttt{extend} the
\texttt{HTMLElement}. Consequently, the new element inherits properties
and methods from it (and it's parent \texttt{Element}) and things like
\texttt{id,\ class,\ addEventListner,\ ...} run out-of-the-box.
Additionally, it is possible to define custom attributes using the
\emph{custom elements'} \textbf{getter / setter interface} to steer the
behavior of the element.

\begin{Shaded}
\begin{Highlighting}[]
\OperatorTok{>} \VariableTok{main}\NormalTok{.}\AttributeTok{js}
\KeywordTok{class} \NormalTok{HelloWorld }\KeywordTok{extends} \NormalTok{HTMLElement }\OperatorTok{\{}
 \AttributeTok{constructor}\NormalTok{() }\OperatorTok{\{}\NormalTok{...}\OperatorTok{\}}
 \NormalTok{set }\AttributeTok{sayhello}\NormalTok{(val) }\OperatorTok{\{}
  \KeywordTok{this}\NormalTok{.}\AttributeTok{_hello} \OperatorTok{=} \NormalTok{val}\OperatorTok{;}
  \VariableTok{console}\NormalTok{.}\AttributeTok{log}\NormalTok{(}\KeywordTok{this}\NormalTok{.}\AttributeTok{_hello}\NormalTok{)}\OperatorTok{;}
 \OperatorTok{\}}
 \NormalTok{get }\AttributeTok{sayhello}\NormalTok{() }\OperatorTok{\{}
  \ControlFlowTok{return} \KeywordTok{this}\NormalTok{.}\AttributeTok{_hello}\OperatorTok{;}
 \OperatorTok{\}}
\OperatorTok{\}}\NormalTok{)}\OperatorTok{;}
\VariableTok{customElements}\NormalTok{.}\AttributeTok{define}\NormalTok{(}\StringTok{'hello-world'}\OperatorTok{,} \NormalTok{HelloWorld)}\OperatorTok{;}
\CommentTok{// Instantiation}
\KeywordTok{var} \NormalTok{el }\OperatorTok{=} \KeywordTok{new} \AttributeTok{HelloWorld}\NormalTok{()}\OperatorTok{;}
\VariableTok{el}\NormalTok{.}\AttributeTok{sayhello} \OperatorTok{=} \StringTok{"earth"}\OperatorTok{;}
\VariableTok{el}\NormalTok{.}\AttributeTok{sayhello}\OperatorTok{;}\CommentTok{//"earth"}
\end{Highlighting}
\end{Shaded}

While getters and setters work great in the JS world they fail crossing
the boundaries to the corresponding HTML node. Declaring
\texttt{\textless{}hello-world\ sayhello="mars"\textgreater{}\textless{}/hello-world\textgreater{}}
would't work in the previous setup. A common workaround is archived by
using the previous mentioned \texttt{attributeChangedCallback} lifecycle
method to \textbf{reflect changing HTML attributes to JS} and/or map JS
attributes to HTML with \texttt{this.setAttributes(...)} respectively.
On \textbf{insertion time} html attributes might raise their hand with
\texttt{this.hasAttributes(...)} and \texttt{this.getAttributes(...)}.
Native DOM properties will reflect their values between HTML and JS
automatically.{[}13, Para. 2.6.1{]}

Concluding this section, a reader might already discover the
\textbf{mental model} behind \emph{web compontents}. A custom element is
similar to a named function where attributes treated as \textbf{input
variables}. In the hierarchical nature of DOM, input can occur either
top-down via assignments and bottom-up via captured events. The same
goes true when talking about output. Even though it seems obvious, it
might be helpful to keep this point in mind.

\subsubsection{Customized build-in
elements}\label{customized-build-in-elements}

One aspect didn't mentioned yet is the possibility of creating
sub-classes of \textbf{build-in elements} by extending the native
Interfaces like the \texttt{HTMLButtonElement} interface. While this
functionality is perfectly spec'd it is strongly rejected by some
browser vendors.\footnote{https://github.com/w3c/webcomponents/issues/509}
Most likely the spec will change in future in one or other way on this
issue and therefore \textbf{customized build-in elements} left out of
this paper intentionally.

\subsection{\texorpdfstring{Shadow DOM
\href{http://w3c.github.io/webcomponents/spec/shadow/}{(w3c)}}{Shadow DOM (w3c)}}\label{shadow-dom-w3c}

The second most important concept of \emph{web components} rewards to
the \emph{shadow DOM} spec. In terms of complexity this spec outpacing
all others by far. A \emph{shadow DOM} is basically an isolated DOM tree
living inside an another (hosting) DOM tree. The spec refers the hosting
tree as \emph{light DOM tree} and the attached DOM as \emph{shadow DOM
tree}. Conceptually, the \emph{shadow DOM} issues a single important
topic in software development: \textbf{Encapsulation}. While the first
spec \emph{custom elements} provides a sufficient way to encapsulate JS
behavior, \emph{shadow DOM} coined strongly to in the direction of style
encapsulation.

With an ever increasing complexity of an single-page application, the
global nature of the DOM creates a daunting situation for code
organization and leads over times to highly fragmented bits of CSS and
obscure CSS selectors or html wrappers. Of course, this situation lowers
code clarity and reusability dramatically. The only solution which won't
break with the existing global paradigm effectively is to allow separate
pieces of encapsulated code sit on top of the global DOM - introducing
the shadowed DOM approach!

Enhancing the previous example the new encapsulated \texttt{HelloWorld}
would like this:

\begin{Shaded}
\begin{Highlighting}[]
\OperatorTok{>} \VariableTok{main}\NormalTok{.}\AttributeTok{js}
\KeywordTok{class} \NormalTok{HelloWorld }\KeywordTok{extends} \NormalTok{HTMLElement }\OperatorTok{\{}
 \AttributeTok{constructor}\NormalTok{() }\OperatorTok{\{}
  \NormalTok{...}
  \KeywordTok{this}\NormalTok{.}\AttributeTok{attachShadow}\NormalTok{(}\OperatorTok{\{}\DataTypeTok{mode}\OperatorTok{:} \StringTok{'open'}\OperatorTok{\}}\NormalTok{)}\OperatorTok{;}
  \VariableTok{shadowRoot}\NormalTok{.}\AttributeTok{innerHTML} \OperatorTok{=} \StringTok{'<p>hello</p>'}\OperatorTok{;}
 \OperatorTok{\}}
\OperatorTok{\}}
\end{Highlighting}
\end{Shaded}

The new global method \texttt{attachShadow} adds a new document root to
the \texttt{HelloWorld} which has the same properties as a normal DOM.
Therefore, invoking \texttt{innerHTML} method would fill the new
document (fragment) with some arbitrary content. Note that
\texttt{shadowRoot} is marked as \textbf{open} which ensures that some
events can bubble out and outside JS can reach in the new root. Nested
children nodes and other content in the light DOM are ``shadowed'' by
the new root and must be invited in by so called \texttt{slots}.

\subsubsection{Slots}\label{slots}

Contradicting to the simplified \texttt{HelloWorld} example, a
\emph{shadow DOM} shouldn't contain any \sout{valuable} content. While
technical possible any change of an element would require deeply nested
calls from the \emph{light DOM} to the \emph{shadow DOM} to update the
element in place. That's why \emph{shadow DOM} should be perceived more
as \textbf{static HTML template} and provide therefore a kind of
internal frame for the render engine. \texttt{Slots} are placeholders
for \emph{light DOM} nodes used to mark the endpoints in question.

Technically, the \emph{light DOM} nodes are not moved inside the
\emph{shadow DOM}. Their just rendered in place. It's an subtle but
important difference towards handling a node. JS behaviour and CSS
styles applied in the \emph{light DOM} will still be valid in the
\emph{shadow DOM}. The render engine literally taking the nodes and
putting them inside the \texttt{slot}. This procedure is commonly
referred as \textbf{flattening} of the DOM trees.

\paragraph{Named slots}\label{named-slots}

A named slot is the preferable way for clear code organization. Taking
for example
\texttt{\textless{}slot\ name="hello"\textgreater{}Drop\ me\ a\ "hello"\ node\textless{}/slot\textgreater{}}
targets all direct \emph{light DOM} child nodes of the hosting node
matching the slot name like
\texttt{\textless{}div\ slot="hello"\textgreater{}\textless{}/div\textgreater{}}.
Writing a little documentation inside the
\texttt{\textless{}slot\textgreater{}} tag is considered as a good
practice as it will be rendered only if no matching \emph{light DOM}
node is available. This functionality makes a \emph{custom element}
pretty much self-explanatory.

\paragraph{Unnamed slots}\label{unnamed-slots}

Inside a so-called \emph{default slot} which looks like
\texttt{\textless{}slot\textgreater{}Unnamed\ content\ goes\ here\textless{}/slot\textgreater{}},
the render engine expands all direct \emph{light DOM} children without a
\texttt{slot} attribute. In case of multiple default slots, the first
slot takes it all.

\subsubsection{Styling}\label{styling}

As mentioned in the last section, there is a distinct difference about
the nature of nodes. Nodes declared and rendered exclusively in the
\emph{shadow DOM} are not affected by any styling from outside. Nodes
which are declared outside and distributed via \texttt{slots} will be
styled in the \emph{light DOM} and can be additionally painted in the
\emph{shadow DOM} through the new CSS-Selector \texttt{::slotted()}.

Note that styles from the outside have an higher specify than styles
assigned after distribution. Therefore it is generally a good advice to
minimize the global stylings to some base styling for uniformity of the
web site while leaving the specific stylings to the component. Due to
the cascading nature of CSS, styles will still ``bleed in'' from
ancestors to the \emph{light DOM} nodes. Therefore it's strongly
recommended to begin every \emph{shadow DOM} with a \textbf{\emph{CSS
reset:} \texttt{:host\ \{all:\ initial;\}}}.

Regarding the importance style encapsulation, a couple of new CSS rules
emerged that are exclusively targeting the \emph{shadow DOM}. The table
below outlines styling possibilities for the use INSIDE the \emph{shadow
DOM}:

\begin{itemize}
\tightlist
\item
  ::slotted(selector)\\
  Applies to distributed nodes and repaints them after distribution.
  \texttt{Slotted} won't override outsides styles but can complement
  them with unset style rules.
\item
  :host\\
  The host property will add styles or change inherited ones inside
  shadow DOM. Using \texttt{all:\ initial;} will ensure browser defaults
  only.
\item
  :host(condition)\\
  Like the previous one this node will style the shadow DOM but this
  time based on attributes/conditions assigned to the hosting node.
\item
  :host-context(condition)\\
  Like the previous one this node will style the shadow DOM but will
  look after context set at the host node or even at the host ancestor.
\end{itemize}

Using the \emph{functional selector} of \texttt{:host()} or even the
only-functional \texttt{:host-contest()} allows the creation of
\textbf{context-aware custom elements}. A possible use case would be
``theming'' a component (example taken from {[}14{]}):

\begin{Shaded}
\begin{Highlighting}[]
\NormalTok{> index.html}
\KeywordTok{<body}\OtherTok{ class=}\StringTok{"darktheme"}\KeywordTok{>}
  \KeywordTok{<fancy-tabs>}
    \NormalTok{...}
  \KeywordTok{</fancy-tabs>}
\KeywordTok{</body>}
\end{Highlighting}
\end{Shaded}

\begin{Shaded}
\begin{Highlighting}[]
\NormalTok{> fancy-tabs shadowRoot}
\NormalTok{<style>}
\DecValTok{:}\NormalTok{host-context(}\FloatTok{.darktheme}\NormalTok{) }\KeywordTok{\{}
\ErrorTok{ } \KeywordTok{color:} \DataTypeTok{white}\KeywordTok{;}
\ErrorTok{ } \KeywordTok{background:} \DataTypeTok{black}\KeywordTok{;}
\KeywordTok{\}}
\NormalTok{</style>}
\end{Highlighting}
\end{Shaded}

\subsubsection{JS Behavior}\label{js-behavior}

As mentioned earlier any logic applied to \emph{light DOM} nodes stays
with the node even after redistribution. For the sake of separation of
concerns the business logic should be part of the \emph{custom element}
(the \emph{light DOM}) and not the part of the \emph{shadow DOM}. On the
other hand there are numerous scenarios where JS is just concerned with
\textbf{styling or animation of an element}.In this case it might be
more straightforward to apply JS inside the \emph{shadow DOM} to avoid
mixing with business logic handlers.

Drilling down to a \emph{light DOM} node from an \emph{shadow DOM}
context is not possible with querying the node directly with
\texttt{.querySelector()} or \texttt{.getElementById()} as the node is
not part of the context. To get a distributed node in question it needs
the way over the slot node and call \texttt{slot.assinedNodes()} to
receive an array of distributed node(s) which can be accessed and
manipulated like any other node. Calling \texttt{.assignedNodes()} on an
empty \texttt{slot} returns an empty array.

Wrapping up this section, \emph{shadow DOM} provides a non-hacky way to
create uniform looking \emph{custom elements} and even enhance styling
possibilities without adding much overhead. Still, for smaller
components with only one or two child nodes, just a little styling
and/or no structured redistribution a \emph{shadow DOM} might be to hard
to reason about. Eventually it all depends on the question of ``how hard
is it to implement it without shadow DOM'' - which can't be answered
universally. For a more in-depth guide, Google Engineer Eric Bidelman
wroten a great primer on \emph{shadow DOM}{[}14{]}.

So far, there is still a missing link between \emph{light DOM} and
\emph{shadow DOM}. The observant reader may have already noticed the
weak point in the \texttt{HelloWorld} example: how to ``vitalize'' the
\emph{shadow DOM}. Recapturing the last \texttt{HelloWorld} example a
string of markup was assigned to the \texttt{shadowRoot.innerHTML}
property. While it works perfectly fine in this simple case, a string of
markup is rather cumbersome and error-prone and doesn't scale well. When
putting quotes inside another quotes things break quickly. It makes the
life hard for developers to work with it because it requires manual
indentation and is out of syntax highlighting. That's the time templates
come into play.

\subsection{\texorpdfstring{HTML Templates
\href{https://www.w3.org/TR/html5/scripting-1.html\#the-template-element}{(w3c)}}{HTML Templates (w3c)}}\label{html-templates-w3c}

Among all other new standards \emph{HTML templates} are the most mature
and adopted standard in the browser environment. All major browsers,
except from Internet Explorer, support this standard.

One core concept in templates is efficiency: Whatever dropped inside a
\texttt{template} tag \sout{bucket} will be parsed on runtime - but not
constructed into the \emph{content tree}. It remains plain HTML Markup
sitting somewhere in the document until the time of activation.

Activation typically takes four steps:

\begin{enumerate}
\def\labelenumi{\arabic{enumi}.}
\tightlist
\item
  \textbf{Querying the template node in question}\\
  const node = document.querySelector(`template');
\item
  \textbf{Parsing the content and preparing the templates' content}\\
  const content = node.content;\\
  -\textgreater{} Returns a \emph{DocumentFragment} object. Handling is
  straight forward content.querySelector(`img').src = `logo.png';
\item
  \textbf{Optional: Cloning the \emph{DocumentFragment} for multiple
  use}\\
  const clone = content.cloneNode(``deep'');
\item
  \textbf{Appending the clone/original to destination}\\
  document.body.appendChild(clone);
\end{enumerate}

As easy and minimal \emph{HTML templates} are, they're missing out a
crucial feature other template implementations usually have. As
templates are basically just dump containers for HTML Markup, there is
no way to define some logic as \textbf{placeholders} where content
should appear. Of course, with heavy use of JS things could be modeled
this way. The idiomatic way tends more towards a \emph{Shadow DOM \&
HTML templates} symbiosis.

\begin{Shaded}
\begin{Highlighting}[]
\NormalTok{> index.html}
\KeywordTok{<hello-world>}
 \KeywordTok{<p}\OtherTok{ id=}\StringTok{"sendto"}\OtherTok{ slot=}\StringTok{"placeholder"}\KeywordTok{>}
  \NormalTok{Hello World Web Component  }
 \KeywordTok{</p>}
\KeywordTok{</hello-world>}
\CommentTok{<!-- COMPONENT STARTS HERE -->}
\KeywordTok{<template}\OtherTok{ id=}\StringTok{"hello"}\KeywordTok{>}
 \CommentTok{<!-- STYLES -->}
 \KeywordTok{<style>}
  \FloatTok{#stylewrapper} \KeywordTok{\{}
   \KeywordTok{font-weight:} \DataTypeTok{bold}\KeywordTok{;}
   \KeywordTok{color:} \NormalTok{orange}\KeywordTok{;}
  \KeywordTok{\}}
 \KeywordTok{</style>}
 \CommentTok{<!-- CONTENT -->}
 \KeywordTok{<div}\OtherTok{ id=}\StringTok{"stylewrapper"}\KeywordTok{>}
  \KeywordTok{<slot}\OtherTok{ name=}\StringTok{"placeholder"}\KeywordTok{>}
   \NormalTok{Named placeholder}
  \KeywordTok{</slot>}
 \KeywordTok{</div>}
\KeywordTok{</template>}

\KeywordTok{<script>}
 \CommentTok{// Switched to anonymous class notation}
 \CommentTok{// for keeping associated code together.}
 \VariableTok{customElements}\NormalTok{.}\AttributeTok{define}\NormalTok{(}\StringTok{'hello-world'}\OperatorTok{,}
  \KeywordTok{class} \KeywordTok{extends} \NormalTok{HTMLElement }\OperatorTok{\{}
   \AttributeTok{constructor}\NormalTok{() }\OperatorTok{\{}
      \KeywordTok{super}\NormalTok{()}\OperatorTok{;}
    \KeywordTok{this}\NormalTok{.}\AttributeTok{attachShadow}\NormalTok{(}\OperatorTok{\{}\DataTypeTok{mode}\OperatorTok{:} \StringTok{'open'}\OperatorTok{\}}\NormalTok{)}\OperatorTok{;}
    \KeywordTok{const} \NormalTok{template }\OperatorTok{=}
     \VariableTok{document}\NormalTok{.}\AttributeTok{querySelector}\NormalTok{(}\StringTok{'#hello'}\NormalTok{)}\OperatorTok{;}
    \KeywordTok{this}\NormalTok{.}\AttributeTok{shadowRoot}
     \NormalTok{.}\AttributeTok{appendChild}\NormalTok{(}\VariableTok{template}\NormalTok{.}\AttributeTok{content}\NormalTok{)}\OperatorTok{;}
   \OperatorTok{\}}
  \OperatorTok{\}}\NormalTok{)}\OperatorTok{;}
\OperatorTok{<}\SpecialStringTok{/script>}
\end{Highlighting}
\end{Shaded}

The updated \texttt{HelloWorld} component looks already pretty mature.
It combines all the previous mentioned standards into one blob of HTML.
\emph{Custom Elements} serves the logic, \emph{Shadow DOM} scopes the
styles and \emph{HTML Templates} efficiently glues DOM and \emph{Shadow
DOM} together. This separation of concerns comes with a huge gain in
flexibility. In a real world scenario \texttt{HelloWorld} would
contain/reference multiple \emph{HTML Templates} and could switch them
around without any fuss. A developer might to split up templates into
\textbf{STYLE} templates and \textbf{CONTENT} templates to increase
reusability even further.

The last standard in the row of four is not concerned with the internals
of a \emph{web component}. \emph{HTML Imports} serves the need for an
efficient distribution mechanism of components.

\subsection{\texorpdfstring{HTML Imports
\href{https://www.w3.org/TR/html-imports/}{(w3c)}}{HTML Imports (w3c)}}\label{html-imports-w3c}

Importing the \texttt{HelloWorld} component is a one-liner:

\begin{Shaded}
\begin{Highlighting}[]
\KeywordTok{<link}\OtherTok{ rel=}\StringTok{"import"}\OtherTok{ href=}\StringTok{"Hello.html"}\OtherTok{ async}\KeywordTok{>}
\end{Highlighting}
\end{Shaded}

The \texttt{async} flag is optional but like in any other fetching
event, strongly recommended. Once the imported HTML document comes into
scope, activation follows a very similar pattern like the previously
mentioned \emph{HTML templates}:

\begin{enumerate}
\def\labelenumi{\arabic{enumi}.}
\item
  \textbf{Querying the link node}
\item
  \textbf{Parsing the content and preparing the render}\\
  const content = linknode.import; -\textgreater{} Unlike the \emph{HTML
  template} the content a fully equipped document object.
\item
  \textbf{Optional: Cloning some nodes for multiple use}
\item
  \textbf{Appending the clone/original to destination}
\end{enumerate}

This again is the imperative way to handle a generic \emph{HTML Import}.
In the declarative world of \emph{web components} a component is
activated, parsed and anchored solely by its' tag name
\texttt{\textless{}hello-world\textgreater{}\textless{}/hello-world\textgreater{}}.
Preliminary, the component needs proper configuration as the last
\texttt{HelloWorld} example wouldn't work like it is currently. The next
section will elaborate the right configuration and composition of a
component to work out-of-the-box.

Despite from being just a practical document importer \emph{HTML
imports} acts like a fully fledged dependency manager for the browser.
Multiple resources, ranging from stylesheets, scripts, documents, media
files and even other \texttt{imports} can be grouped together in a
logical \texttt{import} statement. Internally, the browser engine keeps
track for every imported resource so it won't be loaded twice. The
inherent complexity is in fact a stumbling block for wider browser
adoption. Currently only Googles blink web engine supports \emph{HTML
Imports} as they are the driving force behind the \emph{web components}
spec in general. Mozilla and Apple imposed distaste for \emph{HTML
Imports} as a whole. One reason for this can be found in the
incompatibility of the spec with the upcoming \emph{ES6 module
loader}.\footnote{https://hacks.mozilla.org/2014/12/mozilla-and-web-components/}

Despite the discrepancies among browser vendors \emph{HTML Imports}
should still be part of the paper and future \emph{web components} as no
other native browser technology can bundle up CSS, JS and HTML that
efficient. As of today, January 2017, only Googles Chrome and related
Opera browser supporting the full \emph{web components} spec and,
despite from \emph{HTML Imports}, all other browser vendors most likely
will catch up with \emph{Custom Elements} and \emph{Shadow DOM} within
this year. In the meantime, the full \emph{web components} stack can be
\textbf{polyfilled} and used across all browsers.

\subsection{\texorpdfstring{Appendix A: Custom Events
\href{https://dom.spec.whatwg.org/\#interface-customevent}{(whatwg)}}{Appendix A: Custom Events (whatwg)}}\label{appendix-a-custom-events-whatwg}

Events are first-class citizens in the browser world and \emph{Custom
Events} are no exception. The \emph{Custom Elements} interface is part
of the DOM since years but with the rise of \emph{Custom Elements} they
will most likely become an indispensable building block of \emph{web
components}.

\begin{Shaded}
\begin{Highlighting}[]
\NormalTok{> index.html}
\KeywordTok{<hello-world>}
 \KeywordTok{<button>}\NormalTok{Launch CustomEvent}\KeywordTok{</button>}
\KeywordTok{</hello-world>}
\CommentTok{<!-- COMPONENT STARTS HERE -->}
\KeywordTok{<script>}
 \VariableTok{customElements}\NormalTok{.}\AttributeTok{define}\NormalTok{(}\StringTok{'hello-world'}\OperatorTok{,} 
 \KeywordTok{class} \KeywordTok{extends} \NormalTok{HTMLElement }\OperatorTok{\{}
  \AttributeTok{constructor}\NormalTok{() }\OperatorTok{\{}
     \KeywordTok{super}\NormalTok{()}\OperatorTok{;}
   \CommentTok{// Craft a CustomEvent e}
     \KeywordTok{const} \NormalTok{e }\OperatorTok{=} \KeywordTok{new} \AttributeTok{CustomEvent}\NormalTok{(}\StringTok{'hello-world'}\OperatorTok{,} \OperatorTok{\{}
    \DataTypeTok{bubbles}\OperatorTok{:} \KeywordTok{true}\OperatorTok{,} \CommentTok{//important!}
    \DataTypeTok{detail}\OperatorTok{:} \StringTok{'Contains string or object'}
   \OperatorTok{\}}\NormalTok{)}\OperatorTok{;}
   \CommentTok{// Launch e on child button click}
   \KeywordTok{this}\NormalTok{.}\AttributeTok{addEventListener}\NormalTok{(}\StringTok{'click'}\OperatorTok{,} \NormalTok{click }\OperatorTok{=>} \OperatorTok{\{}
    \KeywordTok{this}\NormalTok{.}\AttributeTok{dispatchEvent}\NormalTok{(e)}
    \VariableTok{click}\NormalTok{.}\AttributeTok{stopPropagation}\NormalTok{()}\OperatorTok{;}
   \OperatorTok{\}}\NormalTok{)}\OperatorTok{;}
   \CommentTok{// Catch e. Typically done by some parent}
     \CommentTok{// node. Message in detail property}
   \KeywordTok{this}\NormalTok{.}\AttributeTok{addEventListener}\NormalTok{(}\StringTok{'hello-world'}\OperatorTok{,} \NormalTok{e }\OperatorTok{=>} \OperatorTok{\{}
    \VariableTok{console}\NormalTok{.}\AttributeTok{log}\NormalTok{(}\VariableTok{e}\NormalTok{.}\AttributeTok{detail}\NormalTok{)}
   \OperatorTok{\}}\NormalTok{)}\OperatorTok{;}
  \OperatorTok{\}}
 \OperatorTok{\}}\NormalTok{)}\OperatorTok{;}
\OperatorTok{<}\SpecialStringTok{/script>}
\end{Highlighting}
\end{Shaded}

Naming events after the emitting tag makes the API almost
self-explanatory. Fortunately the \texttt{detail} property can transmit
even objects witch allows the developer to craft almost all
functionality inside one event. The further sections will elaborate a
feasible architecture for building a scaling microservice architecture.

Chaining and aggregating events from child nodes should be practiced and
exercised quiet frequently. As mentioned earlier in the \emph{Custom
Elements} section, the pattern of \textbf{extending native elements}
should be somewhat dismissed as it may be implemented outside of Chrome.
Nevertheless, it is possible to create own kind of quasi native buttons
when chaining a \emph{CustomEvent} directly after the native click
event.

Another typical \emph{custom element} use case can be as an actor on
(native) child elements. In this case, the \emph{Custom Element} catches
events from children, buffers or rebuild them and eventuall fires an
event towards the document root.

Unfortunately events only work ``upstream'' towards parent nodes. Still
the web platform offers plenty of possibilities to talk back to child
nodes.

\subsection{\texorpdfstring{Appendix B: Web Worker
\href{https://html.spec.whatwg.org/multipage/workers.html}{(whatwg)}}{Appendix B: Web Worker (whatwg)}}\label{appendix-b-web-worker-whatwg}

Like \emph{Custom Events}, \emph{Web Workers} had been around for a long
time and therefore embrace full support among major browsers. They
emerged at around 2009 when discussions about browser performance was
still in the early days. Nevertheless, the addressed problem of
\emph{Web Workers} is a fundamental language problem of JS itself.

JS runs in a single-threaded language environment. Every script in the
browser environment, from handling UI events to query and process larget
amounts of API data and manipulating the DOM, runs on the same
thread{[}15{]}. Putting a lot of work to the single main thread can slow
down the web service significantly. From time to time scripts can block
or fail to whatever reason which leads to a frozen UI on the users side.
A worker can overcome the bottleneck of the single-threaded nature with
spawning a new \textbf{background thread}. Most of the browsers work can
be leveraged to this new thread. Todays web architectures aims to
leverage an increasing amout of proccessing to the client to avoid
time-consuming roundtrips especially in mobiles networks.\footnote{Latency
  numbers: https://gist.github.com/jboner/2841832} While many native
APIs like \texttt{fetch} work seamlessly in the new thread, a worker has
no access to the DOM at all.

A \emph{Web Worker} spawns a new background thread where scripts can run
concurrent to the main thread. Usually a worker is loaded from a workers
dedicated file to embrace this kind of separation.

\begin{Shaded}
\begin{Highlighting}[]
\KeywordTok{const} \NormalTok{worker }\OperatorTok{=} \KeywordTok{new} \AttributeTok{Worker}\NormalTok{(}\StringTok{'worker.js'}\NormalTok{)}\OperatorTok{;}
\end{Highlighting}
\end{Shaded}

After initialization a worker communicates over a simple \textbf{message
based interface} with the main thread.

\begin{Shaded}
\begin{Highlighting}[]
\OperatorTok{>} \VariableTok{main}\NormalTok{.}\AttributeTok{js}
\CommentTok{// Send to worker}
\VariableTok{worker}\NormalTok{.}\AttributeTok{postMessage}\NormalTok{(}\StringTok{'Hello World'}\NormalTok{)}\OperatorTok{;}
\CommentTok{// Receive msg from worker}
\VariableTok{worker}\NormalTok{.}\AttributeTok{addEventListener}\NormalTok{(}\StringTok{'message'}\OperatorTok{,} \NormalTok{e }\OperatorTok{=>}
 \VariableTok{console}\NormalTok{.}\AttributeTok{log}\NormalTok{(}\StringTok{'Worker said: '}\OperatorTok{,} \VariableTok{e}\NormalTok{.}\AttributeTok{data}\NormalTok{))}\OperatorTok{;}
\end{Highlighting}
\end{Shaded}

\begin{Shaded}
\begin{Highlighting}[]
\OperatorTok{>} \VariableTok{worker}\NormalTok{.}\AttributeTok{js}
\CommentTok{// Receive msg and echo back}
\KeywordTok{this}\NormalTok{.}\AttributeTok{addEventListener}\NormalTok{(}\StringTok{'message'}\OperatorTok{,} \NormalTok{e }\OperatorTok{=>}
 \KeywordTok{this}\NormalTok{.}\AttributeTok{postMessage}\NormalTok{(}\StringTok{"Echo "} \OperatorTok{+} \VariableTok{e}\NormalTok{.}\AttributeTok{data}\NormalTok{))}\OperatorTok{;}
\end{Highlighting}
\end{Shaded}

Having no access to the DOM can be seen as hinderance,but i

Macroperspektive / Composition

\subsubsection{Assumptions about custom
elements}\label{assumptions-about-custom-elements}

Apart from the spec'd perspective there is mental model a webdeveloper
might

Creating and using webcomponents might require a new mental model how do
design a

containers

dichotome Pattern

Ein üblicher eventgesteuerter Webservice setzt sich aus
unterschiedlichsten Komponenten zusammen, die wiederum
unterschiedlichste Eventlistener \& -emitter in sich subsummieren. Diese
inhärente Komplexität verlangt geradezu nach einer klaren,
deterministischen Struktur des Webservices, die das Zusammenspiel
orchestriert. In der Analogie des Orchesters gesprochen, benötigt der
Webservice (oder sogar die gesamte Webapplikation) einen Dirigenten, der
für die Steuerung verantwortlich ist.

http://alistair.cockburn.us/Hexagonal+architecture

MVC Pattern

Pure frontend vs heavy backend

\section{Progressive Enhancement}\label{progressive-enhancement}

Chapter about progressive enhancement

\hypertarget{refs}{}
\hypertarget{ref-Filloux2016}{}
{[}1{]} F. Filloux, ``Bloated html, the best and the worse,'' 2016
{[}Online{]}. Available:
\url{https://mondaynote.com/bloated-html-the-best-and-the-worse-cac6eb06496d}

\hypertarget{ref-Baldwin2006}{}
{[}2{]} C. Y. Baldwin and K. B. Clark, ``Modularity in the Design of
Complex Engineering Systems,'' in \emph{Complex engineered systems:
Science meets technology}, D. Braha, A. A. Minai, and Y. Bar-Yam, Eds.
Berlin, Heidelberg: Springer Berlin Heidelberg, 2006, pp. 175--205
{[}Online{]}. Available:
\href{http://dx.doi.org/10.1007/3-540-32834-3\%7B/_\%7D9}{http://dx.doi.org/10.1007/3-540-32834-3\{\textbackslash{}\_\}9}

\hypertarget{ref-Newman2015}{}
{[}3{]} S. Newman, \emph{Building microservices}. O'Reilly Media, Inc.,
2016 {[}Online{]}. Available:
\url{http://www.ebook.de/de/product/22539693/sam_newmann_building_microservices.html}

\hypertarget{ref-Fowler2014}{}
{[}4{]} M. Fowler and J. Lewis, ``Microservices: A definition of this
new architectural term,'' Jan. 2014 {[}Online{]}. Available:
\url{http://www.martinfowler.com/articles/microservices.html}

\hypertarget{ref-Abramov2015}{}
{[}5{]} D. Abramov, ``Presentational and Container Components --
Medium.'' 2015 {[}Online{]}. Available:
\url{https://medium.com/@dan_abramov/smart-and-dumb-components-7ca2f9a7c7d0}.
{[}Accessed: 01-Dec-2016{]}

\hypertarget{ref-Rauschmayer2015}{}
{[}6{]} D. A. Rauschmayer, ``Tree-shaking with webpack 2 and babel 6.''
2015 {[}Online{]}. Available:
\url{http://www.2ality.com/2015/12/webpack-tree-shaking.html}

\hypertarget{ref-Martin}{}
{[}7{]} R. C. Martin, ``The single responsibility principle.''
{[}Online{]}. Available:
\url{http://programmer.97things.oreilly.com/wiki/index.php/The_Single_Responsibility_Principle}

\hypertarget{ref-Conway1968}{}
{[}8{]} M. E. Conway, ``How do committees invent?'' 1968 {[}Online{]}.
Available: \url{http://www.melconway.com/Home/Committees_Paper.html}

\hypertarget{ref-Issa2016}{}
{[}9{]} B. Issa, ``The way of the web.'' Polymer Summit 2016, Oct-2016
{[}Online{]}. Available:
\url{https://www.youtube.com/watch?v=8ZTFEhPBJEE}

\hypertarget{ref-Buchner2016}{}
{[}10{]} D. Buchner, ``Demythstifying web components,'' 2016
{[}Online{]}. Available:
\url{http://www.backalleycoder.com/2016/08/26/demythstifying-web-components/}

\hypertarget{ref-vanKesteren2014}{}
{[}11{]} A. van Kesteren, ``Mozilla and web components: Update,'' 2014
{[}Online{]}. Available:
\url{https://hacks.mozilla.org/2014/12/mozilla-and-web-components/}

\hypertarget{ref-Bidelman2016}{}
{[}12{]} E. Bidelman, ``Custom elements v1: reusable web components.''
2016 {[}Online{]}. Available:
\url{https://developers.google.com/web/fundamentals/primers/customelements/}.
{[}Accessed: 01-Dec-2016{]}

\hypertarget{ref-HTML}{}
{[}13{]} \emph{HTML living standard --- last updated 11 january 2017}.
{[}Online{]}. Available: \url{https://html.spec.whatwg.org/multipage/}

\hypertarget{ref-Bidelman2016shadow}{}
{[}14{]} E. Bidelman, ``Shadow dom v1: Self-contained web components.''
2016 {[}Online{]}. Available:
\url{https://developers.google.com/web/fundamentals/getting-started/primers/shadowdom}

\hypertarget{ref-Bidelman2010}{}
{[}15{]} E. Bidelman, ``The basics of web workers.'' 2010 {[}Online{]}.
Available: \url{https://www.html5rocks.com/en/tutorials/workers/basics/}

\end{document}
