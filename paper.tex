\section{Ausgangssituation}\label{ausgangssituation}

Softwareentwicklung für den Browser ist ohne Zweifel eine komplexe und
fehleranfällige Angelegenheit. Die Gründe für diesen Umstand sind so
vielfältig, dass darüber eine ganze Forschungsarbeit geschrieben werden
könnte. Je nach Blickwinkel könnte man von der technischen Seite
argumentieren und die Designschwächen der Browserplattform
erläutern.{[}1{]} Oder man könnte die organisatorische Seite betrachten
und die Schwerfälligkeit von Standardisierungsprozessen
anprangern.{[}2{]} Oder aber man nimmt die historische Entwicklung in
Augenschein. Die Zeiten statischer HTML/CSS Seiten auf Desktopgeräten
ist mithin noch nicht lange her.

Dennoch tragen alle diese Facetten dazu bei, dass die Entwicklung und
Wartung von komplexen Webapplikationen mit enormen Zeit- und Geldaufwand
verbunden ist. Offensichtlich wird das Problem, wenn man die schiere
Anzahl der Frameworks und Bibliotheken mit Javascript als Zielsprache
betrachtet, die vielen Entwicklern Schwierigkeiten bereitet.{[}3{]}

Um die inhärenten Designschwächen von Browsern aus der Welt zu schaffen,
wurde schon im Jahr 2013 das \emph{Extensible Web Manifesto}
proklamiert.\footnote{https://extensiblewebmanifesto.org/} Darin wird,
kurz gefasst, eine Öffnung der Webplattformen avisiert, um
Webprogrammierung teilweise von Browserstandards zu entkoppeln. Drei
Jahre später sind diese Ziele in W3C Standards konkretisiert worden und
teilweise schon umgesetzt worden.

\section{Zielsetzung}\label{zielsetzung}

Viele populäre Frameworks für den Browser als Zielplattform versprechen
dem Entwickler eine bessere Kontrolle über seinen Webservice. Die
Notwendigkeit rührt unter anderem daher, dass es bisher nicht möglich
war einzelne Browserelemente in ihrem Aussehen und Verhalten zu
isolieren. Selbst kleinste Änderungen können daher die Funktionalität
des Gesamtsystems beeinträchtigen. Mit der neuen W3C Spezifikation
\emph{Web Components},\footnote{https://www.w3.org/standards/techs/components}
die mehrere Substandards unter sich vereint, soll die Modularität nun
auch nativ im Browser unterstützt werden. Modularität als Designpattern
ist in der IT schon länger Standard um komplexe Systeme beherrschbar zu
gestalten und auch zukünftige Unsicherheiten abzuwägen.{[}4, S. 1{]} Die
Unsicherheit, die Frameworks immer anhaftet, ist die Lebensdauer
derselben.

Mit \emph{Web Components} ist der Entwickler in der Lage, eigene HTML
Elemente zu definieren und diese per Templates zu ex- \& importieren.
Darüber hinaus können sie JavaScript und CSS Funktionalität enthalten
und diese vom Rest der Webseite abzukapseln. Ziel dieser Arbeit ist die
systematische Erfassung und Risikoabwägung der neuen Technologien sowie
eine praktische Erläuterung möglicher Architekturmodelle. Bisher gibt es
keine einheitliche Bestpractice, wie denn die Komponenten umzusetzen
sind, obwohl diese Technologien bereits länger durch Polyfills genutzt
werden können. Manch ein Entwickler fürchtet schon eine Flut von
schlecht gebauten Komponenten, die wiederum neue Probleme schaffen
könnten.{[}5{]}

\section{Vorgehensweise}\label{vorgehensweise}

Den Anfang dieser Arbeit soll eine Analyse der aktuellen Situation der
Frontend Entwicklung aufzeigen. Dort soll auf die Problematik der
bisherigen Architekturmodelle und die Probleme mit monolithischen
Frameworks eingegangen werden. Außerdem soll dieser Teil vor Augen
führen, warum es bisher nur mit zusätzlicher Abstraktionsebenen möglich
war einen modularen Aufbau von Web Applikationen zu ermöglichen.

Im nächsten Abschnitt der Arbeit sollen die neuen Standards
aufgeschlüsselt, zugänglich gemacht und auf Anwendungsmöglichkeiten
untersucht werden. Explorativ sollen Bestpractices offengelegt werden
und mögliche Einbettungen ins Gesamtsystem diskutiert werden. Auch die
Probleme, wie sie beispielsweise bei alten Browsern auftreten können,
sollen hier behandelt werden.

Der letzte Abschnitt soll das Thema mit einer Metaebene abrunden, in der
auch die neuen CSS Standards \emph{Houdini} als Teil des Gesamtsystems
betrachtet werden sollen.\footnote{https://drafts.css-houdini.org/}

\hypertarget{refs}{}
\hypertarget{ref-Katz2013}{}
{[}1{]} Y. Katz, „Extend the Web Forward``, 2013 {[}Online{]}. Verfügbar
unter: \url{http://yehudakatz.com/2013/05/21/extend-the-web-forward/}

\hypertarget{ref-Walton2016}{}
{[}2{]} P. Walton, „Houdini: Maybe The Most Exciting Development In CSS
You've Never Heard Of``, 2016 {[}Online{]}. Verfügbar unter:
\url{https://www.smashingmagazine.com/2016/03/houdini-maybe-the-most-exciting-development-in-css-youve-never-heard-of}

\hypertarget{ref-Berner2016}{}
{[}3{]} D. Berner, „Not An Imposter: Fighting Front-End Fatigue``, 2016
{[}Online{]}. Verfügbar unter:
\url{https://www.smashingmagazine.com/2016/11/not-an-imposter-fighting-front-end-fatigue/}

\hypertarget{ref-Baldwin2006}{}
{[}4{]} C. Y. Baldwin und K. B. Clark, „Modularity in the Design of
Complex Engineering Systems``, in \emph{Complex Engineered Systems:
Science Meets Technology}, D. Braha, A. A. Minai, und Y. Bar-Yam, Hrsg.
Berlin, Heidelberg: Springer Berlin Heidelberg, 2006, S. 175--205
{[}Online{]}. Verfügbar unter:
\url{http://dx.doi.org/10.1007/3-540-32834-3_9}

\hypertarget{ref-Keith2014}{}
{[}5{]} J. Keith, „Web Components``, 2014 {[}Online{]}. Verfügbar unter:
\url{https://adactio.com/journal/7431}
