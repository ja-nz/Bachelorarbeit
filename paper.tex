\documentclass[]{article}
\usepackage{lmodern}
\usepackage{amssymb,amsmath}
\usepackage{ifxetex,ifluatex}
\usepackage{fixltx2e} % provides \textsubscript
\ifnum 0\ifxetex 1\fi\ifluatex 1\fi=0 % if pdftex
  \usepackage[T1]{fontenc}
  \usepackage[utf8]{inputenc}
\else % if luatex or xelatex
  \ifxetex
    \usepackage{mathspec}
  \else
    \usepackage{fontspec}
  \fi
  \defaultfontfeatures{Ligatures=TeX,Scale=MatchLowercase}
\fi
% use upquote if available, for straight quotes in verbatim environments
\IfFileExists{upquote.sty}{\usepackage{upquote}}{}
% use microtype if available
\IfFileExists{microtype.sty}{%
\usepackage{microtype}
\UseMicrotypeSet[protrusion]{basicmath} % disable protrusion for tt fonts
}{}
\usepackage[unicode=true]{hyperref}
\hypersetup{
            pdftitle={Browsernative Microservices},
            pdfauthor={Jan Peteler, FH Würzburg-Schweinfurt, jan.peteler@student.fhws.de},
            pdfborder={0 0 0},
            breaklinks=true}
\urlstyle{same}  % don't use monospace font for urls
\usepackage{color}
\usepackage{fancyvrb}
\newcommand{\VerbBar}{|}
\newcommand{\VERB}{\Verb[commandchars=\\\{\}]}
\DefineVerbatimEnvironment{Highlighting}{Verbatim}{commandchars=\\\{\}}
% Add ',fontsize=\small' for more characters per line
\newenvironment{Shaded}{}{}
\newcommand{\KeywordTok}[1]{\textcolor[rgb]{0.00,0.44,0.13}{\textbf{{#1}}}}
\newcommand{\DataTypeTok}[1]{\textcolor[rgb]{0.56,0.13,0.00}{{#1}}}
\newcommand{\DecValTok}[1]{\textcolor[rgb]{0.25,0.63,0.44}{{#1}}}
\newcommand{\BaseNTok}[1]{\textcolor[rgb]{0.25,0.63,0.44}{{#1}}}
\newcommand{\FloatTok}[1]{\textcolor[rgb]{0.25,0.63,0.44}{{#1}}}
\newcommand{\ConstantTok}[1]{\textcolor[rgb]{0.53,0.00,0.00}{{#1}}}
\newcommand{\CharTok}[1]{\textcolor[rgb]{0.25,0.44,0.63}{{#1}}}
\newcommand{\SpecialCharTok}[1]{\textcolor[rgb]{0.25,0.44,0.63}{{#1}}}
\newcommand{\StringTok}[1]{\textcolor[rgb]{0.25,0.44,0.63}{{#1}}}
\newcommand{\VerbatimStringTok}[1]{\textcolor[rgb]{0.25,0.44,0.63}{{#1}}}
\newcommand{\SpecialStringTok}[1]{\textcolor[rgb]{0.73,0.40,0.53}{{#1}}}
\newcommand{\ImportTok}[1]{{#1}}
\newcommand{\CommentTok}[1]{\textcolor[rgb]{0.38,0.63,0.69}{\textit{{#1}}}}
\newcommand{\DocumentationTok}[1]{\textcolor[rgb]{0.73,0.13,0.13}{\textit{{#1}}}}
\newcommand{\AnnotationTok}[1]{\textcolor[rgb]{0.38,0.63,0.69}{\textbf{\textit{{#1}}}}}
\newcommand{\CommentVarTok}[1]{\textcolor[rgb]{0.38,0.63,0.69}{\textbf{\textit{{#1}}}}}
\newcommand{\OtherTok}[1]{\textcolor[rgb]{0.00,0.44,0.13}{{#1}}}
\newcommand{\FunctionTok}[1]{\textcolor[rgb]{0.02,0.16,0.49}{{#1}}}
\newcommand{\VariableTok}[1]{\textcolor[rgb]{0.10,0.09,0.49}{{#1}}}
\newcommand{\ControlFlowTok}[1]{\textcolor[rgb]{0.00,0.44,0.13}{\textbf{{#1}}}}
\newcommand{\OperatorTok}[1]{\textcolor[rgb]{0.40,0.40,0.40}{{#1}}}
\newcommand{\BuiltInTok}[1]{{#1}}
\newcommand{\ExtensionTok}[1]{{#1}}
\newcommand{\PreprocessorTok}[1]{\textcolor[rgb]{0.74,0.48,0.00}{{#1}}}
\newcommand{\AttributeTok}[1]{\textcolor[rgb]{0.49,0.56,0.16}{{#1}}}
\newcommand{\RegionMarkerTok}[1]{{#1}}
\newcommand{\InformationTok}[1]{\textcolor[rgb]{0.38,0.63,0.69}{\textbf{\textit{{#1}}}}}
\newcommand{\WarningTok}[1]{\textcolor[rgb]{0.38,0.63,0.69}{\textbf{\textit{{#1}}}}}
\newcommand{\AlertTok}[1]{\textcolor[rgb]{1.00,0.00,0.00}{\textbf{{#1}}}}
\newcommand{\ErrorTok}[1]{\textcolor[rgb]{1.00,0.00,0.00}{\textbf{{#1}}}}
\newcommand{\NormalTok}[1]{{#1}}
\usepackage[normalem]{ulem}
% avoid problems with \sout in headers with hyperref:
\pdfstringdefDisableCommands{\renewcommand{\sout}{}}
\IfFileExists{parskip.sty}{%
\usepackage{parskip}
}{% else
\setlength{\parindent}{0pt}
\setlength{\parskip}{6pt plus 2pt minus 1pt}
}
\setlength{\emergencystretch}{3em}  % prevent overfull lines
\providecommand{\tightlist}{%
  \setlength{\itemsep}{0pt}\setlength{\parskip}{0pt}}
\setcounter{secnumdepth}{0}
% Redefines (sub)paragraphs to behave more like sections
\ifx\paragraph\undefined\else
\let\oldparagraph\paragraph
\renewcommand{\paragraph}[1]{\oldparagraph{#1}\mbox{}}
\fi
\ifx\subparagraph\undefined\else
\let\oldsubparagraph\subparagraph
\renewcommand{\subparagraph}[1]{\oldsubparagraph{#1}\mbox{}}
\fi

% set default figure placement to htbp
\makeatletter
\def\fps@figure{htbp}
\makeatother


\title{Browsernative Microservices}
\providecommand{\subtitle}[1]{}
\subtitle{Modular web architecture through new W3C specifications}
\author{Jan Peteler, FH Würzburg-Schweinfurt, jan.peteler@student.fhws.de}
\date{Januar 2017}

\begin{document}
\maketitle
\begin{abstract}
Building complex web applications nowadays require additional layers of
abstraction and often heavily depend on proprietary frameworks. New
specifications build right into the browserengine provide a native
service API to overcome tricky abstraction constraints.
\end{abstract}

{
\setcounter{tocdepth}{3}
\tableofcontents
}
\section{Simplicity and the web}\label{simplicity-and-the-web}

\begin{quote}
Simplicity is prerequisite for reliability. - Edsger W. Dijkstra
\end{quote}

Computers can scale, humans can't. Ever since a program or complex
system made by humans has been constrained by humans mental
capabilities. Like in the analogy of juggling balls, our brain can just
``juggle'' a few things at a time. Rich Hickey, the inventor of the
programming language Clojure gave an inspirational keynote on the topic
of \textbf{simplicity}.\footnote{\href{https://www.youtube.com/watch?v=rI8tNMsozo0\&t=46s}{Rails
  Conf 2012 Keynote: Simplicity Matters by Rich Hickey}} In every sphere
of a humans life, simplicity aligns perception with our mental
capacities.

Derived from the ancient Latin word \textbf{simplex}, simple can be
understood as ``literally, uncompounded or onefold''\footnote{\href{http://www.etymonline.com/index.php?term=simple}{Etymology
  Dictionary}} which points directly to the unidimensional aspect. While
complexity describes the multilayered und entangled nature of
conditions, simplicity empowers the human brain to reason about issues
in a straightforward manner. It certainly has some overlapping's with
easy, but while easy is more of a relative nature, simple can be laid
out as a objective manner and therefore universally applicable.

Software development is undoubtedly rich in complexity and full of
subtle pitfalls. In a typical scenario, a piece of software evolves over
time in one or another opinionated direction. Layers of new abstractions
wrestling with old legacy abstractions and mutation becomes untraceable.
Subtle bugs start to creep in. Eventually the small piece of software
may end up in a highly complected monolith which will determine future
design decisions to a painful degree. Future strategies of the
company/organization will be highly determined by the current state in
the need of ``keeping the lights on''.

On the other side of this dystopian scenario, the truly modular system
architecture abandons many of those inconsistencies. The whole system is
divided in pluggable parts, object mutation is either traceable or
avoided altogether in favor of immutable data structures. As Rich Hickey
argues, design decisions should be made under the \textbf{impression of
extending, substitution, moving, combining and repurposing}. The ability
to reason about the program at any given time is crucial for future
decisions and implementations. Recalling again the unidimensional nature
of simplicity.

Simplicity in ``the web'', read as a loose generalization of
``everything that runs in the browser'', is certainly a story full of
misconceptions and will be explained further along. While simplicity in
the backend is mostly a matter of principles and patterns, any
browser-based frontend is restricted on the highly deterministic nature
of the browser platform.

In the last four years the average transfer size of a webpage doubled to
currently around 2.5 MB.\footnote{\href{http://httparchive.org/trends.php}{HTTPArchive
  Trends}} Subtracting images, fonts or other content the size of HTML,
CSS and JS sums up to a total average of 550 kb. One character weights
around 1 byte which means an average webpage is delivering 550.000
character or around 125 pages of single-spaced text. Frederic Filloux
analyzed the payload on different newspaper websites and came to the
conclusion, that only round about 5-6 \% of the transferred characters
made for human consumption.{[}1{]}

Having an 95 \% overhead is rather undesirable for both the consumers
and creators of the website. Since it's a widespread problem without a
single point of failure one can argue the platform itself is the
failure. By design, every pageload results in a monolithic DOM tree
managed by the browser engine. Whether rendering just a bunch of static
text nodes or an ever changing webapp the underlying global nature of
the DOM tree remains the same. Every additional piece of code added to
the webpage will invisibly add another fold of complexity to this global
object.

In an non-deterministic runtime environment, encapsulation and
modularization is a typical pattern to make complexity manageable and
accommodate future uncertainty.{[}2, p. 1{]} Since years the average JS
payload is steadily rising which can be interpreted as a trend towards
more dynamic websites. The demands to the browser platform changed from
a static page renderer to a \textbf{dynamic UI machine} without changing
the underlying architecture significantly. Under the current situation
only additional layers of abstraction can wrestle complexity.

In the recent years many \textbf{frameworks}, libraries and
methodologies approached the global nature of the DOM by scoping assets
and design rules into maintainable components. While the DOM can't be
scoped, JS can. Many frameworks, like ReactJS, AngularJS or VueJS just
to name a few, ditched the old rule of separated HTML, CSS and JS in
favor of an additional layer of abstracted JS components (containing
content, markup and styling). Quiet often those frameworks mimic a MVC
pattern on top of the browser engine which is a reasonable simple design
pattern to build graphical user interfaces. While frameworks are a valid
approach to build scalable web applications they remain highly
opinionated, embody inherent complexity themselves and can change and
break over time. Another downside is code inflation which is a crucial
point for performance and third-party libraries are no exception on
that. A standardized way for creating complex UI interfaces painlessly
requires new build-in browser capabilities.

In the year 2013 thinkers, creators and browser vendors joined together
to propose \emph{The Extensible Web Manifesto}.\footnote{\href{https://extensiblewebmanifesto.org/}{The
  Extensible Web Manifesto}} The claim of the manifesto was to enhance
the current web platforms with new low-level capabilities. Those
features should empower creators of the web to write more declarative
code and therefore overcome known bottlenecks and artificial
abstractions. Four years later, the enhancement of JavaScript
leapfrogged and many new low-level APIs brought to life. With this new
APIs at hand a vivid web developer can create robust websites with less
code and less additional libraries. This paper is an approach to unfold
these \textbf{browsernative} technologies to create overall simple and
resilient \textbf{microservices} for the browser

\emph{Disclaimer:} This paper introduces many new browser build-ins with
the focus on try and test. As the time of writing, many examples can be
tried frictionless in the console of the latest versions of
\textbf{Google Chrome, Opera and Apple Safari}.\footnote{\href{http://jonrimmer.github.io/are-we-componentized-yet/}{Are
  we componentized yet?}} On Mozilla Firefox technologies work behind a
flag and Microsoft Edge implementation is unfortunately far behind. But
Browser implementation changes quickly and soon technology adoption
won't be an issue. Meanwhile new standards can be used through
\textbf{polyfills} even on legacy browsers.

\section{Microservices}\label{microservices}

In search of a better, simpler web architecture we might look on already
established pattern that proofed to fulfill enterprise needs.
Microservices are a good approach for tearing big monolithic systems
into fine-grained simple services with explicit defined boundaries. In a
nutshell a microservice is a small, autonomous service that works
together with other services seamlessly.{[}3, p. 2{]} Or with the words
of Fowler and Lewis: ``\ldots{} the microservice architectural style is
an approach to developing a single application as a suite of small
services, each running in its own process and communicating with
lightweight mechanisms, often an HTTP resource API.''{[}4{]} Just yet at
this point a reader might spot some similarities with microservices and
the browser-based development: Both wrestling the problem of monolithic
architecture and both using lightweight communication mechanisms. In
fact, many big companies of ``the web'' like Amazon or Netflix
successfully transformed their monolithic system into a service based
system which gives a glimpse of the power behind microservices.{[}4{]}

Microservices incorporate many ideas from developing scalable software,
like \emph{domain-driven design} where it pursues the incorporation of
real world structure in the code.{[}3, p. 2{]} Or making use of
\emph{continuous delivery} for pushing software rapidly through
\emph{automated deployment} mechanisms into production.{[}4{]}
Furthermore, microservices transcendent the technical perspective and
reaches into the team organization. As a primary source of truth this
paper relies on the work of Sam Newman, who has written a comprehensive
guide called \emph{Building Microservices} and the work of Fowler and
Lewis. The purpose of this section is to gain confidence about the
microservices architecture in the context of the browser platform.

\subsection{Componentization via
Services}\label{componentization-via-services}

``A~\textbf{component}~is a unit of software that is independently
replaceable and upgradeable.'' {[}4{]} Components are the building
blocks of microservices. And microservices are the building blocks of
applications. Essentially the difference between microservices and
components is just the level of abstraction. Whether a concrete
microservice or a much more generic component, both following pretty
much the same principles. Therefore this paper referring to both parts
when talking about \textbf{services}.

The first principle of services is the \textbf{loose coupling
principle}: changing and deploying one service shouldn't result in
changing other parts of the system.{[}3, p. 30{]}. Mutations or shadowed
variables, which is happening a lot in CSS, making it hard to keep
changes ought to only affect one place in the application. A
\emph{browsernative microservice} therefore pushing encapsulation and
avoiding variable mutations outside its scope as much as possible.
Practically, CSS will be scoped and JS fosters immutable JS entities and
avoids variables leaking into the global namespace preferring the
\texttt{let} or \texttt{const} declaration in favor of the old
\texttt{var} declaration.

The second principle of services is the \textbf{high cohesion
principle}: Whether designing a microservice or it's components we want
related behavior sit together, and unrelated behavior to sit
elsewhere.{[}3, p. 30{]} High cohesion can be expressed in a dynamic way
as the \emph{Single Responsibility Principle}: ``Gather together those
things that change for the same reason and separate those things that
change for different reasons.''{[}5{]} In a very quick and dirty code
quality analysis, the quality can be measured just by counting the
places changes in the code occur in order to implement a functionality.
In an arbitrary MVC system when a \texttt{button} is dropped into the
VIEW part, the CONTROLLER needs some adjustment and maybe the MODEL,
too. Three places for adjustment is a reasonable easy task for the
brain. In the field of browser based development those principle is
often violated in the separated entities HTML, JS and CSS. A
\emph{browsernative microservice} aims for combining all of the
resources cohesively in a single place and file.

A \emph{component for the web}, or web component, is
\textbf{self-contained} which means it embodys all needed functionality
to get it's job done. Therefore it has a much better evolution mechanism
in the service contracts. Changing functionality won't break other
components. A component can progressively enhanced which guarantees
functionality throughout different versions.

Using libraries in web development is a common sense. But compared to
libraries, a component service offers multiple advantages. A library is
only loosely coupled to the implementation and therefore hard to track
in functionality. Changing a library may result in an unforeseen amount
of time fixing implementations. It is not unusual to see websites
embodying different versions of the same library (like with JQuery).
Another issue with libraries is dead code elimination which means the
process of removing code that is never going to be executed. New build
tools for the web, like Webpack 2 or Rollup offer this feature which
relies heavily on the static structure on ES6 modules.{[}6{]} Libraries
for the browsers are traditionally ``shipped'' as non static
\textbf{immediately-invoked functions}. Bootstrapping those is much
harder to archive as dead code may be part of the function itself. A
\emph{webnative microservices} should therefore embrace the high
cohesion principle and abandon most of the libraries in favor of static
ES6 module to lower the amount of dead code and to increase performance
especially at the first render.

Another advantage components have over libraries is the more explicit
interface.{[}4{]} While the functionality of a library needs
documentation, a component functionality is exposed via the components'
signature which comes in the fashion of an HTML element in
\emph{browsernative microservices}. For example, a button component can
consume attributes and becomes a primary buttons just by assignment

.

\begin{Shaded}
\begin{Highlighting}[]
\KeywordTok{<my-botton}\OtherTok{ isPrimary}\KeywordTok{>}\NormalTok{Primary}\KeywordTok{</my-botton>}
\end{Highlighting}
\end{Shaded}

\subsection{Organized around Business
Capabilities}\label{organized-around-business-capabilities}

\begin{quote}
``organizations which design systems \ldots{} are constrained to produce
designs which are copies of the communication~structures~of these
organizations''. {[}7{]}
\end{quote}

Emphasizing the human factor in microservices is a key feature.
Microservices are a product of real-world usage.{[}3, p. 1{]} Instead of
splitting team structures along the technology stack (UI Experts
-\textgreater{} Middleware -\textgreater{} Database) a microservice
approach model teams around \textbf{business capabilities}.{[}4{]}
Consequently every team is capable of planning, designing, implementing,
testing and maintaining their very own microservice. Along the
technology stack every member gains high competence about the
functionality of the service.

Real-world domains tend to be complex and multifaceted. To unfold their
complexity, domains can subdivided into \textbf{bounded contexts}.{[}3,
p. 31{]} For example, customer service is a business domain. Customers
again have different contexts depending on their demands. One context
can be sales, another context could be support. Every context makes
different assumptions about the underlying model. Each bounded context
draws an explicit interface where it decides what models to share with
other contexts.{[}3, p. 30{]} Each context can derived into multiple
microservices or, talking about \emph{Browsernative Microservices},
interfaces to the customers.

By assigning service responsibility to a team, the so called
\textbf{Definition of Done} (DoD) shifts from ``accomplishing projects''
to ``accomplishing products''. This new paradigm not only changes the
administrative overhead like budgeting or resource allocation. It
creates a kind of responsibility connection from the team to the
service. Expectedly, those teams are more motivated within their very
own service and exhibit a more sophisticated iteration time.{[}4{]}

For many companies working in the spheres of the internet the
client-side is highly important for their business. In fact, business
goals and capabilities can be derived from the front-end needs. The
state of the web is not only a story of numerous artifacts, it is also a
story of an highly fragmented market along devices, operating systems,
differing sizes and functionalities. Different devices again have
different assumptions about the technology stack. Splitting teams along
the stack results in an slow paced back and forth negotiation for every
change to be made. As browser technologies, design guidelines and
devices change in a fast paced manner it makes absolutely sense to shift
responsibility towards the teams altogether. A \emph{browsernative
microservice} embraces commitment over its whole lifecycle.

\subsection{Smart endpoints and dumb
pipes}\label{smart-endpoints-and-dumb-pipes}

Microservices for the browsers aim for side-effects like changing the UI
given to input parameters or emitting browser events. An isolated
microservice won't make too much sense after all. To ensure
communication between services, flexible yet powerful communication
channels must be established.

\emph{Smart endpoints and dumb pipes} is coined to the approach of
designing communication mostly decoupled and as cohesive as
possible.{[}4{]} In the analogy to the real world a message channel
should look like sending a letter: Two smart endpoints (sender and
receiver) and a mostly unified ``dumb'' letter format and channel. Each
endpoint owns it specific domain logic. In the browsernative context we
already emphasized the role of presentational components and containers.
The ``smartness'' of presentational components is altering the UI and
reacting on users input. The logic of a container follows the idea of an
filter in a Unix sense - receiving a request, applying logic as
appropriate and producing a response.{[}4{]}

Cited earlier in this paper microservices often rely on simple HTTP
request-response with resource API's and lightweight messaging as
communication protocols.{[}4{]} Those technologies highly accessible and
widely adopted and most probably usable for a browsernative service. In
the implementation part of this paper a reader will explore the
combination of RESTish protocols and a lightweight browser message bus
in action.

By focusing the service into clear defined business boundaries, it is
easier to define a smart API of the service which in our case will be a
set of attributes for the HTML element bound to the service.

\subsection{Decentralized Governance}\label{decentralized-governance}

Microservices are separate entities and decentralization is therefore a
meta concept of microservices. Distributed service responsibility owned
by the team is one aspect. Gluing independent service bricks to a whole
distributed system talking over the network protocols another one.

This distributed nature allows teams to create their own technology
stack, tools and services designed in the spirit of language- and
platform independence and share their knowledge with other
parties.{[}4{]} In the recent years many big companies like Facebook,
Google, Netflix and others followed that spirit and published their
ideas and implementations open source. The previously mentioned ReactJS
for example is a brainchild of Facebook. In fact, many tools and
techniques are byproduct of vital interaction of concrete domain
problems and their implementations.

The spirit of freedom can't be applied universally to
\emph{Browsernative Microservices} as the browser and its underlying DOM
will be the limitation factor to a certain degree. Talking about the
browser, a reader might be tempted to narrowly thinking of the obvious
VIEW layer only - which is not true anymore. In the recent years the
browser engines grows to a kind of virtual machine: there are connectors
to build-in databases, multithreading support, an ever-growing JS
build-ins and even speech synthesis. So-called \emph{Progressive Web
Apps\footnote{\href{https://developers.google.com/web/progressive-web-apps}{Progressive
  Web Apps}}}, a bunch of criteria for building good browserapps, can
achieve a similar look and feel like native apps. Services like
NativeScript\footnote{\href{https://www.nativescript.org}{NativeScript}},
effectively compiling ``the web'' to native code, lower the boundary
between native and browser code even further.

JS is the widely accepted language of the web. Nevertheless, a
microservice engineering team might choose another language for various
reasons. Transpiling languages to JS as target language isn't exotic
anymore. Languages like TypeScript, ClojureScript or PureScript compile
to JS even exclusively. Once web components hit a critical mass there
will be most likely some library support or foreign function interface
towards ES6 modules (which are mandatory for the new specifications).
With the rise of WebAssembly, a new low-level programming language for
the browser, the determination on JS will hypothetical deteriorate and
new quasi native languages for the web might gain traction.

Another more real life decentralization aspect derives from the easiness
of deployment in a safe, sandboxed environment. Web components virtually
ship no overhead or require obscure build tools. This makes them ideal
candidates for sharing and open-sourcing. Webcomponents.org\footnote{\href{https://webcomponents.org}{Webcomponents.org}}
is a registry for ready-to-use components of every scale and purpose
where for even Google shares a lot of their material design elements.

\subsection{Decentralized Data
Management}\label{decentralized-data-management}

Data Management in Microservice follows the same modular philosophy like
the service implementation. As mentioned earlier different bounded
contexts make different assumptions of the underlying models. A
\emph{browsernative microservices} takes this idea even further and
expands it to the fragmented world of electronic devices. Decentralized
decisions about conceptual models demand for decentralized data storage
decisions.{[}4{]} Todays web architectures aims to leverage an
increasing amout of processing to the client to avoid time-consuming
roundtrips especially in mobiles networks.\footnote{Latency numbers:
  https://gist.github.com/jboner/2841832} Since network roundtrips are
costly it is a good advice to only query as much data as needed and
cache as much as possible.

``Microservices prefer letting each service manage its own
database.''{[}Fowler2014{]} Ben Issa, chief architect of ING Australia
emphasizes this pragmatism on APIs in a conference talk. At ING the
frontend demands tailor the backend APIs, APIs may be produced
automatically and not even Issa knows how many APIs exists.{[}8{]} They
are using a pattern called \textbf{backend for frontends} empowering the
team working to craft their UI and backend in a one-to-one
relationship.{[}3, p. 72{]}

To see this pattern in the field a reader might have a look at Facebooks
GraphQL\footnote{\href{http://graphql.org/}{GraphQL}}. GraphQL is a
query language for the frontend. The backend solely replies on the
frontend needs. Another well documented example in the field is
Cognitects Datomic\footnote{\href{http://www.datomic.com/}{Datomic}},
where parts of the database will be reflected to the client. A so-called
Transactor ensures ACID compliance.

The simplified microservice example later in this paper assumes a
generic build-in API accompanied by build-in frontend components.
Instead of gluing frontend and backend together on runtime, the
microservice is designed holistically containing both front- and
backends. To reduce network calls especially for mobile devices it is a
good advice to cache data (f.e. in a global Service Worker). Revamping
offline capabilities even further data can be stored in a browser based
database like PouchDB. For the sake of simplicity data management won't
be explored into depth throughout this paper.

\subsection{Infrastructure Automation}\label{infrastructure-automation}

In the global nature of web development the development couldn't
completely decoupled from the production environment. This circumstance
left developers switching back and forth between files developing tricky
opinionated (and more often biased) ways to glue related parts together.
With scoped components, code blocks can be developed more sane and
conveyed into production without headache. Even more, in the scripted
environment of the browser this can be implemented rapidly and
continuous.

Previously mentioned Ben Issa, described the ING standard workflow.
Every component deserves a own \textbf{git repo} containing

\begin{itemize}
\tightlist
\item
  Internationalization conformity (i18n)
\item
  Accessibility conformity (a11y)
\item
  Tests for the component
\item
  Demos of the component
\item
  Blueprints to mock the one to one APIs
\item
  Docs
\end{itemize}

Every check-in is handled as release candidate and can be independently
deployed by a fully automated deployment machinery.{[}8{]} Even though
this example is an opinionated perception it gives a sense of a mature
component build for the web. Due to an exhaustive amount of testing and
deployment tools for the browser an automated infrastructure shouldn't
be a problem.

\subsection{Design for failure}\label{design-for-failure}

In theory a microservice is designed with a lot of emphasizes on
real-time monitoring for both the architectural elements and business
relevant metrics.{[}4{]} Due to the modular structure weak points can
occur in the orchestration of the services. Testing and automation is a
feasible task but failures may occur in another end users setting
undetected. Legacy browsers for example remain a problem for enhancing
websites with new technologies.

Regarding the evolution of the web, the ``next billion'' internet users
most likely using Android, have decent specs mobile phones, use an
evergreen browser but won't have a reliable internet connection.{[}9{]}
While \emph{Progressive Enhancement} was once coined on the principle to
build websites both for Browsers with JS and HTML only, the new
\emph{Progressive Enhancement} tends towards an ``\textbf{offline
first}'' build principle. A \emph{Browsernative Microservice} therefore
not only tries to cache data as much as possible, it should also bring
in a lot of program logic as described in the previous chapters.

Working in a JS heavy infrastructure demands for optimization to avoid
unexpected side-effects like the \emph{flash of unstyled content
(FOUC)}. Googles Polymer propagates the a general-purpose \textbf{PRLP
pattern}\footnote{\href{https://www.polymer-project.org/1.0/toolbox/server}{PRLP
  pattern}}:

\begin{itemize}
\tightlist
\item
  Push critical resources for the initial route
\item
  Render initial route
\item
  Pre-cache remaining routes
\item
  Lazy-load and create remaining routes on demand
\end{itemize}

Following this pattern a critical resource can detect browser
functionalities beforehand and switch to a \textbf{polyfill} instead of
the latest browser optimized version. After the initial paint, critical
resources like top-level microservices or other app logic can be loaded
and registered.

\subsection{Evolutionary Design}\label{evolutionary-design}

Microservices tend to become smaller over time. An evolutionary design
approach puts emphasizes on decomposition and scrapping the service.
``The key property of a component is the notion of independent
replacement and upgradeability.''{[}4{]} Therefore we can safely change
and chop services. Lazy parts of the system which won't change often
should be separated from parts undergoing a lot of churn.{[}4{]} Parts
that needs coupled changes could should be moved together or should be
even merged.

This flexible approach fits good in the world of browser based
development. In the last decade we have seen a lot of of changes in the
way we develop for the web. New approaches like the virtual DOM approach
found their way into mainstream web development followed by frameworks
and libraries.

\emph{Browsernative Microservices} should be perceived as complementary
technology. Contrary to Angular, React and other frameworks they have a
strong interop with existing systems and can be used with them together.

Andrew Rota, developer at Wayfair, came up with the idea to compose
small web components around a managing React system.{[}10{]} As web
components are native elements there is no difference to use a native
\texttt{button} or a native \texttt{custom-button}. With this pattern a
web developer can make use of the encapsulated advantages of web
components while still making use of the declarative event management
from various frameworks. Whatever new framework will be on the rise
within the next years, this approach allows rapid decomposition and
reassembling towards a new system.

\section{W3C specifications}\label{w3c-specifications}

For building a native microservice running on the ``bare-metal'' browser
engine requires a bunch of new specifications and assumptions. Most
importantly the quasi specification \textbf{Web Components} is needed.
\emph{Web Components} is not a real standard. It's an amalgam of APIs
from multiple w3c specs which can be used independently, too. A
webdeveloper may choose one spec and embrace the freedom in architecture
which can be combined with other frameworks/libraries.

Depending on the context, some people argue for only two specs which
essentially make it possible to create a scoped component but not caring
too much on it's distribution{[}11{]}. Some people prefer the three
specs {[}12{]}, but the majority advocating the four specs variant,
which is listed on \sout{the official}
\href{http://webcomponents.org}{webcomponents.org} website. For the
purpose of this article, all four specs will be discussed briefly to
provide a rough understanding. It is not meant to cover all bits and
pieces.

\subsection{\texorpdfstring{Custom Elements
\href{http://w3c.github.io/webcomponents/spec/custom/}{(w3c)}}{Custom Elements (w3c)}}\label{custom-elements-w3c}

Custom elements are the fundamental building blocks for web components
introducing the \emph{Single Responsibility Principle} to the browser.
In essence, they provide a way to create \textbf{custom HTML tags}
subsuming behavior, design and functionality. An obligatory
\textbf{HelloWorld} will give a flavor about the spec:

\begin{Shaded}
\begin{Highlighting}[]
\OperatorTok{>} \VariableTok{main}\NormalTok{.}\AttributeTok{js}
\KeywordTok{class} \NormalTok{HelloWorld }\KeywordTok{extends} \NormalTok{HTMLElement }\OperatorTok{\{}
 \AttributeTok{constructor}\NormalTok{() }\OperatorTok{\{}
  \KeywordTok{super}\NormalTok{()}\OperatorTok{;} \CommentTok{// mandatory!}
  \KeywordTok{this}\NormalTok{.}\AttributeTok{onclick} \OperatorTok{=} \NormalTok{e }\OperatorTok{=>} \AttributeTok{alert}\NormalTok{(}\StringTok{"hello"}\NormalTok{)}\OperatorTok{;}
 \OperatorTok{\}}
\OperatorTok{\}}
\VariableTok{customElements}\NormalTok{.}\AttributeTok{define}\NormalTok{(}\StringTok{'hello-world'}\OperatorTok{,} \NormalTok{HelloWorld)}
\end{Highlighting}
\end{Shaded}

\begin{Shaded}
\begin{Highlighting}[]
\NormalTok{> index.html}
\KeywordTok{<hello-world>}\NormalTok{say hello}\KeywordTok{</hello-world>}
\end{Highlighting}
\end{Shaded}

This example should be self-explanatory. Notably, custom elements come
in the fashion of \emph{ES6 Classes}\footnote{\href{https://developer.mozilla.org/en/docs/Web/JavaScript/Reference/Classes}{JavaScript
  Classes}} in favor of the normal JavaScript prototype-based
inheritance model. This class must inherit the base \texttt{HTMLElement}
interface which~``ensures the newly created element inherits the entire
DOM API and any properties/methods that you add to the class become part
of the element's DOM interface.''{[}13{]} Like any other \emph{ES6
class}, the new element can be specialized further on using the typical
\texttt{extends} inheritance.

The beauty of \emph{custom elements} comes with the keyword
\texttt{this} which points to the element itself. Instead of querying
and assigning behavior AFTER creation of the node, custom elements ship
their functionality PRIOR initialization of the element. The so called
\emph{fat-arrow} (\texttt{=\textgreater{}}) is just a new ES6 syntax
feature for an anonymous function declaration.

After declaration, the new HTML element needs to be registered in the
global build-in \texttt{customElements} object with an tag name like
\texttt{\textless{}hello-world\textgreater{}} acting as key to the
element. Mind the dash inside the tag name to conform the spec. Finally,
the new element can go live inside the HTML Document
\texttt{index.html}.

\subsubsection{Lifecycle methods}\label{lifecycle-methods}

In addition to the \texttt{constructor()}, the spec defines so called
\emph{lifecycle callbacks} for controlling the \textbf{behaviour in the
DOM}. Many popular frameworks like ReactJS or AngularJS rely on similar
approaches:

\begin{itemize}
\tightlist
\item
  \texttt{connectedCallback()}\\
  Called upon the time of \emph{connecting or upgrading the node} which
  means the moment the node is rendered inside the DOM. Typically this
  method is called straight after the \texttt{constructor()} on insert.
  Typically, this method contains setup code, such as fetching resources
  or rendering elements according to attributes.{[}13{]} For a fast
  initial render of the page, it is highly preferable to put many
  proceedings in favor of the constructor.
\item
  \texttt{disconnectedCallback()}\\
  Called upon the time of \emph{node removal}. Cleanup code like
  removing eventListeners or disconnecting websockets can be put here.
\item
  \texttt{attributeChangedCallback(attrName,\ oldVal,\ newVal)}\\
  This method provides an \emph{Onchange handler} that runs for certain
  attributes called with three values as defined in the signature. It is
  meant to control an elements' transition from on \texttt{oldVal} to a
  \texttt{newVal}. Due to performance issues, this callback is only
  triggered for attributes registered in an \emph{observedAttributes}
  array.
\item
  \texttt{adoptedCallback()}\\
  Called when moving the node \emph{between documents}.
\end{itemize}

\subsubsection{Custom attributes}\label{custom-attributes}

As previously mentioned, the custom elements must \texttt{extend} the
\texttt{HTMLElement} interface. Therefore, the new element inherits base
properties and methods commonly used in all HTML elements like
\texttt{id,\ class,\ addEventListner} . Additionally, it is possible to
define custom attributes using the \emph{custom elements'}
\textbf{getter / setter interface} to steer the behavior of the element.

\begin{Shaded}
\begin{Highlighting}[]
\OperatorTok{>} \VariableTok{main}\NormalTok{.}\AttributeTok{js}
\KeywordTok{class} \NormalTok{HelloWorld }\KeywordTok{extends} \NormalTok{HTMLElement }\OperatorTok{\{}
 \AttributeTok{constructor}\NormalTok{() }\OperatorTok{\{}\NormalTok{...}\OperatorTok{\}}
 \NormalTok{set }\AttributeTok{sayhello}\NormalTok{(val) }\OperatorTok{\{}
  \KeywordTok{this}\NormalTok{.}\AttributeTok{_hello} \OperatorTok{=} \NormalTok{val}\OperatorTok{;}
  \VariableTok{console}\NormalTok{.}\AttributeTok{log}\NormalTok{(}\KeywordTok{this}\NormalTok{.}\AttributeTok{_hello}\NormalTok{)}\OperatorTok{;}
 \OperatorTok{\}}
 \NormalTok{get }\AttributeTok{sayhello}\NormalTok{() }\OperatorTok{\{}
  \ControlFlowTok{return} \KeywordTok{this}\NormalTok{.}\AttributeTok{_hello}\OperatorTok{;}
 \OperatorTok{\}}
\OperatorTok{\}}\NormalTok{)}\OperatorTok{;}
\VariableTok{customElements}\NormalTok{.}\AttributeTok{define}\NormalTok{(}\StringTok{'hello-world'}\OperatorTok{,} \NormalTok{HelloWorld)}\OperatorTok{;}
\CommentTok{// Instantiation}
\KeywordTok{var} \NormalTok{el }\OperatorTok{=} \KeywordTok{new} \AttributeTok{HelloWorld}\NormalTok{()}\OperatorTok{;}
\VariableTok{el}\NormalTok{.}\AttributeTok{sayhello} \OperatorTok{=} \StringTok{"earth"}\OperatorTok{;}
\VariableTok{el}\NormalTok{.}\AttributeTok{sayhello}\OperatorTok{;}\CommentTok{//"earth"}
\end{Highlighting}
\end{Shaded}

Native DOM properties, like \texttt{id} or \texttt{onclick}, reflect
their values between HTML and JS.{[}14, Para. 2.6.1{]} For example
declaring the HTML
\texttt{\textless{}hello-world\ id="hello"\textgreater{}} like this
equals to assign the ID in JS like \texttt{Node.id\ =\ "hello"}. This
behavior won't work out-of-the-box with methods or properties defined
setters. For example declaring
\texttt{\textless{}hello-world\ sayhello="mars"\textgreater{}\textless{}/hello-world\textgreater{}}
would't call the \texttt{sayhello} function in the previous setup.

A common workaround to bind HTML and JS behavior together is archived by
using the aforementioned lifecycle method
\texttt{attributeChangedCallback} to \textbf{bind changing HTML
attributes to JS properties} and to map JS attributes to HTML with
\texttt{this.setAttributes(...)} respectively. On insertion time HTML
attributes might trigger custom JS methods retrieving HTML attributes
using \texttt{this.getAttributes(...)} method.

Concluding this section, a reader might already discover the
\textbf{mental model} behind \emph{web compontents}. A custom element is
similar to a named function where attributes treated as input variables.
In the hierarchical nature of DOM, input can occur either top-down via
assignments and bottom-up via captured events. The same goes true when
talking about output. Even though it seems obvious, it might be helpful
to keep this point in mind.

\subsubsection{Customized build-in
elements}\label{customized-build-in-elements}

One aspect didn't mentioned yet is the possibility of creating
sub-classes of build-in elements by extending the native Interfaces like
the \texttt{HTMLButtonElement} interface. While this functionality is
perfectly spec'd it is strongly rejected by some browser
vendors.\footnote{https://github.com/w3c/webcomponents/issues/509} Most
likely the spec will change in future in one or other way on this issue
and therefore customized build-in elements left out of this paper
intentionally.

\subsection{\texorpdfstring{Shadow DOM
\href{http://w3c.github.io/webcomponents/spec/shadow/}{(w3c)}}{Shadow DOM (w3c)}}\label{shadow-dom-w3c}

A \emph{shadow DOM} is basically an isolated DOM tree living inside an
another (hosting) DOM tree. The spec refers the hosting tree as
\emph{light DOM tree} and the attached DOM as \emph{shadow DOM tree}.
Conceptually, the \emph{shadow DOM} issues a single important topic in
software development: \textbf{Encapsulation}. While the first spec
\emph{custom elements} provides a sufficient way to encapsulate JS
behavior, \emph{shadow DOM} coined strongly to in the direction of style
encapsulation.

With an ever increasing complexity of an single-page application, the
global nature of the DOM creates a daunting situation for code
organization and leads over times to highly fragmented bits of CSS and
obscure CSS selectors or html wrappers. Of course, this situation lowers
code clarity and reusability dramatically. The only solution which won't
break with the existing global paradigm effectively is to allow separate
pieces of encapsulated code sit on top of the global DOM - introducing
the shadowed DOM approach!

Enhancing the previous example the new encapsulated \texttt{HelloWorld}
would like this:

\begin{Shaded}
\begin{Highlighting}[]
\OperatorTok{>} \VariableTok{main}\NormalTok{.}\AttributeTok{js}
\KeywordTok{class} \NormalTok{HelloWorld }\KeywordTok{extends} \NormalTok{HTMLElement }\OperatorTok{\{}
 \AttributeTok{constructor}\NormalTok{() }\OperatorTok{\{}
  \NormalTok{...}
  \KeywordTok{this}\NormalTok{.}\AttributeTok{attachShadow}\NormalTok{(}\OperatorTok{\{}\DataTypeTok{mode}\OperatorTok{:} \StringTok{'open'}\OperatorTok{\}}\NormalTok{)}\OperatorTok{;}
  \VariableTok{shadowRoot}\NormalTok{.}\AttributeTok{innerHTML} \OperatorTok{=} \StringTok{'<p>hello</p>'}\OperatorTok{;}
 \OperatorTok{\}}
\OperatorTok{\}}
\end{Highlighting}
\end{Shaded}

The new global method \texttt{attachShadow} adds a new document root to
the \texttt{HelloWorld} which has the same properties as a normal DOM.
Therefore, invoking \texttt{innerHTML} method would fill the new
document (fragment) with some arbitrary content. Note that
\texttt{shadowRoot} is marked as \textbf{open} which ensures that some
events can bubble out and outside JS can reach in the new root. Nested
children nodes and other content in the light DOM are ``shadowed'' by
the new root and must be invited in by so called \texttt{slots}.

\subsubsection{Slots}\label{slots}

Contradicting to the simplified \texttt{HelloWorld} example, a
\emph{shadow DOM} shouldn't contain any \sout{valuable} content. While
technical possible any change of an element would require deeply nested
calls from the \emph{light DOM} to the \emph{shadow DOM} to update the
element in place. That's why \emph{shadow DOM} should be perceived more
as \textbf{static HTML template} and provide therefore a kind of
internal frame for the render engine. \texttt{Slots} are placeholders
for \emph{light DOM} nodes used to mark the endpoints in question.

Technically, the \emph{light DOM} nodes are not moved inside the
\emph{shadow DOM}. Their just rendered in place. It's an subtle but
important difference towards handling a node. JS behaviour and CSS
styles applied in the \emph{light DOM} will still be valid in the
\emph{shadow DOM}. The render engine literally taking the nodes and
putting them inside the \texttt{slot}. This procedure is commonly
referred as \textbf{flattening} of the DOM trees.

\paragraph{Named slots}\label{named-slots}

A named slot is the preferable way for clear code organization. Taking
for example
\texttt{\textless{}slot\ name="hello"\textgreater{}Drop\ me\ a\ "hello"\ node\textless{}/slot\textgreater{}}
targets all direct \emph{light DOM} child nodes of the hosting node
matching the slot name like
\texttt{\textless{}div\ slot="hello"\textgreater{}\textless{}/div\textgreater{}}.
Writing a little documentation inside the
\texttt{\textless{}slot\textgreater{}} tag is considered as a good
practice as it will be rendered only if no matching \emph{light DOM}
node is available. This functionality makes a \emph{custom element}
pretty much self-explanatory.

\paragraph{Unnamed slots}\label{unnamed-slots}

Inside a so-called \emph{default slot} which looks like
\texttt{\textless{}slot\textgreater{}Unnamed\ content\ goes\ here\textless{}/slot\textgreater{}},
the render engine expands all direct \emph{light DOM} children without a
\texttt{slot} attribute. In case of multiple default slots, the first
slot takes it all.

\subsubsection{Styling}\label{styling}

As mentioned in the last section, there is a distinct difference about
the nature of nodes. Nodes declared and rendered exclusively in the
\emph{shadow DOM} are not affected by any styling from outside. Nodes
which are declared outside and distributed via \texttt{slots} will be
styled in the \emph{light DOM} and can be additionally painted in the
\emph{shadow DOM} through the new CSS-Selector \texttt{::slotted()}.

Note that styles from the outside have an higher specify than styles
assigned after distribution. Therefore it is generally a good advice to
minimize the global stylings to some base styling for uniformity of the
web site while leaving the specific stylings to the component. Due to
the cascading nature of CSS, styles will still ``bleed in'' from
ancestors to the \emph{light DOM} nodes.

Regarding the importance style encapsulation, a couple of new CSS rules
emerged that are exclusively targeting the \emph{shadow DOM}. The table
below outlines styling possibilities for the use INSIDE the \emph{shadow
DOM}:

\begin{itemize}
\tightlist
\item
  ::slotted(selector)\\
  Applies to distributed nodes and repaints them after distribution.
  \texttt{Slotted} won't override outsides styles but can complement
  them with unset style rules.
\item
  :host\\
  The host property will add styles or change inherited ones inside
  shadow DOM. Using \texttt{all:\ initial;} will ensure browser defaults
  only.
\item
  :host(condition)\\
  Like the previous one this node will style the shadow DOM but this
  time based on attributes/conditions assigned to the hosting node.
\item
  :host-context(condition)\\
  Like the previous one this node will style the shadow DOM but will
  look after context set at the host node or even at the host ancestor.
\end{itemize}

Using the \emph{functional selector} of \texttt{:host()} or even the
only-functional \texttt{:host-contest()} allows the creation of
\textbf{context-aware custom elements}. A possible use case would be
``theming'' a component (example taken from {[}15{]}):

\begin{Shaded}
\begin{Highlighting}[]
\NormalTok{> index.html}
\KeywordTok{<body}\OtherTok{ class=}\StringTok{"darktheme"}\KeywordTok{>}
  \KeywordTok{<fancy-tabs>}
    \NormalTok{...}
  \KeywordTok{</fancy-tabs>}
\KeywordTok{</body>}
\end{Highlighting}
\end{Shaded}

\begin{Shaded}
\begin{Highlighting}[]
\NormalTok{> fancy-tabs shadowRoot}
\NormalTok{<style>}
\DecValTok{:}\NormalTok{host-context(}\FloatTok{.darktheme}\NormalTok{) }\KeywordTok{\{}
\ErrorTok{ } \KeywordTok{color:} \DataTypeTok{white}\KeywordTok{;}
\ErrorTok{ } \KeywordTok{background:} \DataTypeTok{black}\KeywordTok{;}
\KeywordTok{\}}
\NormalTok{</style>}
\end{Highlighting}
\end{Shaded}

\subsubsection{JS Behavior}\label{js-behavior}

As mentioned earlier any logic applied to \emph{light DOM} nodes stays
with the node even after redistribution. For the sake of separation of
concerns the business logic should be part of the \emph{custom element}
(the \emph{light DOM}) and not the part of the \emph{shadow DOM}. On the
other hand there are numerous scenarios where JS is just concerned with
\textbf{styling or animation of an element}.In this case it might be
more straightforward to apply JS inside the \emph{shadow DOM} to avoid
mixing with business logic handlers.

Drilling down to a \emph{light DOM} node from an \emph{shadow DOM}
context is not possible with querying the node directly with
\texttt{.querySelector()} or \texttt{.getElementById()} as the node is
not part of the context. To get a distributed node in question it needs
the way over the slot node and call \texttt{slot.assinedNodes()} to
receive an array of distributed node(s) which can be accessed and
manipulated like any other node. Calling \texttt{.assignedNodes()} on an
empty \texttt{slot} returns an empty array.

Wrapping up this section, \emph{shadow DOM} provides a non-hacky way to
create uniform looking \emph{custom elements} and even enhance styling
possibilities without adding much overhead. Still, for smaller
components with only one or two child nodes, just a little styling
and/or no structured redistribution a \emph{shadow DOM} might be to hard
to reason about. Eventually it all depends on the question of ``how hard
is it to implement it without shadow DOM'' - which can't be answered
universally. For a more in-depth guide, Google Engineer Eric Bidelman
wroten a great primer on \emph{shadow DOM}{[}15{]}.

So far, there is still a missing link between \emph{light DOM} and
\emph{shadow DOM}. The observant reader may have already noticed the
weak point in the \texttt{HelloWorld} example: how to ``vitalize'' the
\emph{shadow DOM}. Recapturing the last \texttt{HelloWorld} example a
string of markup was assigned to the \texttt{shadowRoot.innerHTML}
property. While it works perfectly fine in this simple case, a string of
markup is rather cumbersome and error-prone and doesn't scale well. When
putting quotes inside another quotes things break quickly. It makes the
life hard for developers to work with it because it requires manual
indentation and is out of syntax highlighting. That's the time templates
come into play.

\subsection{\texorpdfstring{HTML Templates
\href{https://www.w3.org/TR/html5/scripting-1.html\#the-template-element}{(w3c)}}{HTML Templates (w3c)}}\label{html-templates-w3c}

Among all other new standards \emph{HTML templates} are the most mature
and adopted standard in the browser environment. All major browsers,
except from Internet Explorer, support this standard.

One core concept in templates is efficiency: Whatever dropped inside a
\texttt{template} tag \sout{bucket} will be parsed on runtime - but not
constructed into the \emph{content tree}. It remains plain HTML Markup
sitting somewhere in the document until the time of activation.

Activation typically takes four steps:

\begin{enumerate}
\def\labelenumi{\arabic{enumi}.}
\tightlist
\item
  \textbf{Querying the template node in question}\\
  const node = document.querySelector(`template');
\item
  \textbf{Parsing the content and preparing the templates' content}\\
  const content = node.content;\\
  -\textgreater{} Returns a \emph{DocumentFragment} object. Handling is
  straight forward content.querySelector(`img').src = `logo.png';
\item
  \textbf{Optional: Cloning the \emph{DocumentFragment} for multiple
  use}\\
  const clone = content.cloneNode(``deep'');
\item
  \textbf{Appending the clone/original to destination}\\
  document.body.appendChild(clone);
\end{enumerate}

As easy and minimal \emph{HTML templates} are, they're missing out a
crucial feature other template implementations usually have. As
templates are basically just dump containers for HTML Markup, there is
no way to define some logic as \textbf{placeholders} where content
should appear. Of course, with heavy use of JS things could be modeled
this way. The idiomatic way tends more towards a \emph{Shadow DOM \&
HTML templates} symbiosis.

\begin{Shaded}
\begin{Highlighting}[]
\NormalTok{> index.html}
\KeywordTok{<hello-world>}
 \KeywordTok{<p}\OtherTok{ id=}\StringTok{"sendto"}\OtherTok{ slot=}\StringTok{"placeholder"}\KeywordTok{>}
  \NormalTok{Hello World Web Component  }
 \KeywordTok{</p>}
\KeywordTok{</hello-world>}

\CommentTok{<!-- COMPONENT STARTS HERE -->}
\KeywordTok{<template}\OtherTok{ id=}\StringTok{"hello"}\KeywordTok{>}
 \CommentTok{<!-- STYLES -->}
 \KeywordTok{<style>}
  \FloatTok{#stylewrapper} \KeywordTok{\{}
   \KeywordTok{font-weight:} \DataTypeTok{bold}\KeywordTok{;}
   \KeywordTok{color:} \NormalTok{orange}\KeywordTok{;}
  \KeywordTok{\}}
 \KeywordTok{</style>}
 \CommentTok{<!-- CONTENT -->}
 \KeywordTok{<div}\OtherTok{ id=}\StringTok{"stylewrapper"}\KeywordTok{>}
  \KeywordTok{<slot}\OtherTok{ name=}\StringTok{"placeholder"}\KeywordTok{>}
   \NormalTok{Named placeholder}
  \KeywordTok{</slot>}
 \KeywordTok{</div>}
\KeywordTok{</template>}

\KeywordTok{<script>}
 \CommentTok{// Switched to anonymous class notation}
 \CommentTok{// for keeping associated code together.}
 \VariableTok{customElements}\NormalTok{.}\AttributeTok{define}\NormalTok{(}\StringTok{'hello-world'}\OperatorTok{,}
  \KeywordTok{class} \KeywordTok{extends} \NormalTok{HTMLElement }\OperatorTok{\{}
   \AttributeTok{constructor}\NormalTok{() }\OperatorTok{\{}
    \KeywordTok{super}\NormalTok{()}\OperatorTok{;}
    \KeywordTok{this}\NormalTok{.}\AttributeTok{attachShadow}\NormalTok{(}\OperatorTok{\{}\DataTypeTok{mode}\OperatorTok{:} \StringTok{'open'}\OperatorTok{\}}\NormalTok{)}\OperatorTok{;}
    \KeywordTok{const} \NormalTok{helloTemplate }\OperatorTok{=} \VariableTok{document}\NormalTok{.}\AttributeTok{querySelector}\NormalTok{(}\StringTok{'#hello'}\NormalTok{)}\OperatorTok{;}
    \KeywordTok{this}\NormalTok{.}\VariableTok{shadowRoot}\NormalTok{.}\AttributeTok{appendChild}\NormalTok{(}\VariableTok{helloTemplate}\NormalTok{.}\AttributeTok{content}\NormalTok{)}\OperatorTok{;}
   \OperatorTok{\}}
  \OperatorTok{\}}\NormalTok{)}\OperatorTok{;}
\OperatorTok{<}\SpecialStringTok{/script>}
\end{Highlighting}
\end{Shaded}

The updated \texttt{HelloWorld} component looks already pretty mature.
It combines all the previous mentioned standards into one blob of HTML.
\emph{Custom Elements} serves the logic, \emph{Shadow DOM} scopes the
styles and \emph{HTML Templates} efficiently glues DOM and \emph{Shadow
DOM} together. This separation of concerns comes with a huge gain in
flexibility. In a real world scenario \texttt{HelloWorld} would
contain/reference multiple \emph{HTML Templates} and could switch them
around without any fuss. A developer might to split up templates into
\textbf{STYLE} templates and \textbf{CONTENT} templates to increase
reusability even further.

The last standard in the row of four is not concerned with the internals
of a \emph{web component}. \emph{HTML Imports} serves the need for an
efficient distribution mechanism of components.

\subsection{\texorpdfstring{HTML Imports
\href{https://www.w3.org/TR/html-imports/}{(w3c)}}{HTML Imports (w3c)}}\label{html-imports-w3c}

Importing the \texttt{HelloWorld} component is a one-liner:

\begin{Shaded}
\begin{Highlighting}[]
\KeywordTok{<link}\OtherTok{ rel=}\StringTok{"import"}\OtherTok{ href=}\StringTok{"Hello.html"}\OtherTok{ async}\KeywordTok{>}
\end{Highlighting}
\end{Shaded}

The \texttt{async} flag is optional but like in any other fetching
event, strongly recommended. Once the imported HTML document comes into
scope, activation follows a very similar pattern like the aforementioned
\emph{HTML templates}:

\begin{enumerate}
\def\labelenumi{\arabic{enumi}.}
\item
  \textbf{Querying the link node}
\item
  \textbf{Parsing the content and preparing the render}\\
  const content = linknode.import; -\textgreater{} Unlike the \emph{HTML
  template} the content a fully equipped document object.
\item
  \textbf{Optional: Cloning some nodes for multiple use}
\item
  \textbf{Appending the clone/original to destination}
\end{enumerate}

This again is the imperative way to handle a generic \emph{HTML Import}.
In the declarative world of \emph{web components} a component is
activated, parsed and anchored solely by its' tag name
\texttt{\textless{}hello-world\textgreater{}\textless{}/hello-world\textgreater{}}.
Preliminary, the component needs proper configuration. The next section
will elaborate the right configuration and composition of a component to
work out-of-the-box.

Despite from being just a practical document importer \emph{HTML
imports} acts like a fully fledged dependency manager for the browser.
Multiple resources, ranging from stylesheets, scripts, documents, media
files and even other \texttt{imports} can be grouped together in a
logical \texttt{import} statement. Internally, the browser engine keeps
track for every imported resource so it won't be loaded twice. The
inherent complexity is in fact a stumbling block for wider browser
adoption. Currently only Googles blink web engine supports \emph{HTML
Imports} as they are the driving force behind the \emph{web components}
spec in general. Mozilla and Apple imposed distaste for \emph{HTML
Imports} as a whole. One reason for this can be found in the
incompatibility of the spec with the upcoming \emph{ES6 module
loader}.\footnote{https://hacks.mozilla.org/2014/12/mozilla-and-web-components/}

Despite the discrepancies among browser vendors \emph{HTML Imports}
should still be part of the paper as no other native browser technology
can bundle up CSS, JS and HTML that efficient.

\subsection{\texorpdfstring{Appendix A: Custom Events
\href{https://dom.spec.whatwg.org/\#interface-customevent}{(whatwg)}}{Appendix A: Custom Events (whatwg)}}\label{appendix-a-custom-events-whatwg}

Events are first-class citizens in the browser world and \emph{Custom
Events} are no exception. The \emph{Custom Elements} interface is part
of the DOM since years but with the rise of \emph{Custom Elements} they
will most likely become an indispensable building block of \emph{web
components}.

\begin{Shaded}
\begin{Highlighting}[]
\NormalTok{> index.html}
\KeywordTok{<hello-world>}
 \KeywordTok{<button>}\NormalTok{Launch CustomEvent}\KeywordTok{</button>}
\KeywordTok{</hello-world>}
\CommentTok{<!-- COMPONENT STARTS HERE -->}
\KeywordTok{<script>}
 \VariableTok{customElements}\NormalTok{.}\AttributeTok{define}\NormalTok{(}\StringTok{'hello-world'}\OperatorTok{,} 
 \KeywordTok{class} \KeywordTok{extends} \NormalTok{HTMLElement }\OperatorTok{\{}
  \AttributeTok{constructor}\NormalTok{() }\OperatorTok{\{}
     \KeywordTok{super}\NormalTok{()}\OperatorTok{;}
   \CommentTok{// Craft a CustomEvent e}
     \KeywordTok{const} \NormalTok{e }\OperatorTok{=} \KeywordTok{new} \AttributeTok{CustomEvent}\NormalTok{(}\StringTok{'hello-world'}\OperatorTok{,} \OperatorTok{\{}
    \DataTypeTok{bubbles}\OperatorTok{:} \KeywordTok{true}\OperatorTok{,} \CommentTok{//important!}
    \DataTypeTok{detail}\OperatorTok{:} \StringTok{'Contains string or object'}
   \OperatorTok{\}}\NormalTok{)}\OperatorTok{;}
   \CommentTok{// Launch e on child button click}
   \KeywordTok{this}\NormalTok{.}\AttributeTok{addEventListener}\NormalTok{(}\StringTok{'click'}\OperatorTok{,} \NormalTok{click }\OperatorTok{=>} \OperatorTok{\{}
    \KeywordTok{this}\NormalTok{.}\AttributeTok{dispatchEvent}\NormalTok{(e)}
    \VariableTok{click}\NormalTok{.}\AttributeTok{stopPropagation}\NormalTok{()}\OperatorTok{;}
   \OperatorTok{\}}\NormalTok{)}\OperatorTok{;}
   \CommentTok{// Catch e. Typically done by some parent}
     \CommentTok{// node. Message in detail property}
   \KeywordTok{this}\NormalTok{.}\AttributeTok{addEventListener}\NormalTok{(}\StringTok{'hello-world'}\OperatorTok{,} \NormalTok{e }\OperatorTok{=>} \OperatorTok{\{}
    \VariableTok{console}\NormalTok{.}\AttributeTok{log}\NormalTok{(}\VariableTok{e}\NormalTok{.}\AttributeTok{detail}\NormalTok{)}
   \OperatorTok{\}}\NormalTok{)}\OperatorTok{;}
  \OperatorTok{\}}
 \OperatorTok{\}}\NormalTok{)}\OperatorTok{;}
\OperatorTok{<}\SpecialStringTok{/script>}
\end{Highlighting}
\end{Shaded}

Naming events after the emitting tag makes the API almost
self-explanatory. The \texttt{detail} property can be loaded with
primitives as well as objects. The further sections will elaborate a
feasible architecture for building a scaling microservice architecture.

Chaining and aggregating events from child nodes should be practiced and
exercised quiet frequently. As mentioned earlier in the \emph{Custom
Elements} section, the pattern of \textbf{extending native elements}
should be somewhat dismissed as it may never implemented outside of
Chrome. Nevertheless, it is possible to create own kind of quasi native
buttons when chaining a \emph{CustomEvent} directly after the native
click event.

Another typical \emph{custom element} use case can be as an actor on
(native) child elements. In this case, the \emph{Custom Element} catches
events from children, buffers or rebuild them and eventually fires an
event towards the document root.

Unfortunately events only work ``upstream'' towards parent nodes. Still
the web platform offers plenty of possibilities to talk back to child
nodes.

\subsection{\texorpdfstring{Appendix B: Web Worker
\href{https://html.spec.whatwg.org/multipage/workers.html}{(whatwg)}}{Appendix B: Web Worker (whatwg)}}\label{appendix-b-web-worker-whatwg}

Like \emph{Custom Events}, \emph{Web Workers} had been around for a long
time and therefore enjoy full support among major browsers. They emerged
at around 2009 when discussions about browser performance was still in
the early days. Nevertheless, the addressed problem of \emph{Web
Workers} is a fundamental language problem of JS itself.

JS runs in a single-threaded language environment. Every script in the
browser environment, from handling UI events to query and process larget
amounts of API data and manipulating the DOM, runs on the same
thread{[}16{]}. Putting a lot of work to the single main thread can slow
down the web service significantly. From time to time scripts can block
or fail for whatever reason which leads to a frozen or crashed UI. A
worker can overcome the bottleneck of the single-threaded nature with
spawning new \textbf{background threads} which allows the UI to stay
responsive even when computation-heavy tasks should be carried out.
Furthermore, a worker thread adds a performance advantage embracing the
multi core CPU architecture most devices running on today. To grasp the
full potential of workers, a reader might dive deeper into the Angular 2
architecture, where most of the application layer is abstracted from the
main rendering thread into worker threads.\footnote{\href{https://docs.google.com/document/d/1M9FmT05Q6qpsjgvH1XvCm840yn2eWEg0PMskSQz7k4E}{Angular
  2 Rendering Architecture}}

A \emph{Web Worker} spawns a new background thread where scripts can run
concurrent to the main thread. Usually a worker is loaded from a workers
dedicated file to embrace this kind of separation.

\begin{Shaded}
\begin{Highlighting}[]
\KeywordTok{const} \NormalTok{worker }\OperatorTok{=} \KeywordTok{new} \AttributeTok{Worker}\NormalTok{(}\StringTok{'worker.js'}\NormalTok{)}\OperatorTok{;}
\end{Highlighting}
\end{Shaded}

After initialization a worker communicates over a simple \textbf{message
based interface} with the main thread.

\begin{Shaded}
\begin{Highlighting}[]
\OperatorTok{>} \VariableTok{main}\NormalTok{.}\AttributeTok{js}
\CommentTok{// Send to worker}
\VariableTok{worker}\NormalTok{.}\AttributeTok{postMessage}\NormalTok{(}\StringTok{'Hello World'}\NormalTok{)}\OperatorTok{;}
\CommentTok{// Receive msg from worker}
\VariableTok{worker}\NormalTok{.}\AttributeTok{addEventListener}\NormalTok{(}\StringTok{'message'}\OperatorTok{,} \NormalTok{e }\OperatorTok{=>}
 \VariableTok{console}\NormalTok{.}\AttributeTok{log}\NormalTok{(}\StringTok{'Worker said: '}\OperatorTok{,} \VariableTok{e}\NormalTok{.}\AttributeTok{data}\NormalTok{))}\OperatorTok{;}
\end{Highlighting}
\end{Shaded}

\begin{Shaded}
\begin{Highlighting}[]
\OperatorTok{>} \VariableTok{worker}\NormalTok{.}\AttributeTok{js}
\CommentTok{// Receive msg and echo back}
\KeywordTok{this}\NormalTok{.}\AttributeTok{addEventListener}\NormalTok{(}\StringTok{'message'}\OperatorTok{,} \NormalTok{e }\OperatorTok{=>}
 \KeywordTok{this}\NormalTok{.}\AttributeTok{postMessage}\NormalTok{(}\StringTok{"Echo "} \OperatorTok{+} \VariableTok{e}\NormalTok{.}\AttributeTok{data}\NormalTok{))}\OperatorTok{;}
\end{Highlighting}
\end{Shaded}

\section{Building an browsernative
microservice}\label{building-an-browsernative-microservice}

After getting confidence in microservice principles and technical
background the paper should bring them both together to form a
\emph{browsernative microservices}. Needless to say the described
example is overall simplified to only illustrate the connection between
browsernative technologies and microservice patterns. Googles library
Polymer is a good place for learning about web components in depth and
make use of their simple command line tools. One of their most famous
proof of concept is the so-called
\href{https://shop.polymer-project.org/}{Polymer Shop} which is a
fully-fledged online shop nested within a single root element
\texttt{\textless{}shop-app\textgreater{}}. This app made of several
main views and many more invisible custom elements for routing, service
worker caching, theming, etc. The whole shop runs as a single
application only fetching and sending resources and switching views.
Let's assume we work in a sales engineering team of the Polymer Shop and
need to rebuild the checkout microservice.

The current checkout can be found at
https://shop.polymer-project.org/checkout. Currently the checkout is a
single, 671 lines of code long Polymer component incorporation all
required fields for sign in, shipping, billing and summarizing the
order. In the spirit of microservices we will split up the component to
their smaller components. The shopping cart data is pulled out of a
local storage JSON entity set up previously by another custom-element.
Another in-memory opportunity is storing the data in the build-in
database IndexedDB.

\subsection{Checkout microservice}\label{checkout-microservice}

Like a usual checkout this example has 3 to 4 steps:

\begin{enumerate}
\def\labelenumi{\arabic{enumi}.}
\tightlist
\item
  Sign in or Sign up
\item
  Shipping details
\item
  Payment details
\item
  Review and place order
\end{enumerate}

Translated into a raw \textbf{Custom Element} HTML structure, the
top-level microservice might look like the following snippet:

\begin{Shaded}
\begin{Highlighting}[]
\NormalTok{>shop-checkout.html}
\KeywordTok{<shop-checkout>}
  \KeywordTok{<sign-in></sign-in>}
  \KeywordTok{<shipping-details></shipping-details>}
  \KeywordTok{<payment-details></payment-details>}
  \KeywordTok{<place-order></place-order>}
\KeywordTok{</shop-checkout>}
\end{Highlighting}
\end{Shaded}

Yet already we see the simplicity around web components as they persue a
clear structure. Each of the child components should act independently
over other child nodes utilizing the loose coupling principle. Each
child ships all the HTML, CSS and JS code needed to fulfil its work
following the high cohesion principle. Each component may contain
different views to accommodate different bounded contexts resulting from
different devices. And last but not least, all of them communicate over
an unobstrusive message bus via the service root component
\texttt{shop-checkout}.

Before diving deeper into implementation, its worth to clarify the
\textbf{architectural pattern} behind components. Any reader of the
paper came across ReactJS / Redux, the concept of components may look
familiar. Dan Abramov, the creator of Redux, once defined a simple
dichotomous pattern for creating UI components. Firstly, he came up with
the idea of \textbf{presentational components} only related with the
concern about \emph{how things look}. This component literally doesn't
know anything about the service in question which makes the component
highly flexible and reusable. They are controlled solely from the
outside, receiving data and dispatching unbiased events on user
interaction.{[}17{]} Most probably every presentational component
embodies more HTML/CSS markup and less JS code. It should encapsulate
its styles from bleeding out and protect its styles being overwritten.
Furthermore, it may contain several templates to change it's look on
different demands.

Secondly, Abramov described components he refers as \textbf{containers}.
A container component is concerned with \emph{how things work}.{[}17{]}
Containers acts as invisible wrappers around presentational components
acting more in the sense of UNIX filters. Their job is to fetch data
from child nodes, aggregating events, interacting with the model and
push state back to the presentational components. Consequently, they
might contain more JS and less, if any, HTML markup. We probably don't
need to utilize ShadowDOM as no styles are involved.

Last but not least, there are \textbf{build-in components}, which is
every traditional HTMLElement. They are solely controlled and styled
from the outside and must therfore wrapped in presentationals and/or
containers.

Lets start the service description top-down with the \textbf{root
container}.

\subsubsection{Root container}\label{root-container}

Messages from child nodes will just forwarded to the worker and answers
passed back to the child nodes. Therefore, the basic microservice
communication looks like this

\begin{verbatim}
VIEW root           |               MODEL worker
                    |
Event  --------Event msg--------->    Msg
    |               |               Handler
    |               |                   |
Idle                |               Effekt
    |               |                   |
  Msg               |                   |
Handler <-----Action msg----------  Action                                      
                    |                   
                    |
\end{verbatim}

Effects are yielded by the asynchronous operation of messages and create
actions returned to sender. Effects can be created by external messages,
like subscription to an WebSocket, too. Effects may be created by
fetching additional resources from the server. The VIEW side of the root
element forwards all the messages and won't be bothered about the
content.

\section{Thinking further}\label{thinking-further}

endgeräte werden besser

multicore

query languages from handheld

Most obvious ist the gathering of all related code under the umbrellar
of a single HTML tag.

Secondly, the sub-standard \emph{custom elements} introduces so called
lifecycle methods and a getter/setter interface exposing the
functionality to the developer. Event handling, for example, can be
registered in place which is much more declarative than assigning event
listeners from the outside. Of course, this events can be pushed down to
nested tags, allowing an increasingly granular system design. This
approach will be explained further in the upcoming sections.

Following this logic any company, whether it is web-related or not,
should be devided in units grouped around a destinct business service to
optimise the workflow. Fowler and Lewis outlines this approach as an
``alignment of business capabilities''{[}4{]} While this kind of
structure may be true for companies like Google or Amazon, there is a
vast majority of companies developing for the web which are grouped
around tasks.{[}8{]} A very common structure is formed by the technology
stack (UX Designers, Frontend- \& Backend Developers) or by separating
teams along the product lifecycle (development, testing, deployment).

Advocators from the microservice approach propose a different model.
best described by . Web components are one (but important) way to tie up
those diciplines as one component can host a single independent business
service. Combined with a flexible backend service these components can
be huge gain over the cumbersome functional organizational approach.

dichotome Pattern

Ein üblicher eventgesteuerter Webservice setzt sich aus
unterschiedlichsten Komponenten zusammen, die wiederum
unterschiedlichste Eventlistener \& -emitter in sich subsummieren. Diese
inhärente Komplexität verlangt geradezu nach einer klaren,
deterministischen Struktur des Webservices, die das Zusammenspiel
orchestriert. In der Analogie des Orchesters gesprochen, benötigt der
Webservice (oder sogar die gesamte Webapplikation) einen Dirigenten, der
für die Steuerung verantwortlich ist.

http://alistair.cockburn.us/Hexagonal+architecture

MVC Pattern

Pure frontend vs heavy backend

\hypertarget{refs}{}
\hypertarget{ref-Filloux2016}{}
{[}1{]} F. Filloux, ``Bloated html, the best and the worse,'' 2016
{[}Online{]}. Available:
\url{https://mondaynote.com/bloated-html-the-best-and-the-worse-cac6eb06496d}

\hypertarget{ref-Baldwin2006}{}
{[}2{]} C. Y. Baldwin and K. B. Clark, ``Modularity in the Design of
Complex Engineering Systems,'' in \emph{Complex engineered systems:
Science meets technology}, D. Braha, A. A. Minai, and Y. Bar-Yam, Eds.
Berlin, Heidelberg: Springer Berlin Heidelberg, 2006, pp. 175--205
{[}Online{]}. Available:
\href{http://dx.doi.org/10.1007/3-540-32834-3\%7B/_\%7D9}{http://dx.doi.org/10.1007/3-540-32834-3\{\textbackslash{}\_\}9}

\hypertarget{ref-Newman2015}{}
{[}3{]} S. Newman, \emph{Building microservices}. O'Reilly Media, Inc.,
2016 {[}Online{]}. Available:
\url{http://www.ebook.de/de/product/22539693/sam_newmann_building_microservices.html}

\hypertarget{ref-Fowler2014}{}
{[}4{]} M. Fowler and J. Lewis, ``Microservices: A definition of this
new architectural term,'' Jan. 2014 {[}Online{]}. Available:
\url{http://www.martinfowler.com/articles/microservices.html}

\hypertarget{ref-Martin}{}
{[}5{]} R. C. Martin, ``The single responsibility principle.''
{[}Online{]}. Available:
\url{http://programmer.97things.oreilly.com/wiki/index.php/The_Single_Responsibility_Principle}

\hypertarget{ref-Rauschmayer2015}{}
{[}6{]} D. A. Rauschmayer, ``Tree-shaking with webpack 2 and babel 6.''
2015 {[}Online{]}. Available:
\url{http://www.2ality.com/2015/12/webpack-tree-shaking.html}

\hypertarget{ref-Conway1968}{}
{[}7{]} M. E. Conway, ``How do committees invent?'' 1968 {[}Online{]}.
Available: \url{http://www.melconway.com/Home/Committees_Paper.html}

\hypertarget{ref-Issa2016}{}
{[}8{]} B. Issa, ``The way of the web.'' Polymer Summit 2016, Oct-2016
{[}Online{]}. Available:
\url{https://www.youtube.com/watch?v=8ZTFEhPBJEE}

\hypertarget{ref-Lawson2016}{}
{[}9{]} N. Lawson, ``Progressive enhancement isn't dead, but it smells
funny.'' 2016 {[}Online{]}. Available:
\url{https://nolanlawson.com/2016/10/13/progressive-enhancement-isnt-dead-but-it-smells-funny/}

\hypertarget{ref-Rota2015}{}
{[}10{]} A. Rota, ``React.js conf 2015 - the complementarity of react
and web components.'' 2015 {[}Online{]}. Available:
\url{https://www.youtube.com/watch?t=124\&v=g0TD0efcwVg}

\hypertarget{ref-Buchner2016}{}
{[}11{]} D. Buchner, ``Demythstifying web components,'' 2016
{[}Online{]}. Available:
\url{http://www.backalleycoder.com/2016/08/26/demythstifying-web-components/}

\hypertarget{ref-vanKesteren2014}{}
{[}12{]} A. van Kesteren, ``Mozilla and web components: Update,'' 2014
{[}Online{]}. Available:
\url{https://hacks.mozilla.org/2014/12/mozilla-and-web-components/}

\hypertarget{ref-Bidelman2016}{}
{[}13{]} E. Bidelman, ``Custom elements v1: reusable web components.''
2016 {[}Online{]}. Available:
\url{https://developers.google.com/web/fundamentals/primers/customelements/}.
{[}Accessed: 01-Dec-2016{]}

\hypertarget{ref-HTML}{}
{[}14{]} \emph{HTML living standard --- last updated 11 january 2017}.
{[}Online{]}. Available: \url{https://html.spec.whatwg.org/multipage/}

\hypertarget{ref-Bidelman2016shadow}{}
{[}15{]} E. Bidelman, ``Shadow dom v1: Self-contained web components.''
2016 {[}Online{]}. Available:
\url{https://developers.google.com/web/fundamentals/getting-started/primers/shadowdom}

\hypertarget{ref-Bidelman2010}{}
{[}16{]} E. Bidelman, ``The basics of web workers.'' 2010 {[}Online{]}.
Available: \url{https://www.html5rocks.com/en/tutorials/workers/basics/}

\hypertarget{ref-Abramov2015}{}
{[}17{]} D. Abramov, ``Presentational and Container Components --
Medium.'' 2015 {[}Online{]}. Available:
\url{https://medium.com/@dan_abramov/smart-and-dumb-components-7ca2f9a7c7d0}.
{[}Accessed: 01-Dec-2016{]}

\end{document}
